%% -*- coding:utf-8 -*-
\chapter{Ring extensions, norms and traces, reduction modulo $p$}
We build a tool for finding elements in Galois groups, learning to use
the reduction modulo $p$. For this, we have to talk a little bit about
integral ring extensions and also about norms and traces.


\section{Integral elements over a ring}

Let $P \in \mathbb{Z}\left[X\right]$. We want to know what is
$Gal\left(P\right)$. Just a reminder that $Gal\left(P\right) =
Gal\left(K/\mathbb{Q}\right)$ where $K$ is a
\mynameref{def:splittingfield} of $P$. We have already done the work for
several types of polynomials: \mynameref{def:cyclotomicpolynomial}s,
\mynameref{sec:kummerextension} and so on.

Sometimes, if our polynomial is a kind of combination of them, then the
explicit information about the roots helps to calculate the Galois
group. For instance if we have polynomial $X^5 - 2$ we know it's
roots: $\sqrt[5]{2}, j^k \sqrt[5]{2}$, where
$j= e^{\frac{2 \pi i}{5}}, 1 \le k \le 4$. Now we have a lot about the
\mynameref{def:galoisgroup}. If $K$ is the splitting field of the
polynomial then we have the following towers:

  \begin{tikzpicture}[descr/.style={fill=white,inner sep=2.5pt}]
    \matrix (m) [matrix of math nodes, row sep=3em,
      column sep=3em]
            { & \mathbb{Q}\left(j\right) & \\
              \mathbb{Q} & & \mathbb{Q}\left(\sqrt[5]{2}, j\right) = K\\
              & \mathbb{Q}\left(\sqrt[5]{2}\right) & \\ };
            \path[->,font=\scriptsize]
            (m-2-1) edge node[descr] {$ 4 $} (m-1-2)
            (m-1-2) edge node[descr] {$ 5 $} (m-2-3)
            (m-2-1) edge node[descr] {$ 5 $} (m-3-2)
            (m-3-2) edge node[descr] {$ 4 $} (m-2-3);
  \end{tikzpicture}
  
From that we know we can conclude that it follows that our Galois
group, contains a normal cyclic subgroup of a order 5:
$\mathbb{Z}/5\mathbb{Z}$.
\footnote{
  We have the following towers:
  $\mathbb{Q} \subset \mathbb{Q}\left(j\right) \subset
  \mathbb{Q}\left(j, \sqrt[5]{2}\right) = K$, where
  $\mathbb{Q}\left(j\right)$ and $\mathbb{Q}\left(j,
  \sqrt[5]{2}\right)$ are \mynameref{def:galoisextension}s.
  By \mynameref{thm:galoiscorrespondence} one can get that
  \[
  Gal\left(K/\mathbb{Q}\left(j\right)\right) \triangleleft
  Gal\left(K/\mathbb{Q}\right) = Gal\left(P\right).
  \]
  As soon as (see theorem \ref{thm:lec5_4}),
  \[
  \left|Gal\left(K/\mathbb{Q}\left(j\right)\right)\right| =
  \left[K:\mathbb{Q}\left(j\right)\right] = 5,
  \]
  the $Gal(P)$ has a normal subgroup of order 5.
}
And then 
the quotient is the Galois group of cyclotomic extension
\footnote{
  The quotient is
  \[
  Gal\left(K/\mathbb{Q}\right)/Gal\left(K/\mathbb{Q}\left(j\right)\right).
  \]
  It has order $20/5=4$ and therefore it is
  $\left(\mathbb{Z}/5\mathbb{Z}\right)^\times$. It is also
  (see claim \ref{claim:galoisquotient})
  $Gal\left(\mathbb{Q}\left(j\right)/\mathbb{Q}\right)$
}
, so this is
$\left(\mathbb{Z}/5\mathbb{Z}\right)^\times$.
So this is a group of
order 20.
\footnote{
  There is a General affine group:
  $Gal(P) \cong GA(1,5) \triangleleft S_5$
  \cite{groupwiki:s5structure, groupwiki:generalaffinegroup}
}
You can show that this 
is noncommutative, and from this exact sequence, you have some
information about it. But what will we do if we don't know the
roots. One of the tool that we will use is the reduction of modulo
prime and this will be the subject of the lecture.

\begin{gapexample}[Galois group]
  Let investigate the \mynameref{def:galoisgroup}
  $Gal\left(K/\mathbb{Q}\right)$ via GAP:
  \begin{verbatim}
    gap> x:=Indeterminate(Rationals,"x");;
    gap> n:=5;;
    gap> i:=GaloisType(x^n-2);;
    gap> g:=TransitiveGroup(n,i);;
    gap> g;
    F(5) = 5:4
    gap> 
  \end{verbatim}
\end{gapexample}

\subsection{Ring extensions}
\begin{definition}[Integral element]
  Let $A$ be an \mynameref{def:integraldomain}, i.e. a ring without zero
  divisors and let $B$ is an extension of $A$. The element $\alpha
  \in B$ is called integral over $A$ if $\alpha$ is a root of a
  \mynameref{def:monicpolynomial} $P \in A\left[X\right]$.

  So one can write the following relation
  \[
  \alpha^n + a_{n-1}\alpha^{n-1} + \dots + a_1 \alpha + a_0 = 0, a_i
  \in A.
  \]
  \label{def:integralelement}
\end{definition}

\begin{example}
  $\frac{1}{2}$ is not integral element over $\mathbb{Z}$ but
  $\sqrt{2}$ is an \mynameref{def:integralelement} over $\mathbb{Z}$.

  This is because the polynomial in the definition
  \ref{def:integralelement} is monic i.e. the leading coefficient is
  1. 
\end{example}

\begin{lemma}
  The following conditions are equvivalent
  \begin{enumerate}
  \item $\alpha$ is integral over $A$.
  \item $A\left[\alpha\right]$ is a finitely generated $A$-module
    (see definition \ref{def:fgmodule}).
  \item $A\left[\alpha\right] \subset C \subset B$ where $C$ is a a
    finitely generated 
    $A$-module (see definition \ref{def:fgmodule}). I.e.
    $A\left[\alpha\right]$ is contained in a finitely generated $A$-module.
  \end{enumerate}
  \label{lem:lec9_1}
  \begin{proof}
    $1 \to 2 \to 3$ is easy
    \footnote{
      $1 \to 2$ is really easy because the finite set $\{1,\alpha,
      \dots, \alpha^{n-1}\}$ generates
      $A\left[\alpha\right]$ i.e.
      $\forall x \in A\left[\alpha\right], \exists \{x_i\}
      \subset A$ such that $x = \sum_{i=0}^{n-1} x_i \alpha^{i}$.

      $2 \to 3$ is even more easy because $C=A\left[\alpha\right]$ will work.
    } and we will concentrate on $3 \to 1$.

    Let $x_1, \dots, x_r$ generate $C$ as $A$-module then we can write
    \footnote{
      As soon as $\alpha x_i \in C$ is an element of $C$ that can be written
      as a linear combination of
      $x_1, \dots, x_r$ with coefficients from $A$.
    }
    \[
    \alpha x_i = \sum \lambda_{ij}x_j,
    \]
    where $\lambda_{ij} \in A$. Consider the matrix
    $\Lambda = \{\lambda_{ij}\}$ and let $M = \alpha \cdot id -
    \Lambda$. Then
    \[
    M \cdot
    \begin{pmatrix}
      x_1 \\ \vdots \\ x_r
    \end{pmatrix} = 0.
    \]
    Thus (see (\ref{eq:lec4_det}) at theorem
    \ref{thm:structurefinitekalgebra} proof)
    \[
    \det M \cdot
    \begin{pmatrix}
      x_1 \\ \vdots \\ x_r
    \end{pmatrix} = 0.
    \]
    Therefore $\det M \cdot C = 0$ but $1 \in C$ thus $\det M = 0$.
    \footnote{
      There is a question why do we need $\det M \cdot C = 0$, may be
      we can directly write $\det M = 0$? Staff provided a good
      example why it's necessary: Sending generators to zero is not
      sufficient for having zero determinant. For example, consider a
      ring $A$, its ideal $I \subset A$ and $A/I$ as a
      $A$-\mynameref{def:cyclicmodule} i.e. module with one generator.
      If $x \in A$ is the generator then we can just say that $\det M x
      \in I$ and as result $\det M$ not necessary to be zero.
    }
    The equation $\det M = 0$ can be considered as a polynomial with
    $\alpha$ as a root.
    \footnote{
      The polynomial will be a \mynameref{def:monicpolynomial} because
      the matrix element can have $\alpha$ with coefficient equal $1$.
    }
  \end{proof}
\end{lemma}

\section{Integral extensions, integral closure, ring of integers of a
  number field}

\subsection{Integral extensions and integral closure}

\begin{definition}[Integral extension]
  Let $A \subset B$. $B$ is integral over $A$ if $\forall \alpha \in
  B$, $\alpha$ is an \mynameref{def:integralelement} over $A$.  
  \label{def:integralextension}
\end{definition}

The following proposition is not a part of the lectures but it is
required for propositions \ref{prop:lec9_1} and \ref{prop:lec9_2}
proof.
\begin{proposition}
  Let $A$ is a sub-ring of $C$ and
  $\alpha_1, \alpha_2, \dots, \alpha_n \in C$,
  $\alpha_1$ is an integral over $A$,
  $\alpha_2$ is an integral over $A\left[\alpha_1\right]$, and so on,
  $\alpha_n$  is an integral over
  $A\left[\alpha_1, \alpha_2, \dots,   \alpha_{n-1}\right]$ then
  $A\left[\alpha_1, \alpha_2, \dots,\alpha_{n}\right]$ is a finitely
  generated $A$ module.
  \begin{proof}
    Lets proof by induction. The case $n=1$ follows directly from
    lemma \ref{lem:lec9_1}.

    Induction hypothesis gives us that
    $B = A\left[\alpha_1, \alpha_2, \dots,\alpha_{n-1}\right]$ is a finitely
    generated $A$ module and we have to prove that
    $S = A\left[\alpha_1, \alpha_2, \dots, \alpha_{n}\right]$ is a finitely
    generated $A$ module.

    For $B$ as a finitely generated $A$ module we have
    \[
    B = \sum_i A b_i,
    \]
    but $S = B\left[\alpha_n\right]$ is a finitely generated $B$
    module (thanks lemma \ref{lem:lec9_1}) and therefore
    \[
    S = \sum_j B s_j
    \]
    and as result
    \[
    S = \sum_{ij} A b_i s_j =
    \sum_{ij} A k_{ij}
    \]
    where $b_i s_j = k_{ij} \in S$ form a finite set of generators for $S$.
  \end{proof}
  \label{prop:lec9_add}
\end{proposition}

\begin{proposition}
  Let $A \subset B \subset C$. $B$ integral over $A$, $C$ integral
  over $B$ then $C$ is an \mynameref{def:integralextension} over $A$.
  \begin{proof}
    Proof is left as an exercise.     
    \footnote{
      Let $x \in C$ then
      \[
      x^n + b_{n-1} x^{n-1} + \dots + b_1 x + b_0 = 0,
      \]
      where $b_i \in B$. Thus $x$ is an integral element over
      $A\left[b_0, b_1, \dots, b_{n-1}\right]$. We also have that $B$
      is integral over $A$ therefore $b_i$ is an integral over $A$. As
      result we have $b_0$ is integral over $A$, $b_1$ is integral
      over $A\left[b_0\right]$, and so on $x$ is integral over
       $A\left[b_0, b_1, \dots, b_{n-1}\right]$. As result with
      proposition \ref{prop:lec9_add} we have that
      $A\left[b_0, b_1, \dots, b_{n-1}, x\right]$ is a finitely
      generated $A$ module and as
      $A\left[x\right] \subset A\left[b_0, b_1, \dots, b_{n-1},
        x\right]$, the lemma \ref{lem:lec9_1} gives us that $x$ is an
      integral element over $A$.
    }
  \end{proof}
  \label{prop:lec9_1}
\end{proposition}

\begin{proposition}
  $B$ is a finitely generated over $A$ as a module (see definition
  \ref{def:fgmodule}) if and only if
  $B=A\left[\alpha_1, \dots, \alpha_r\right]$ where each $\alpha_i$ is
  an \mynameref{def:integralelement} over $A$.
  \begin{proof}
    Proof is left as an exercise
    \footnote{
      If $\alpha_i$ is an integral over $A$ then it also an integral
      over $A\left[\alpha_1, \alpha_2, \dots, \alpha_{i-1}\right]$ and
      therefore by proposition \ref{prop:lec9_add},
      $A\left[\alpha_1, \alpha_2, \dots, \alpha_r\right] = B$ is finitely
      generated $A$-module.

      Let $B$ is finitely generated $A$ module i.e.
      $B = \sum_{i=1}^r \alpha_i x_i$, where
      $\alpha_i \in B$ - the module generators and $x_i \in A$.
      If we look at any $\alpha_i$ we can notice that
      $A\left[\alpha_i\right] \subset B$ - finitely generated $A$
      module and therefore by lemma \ref{lem:lec9_1},
      $\alpha_i$ is an integral element. We also have (from definition
      \ref{def:fgmodule}) that $\forall b \in B, \exists \{x_i\} \in
      A$ such that $b = \sum_{i=1}^r \alpha_i x_i$. Therefore
      $B = A\left[\alpha_1, \alpha_2, \dots, \alpha_r\right]$.
    }
  \end{proof}
  \label{prop:lec9_2}
\end{proposition}

\begin{proposition}
  Let $A \subset B$. I.e. $B$ is an arbitrary extension of $A$. The
  elements of $B$ which are integral over $A$ form a subring of $B$
  (one calls it as the integral closure of $A$ in $B$).
  \begin{proof}
    Let $\alpha, \beta$ are integral over $A$ then $A\left[\alpha,
      \beta\right]$ - finitely generated $A$-module (see definition
    \ref{def:fgmodule}). This follows directly from lemma
    \ref{lem:lec9_1}. It contains $\alpha + \beta$ and $\alpha \beta$
    and by lemma \ref{lem:lec9_1} the $\alpha + \beta$ and $\alpha
    \beta$ are integral over $A$. But this is exactly we need to
    proof.
    \footnote{
      $A\left[\alpha + \beta\right] \subset A\left[\alpha,
      \beta\right]$ and as soon as $A\left[\alpha,
        \beta\right]$ - finetely generated $A$ module then
      $\alpha+\beta$ is an integral element by lemma \ref{lem:lec9_1}.

      Same result can be got for $\alpha \beta$ from the following
      inclusion:  $A\left[\alpha \beta\right] \subset A\left[\alpha,
      \beta\right]$
    }
  \end{proof}
  \label{prop:lec9_3}
\end{proposition}

\begin{definition}[Integrally closed]
  Let $A \subset B$. $A$ is integrally closed in $B$ if the integral
  closure of $A$ in $B$ equals to $A$.

  $A$ is integrally closed (without mention of any $B$) if it is
  integrally closed in \mynameref{def:fractionfield} $\mathrm {Frac}(A)$.
  \label{def:integrallyclosed}
\end{definition}

\begin{example}
  $\mathbb{Z}$ is \mynameref{def:integrallyclosed}.
  \footnote{
    $\mathrm {Frac}(\mathbb{Z}) = \mathbb{Q}$. An arbitrary element
    $q \in \mathbb{Q}$ is an integral element over
    $\mathbb{Z}$ if and only if $q \in \mathbb{Z}$. For example
    $5 \in \mathbb{Z} \subset \mathbb{Q}$ is integral over $\mathbb{Z}$ as a root of a
    monic polynomial $P = X - 5$. But $\frac{5}{2} \in \mathbb{Q}$ is
    not integral over $\mathbb{Z}$ because it is a root of a non-monic
    polynomial $P = 2 X - 5$.
  }
\end{example}

\begin{remark}
  More generally any
  \mynameref{def:ufd} is \mynameref{def:integrallyclosed}.
  \begin{proof}
    Let $A$ be a \mynameref{def:ufd} and $x \in \mathrm {Frac}(A)$ such
    that $x \ne 0$. So
    $x = \frac{p}{q}$ such that $p,q \in A, \left(p, q\right) = 1$
    (this means no common prime divisor). If $x$ integral over $A$
    then
    \[
    \left(\frac{p}{q}\right)^n + a_{n-1}\left(\frac{p}{q}\right)^{n-1}
    + \dots + a_1 \frac{p}{q} + a_0 = 0.
    \]
    Thus
    \[
    \frac{p^n + q a_n p^{n-1} + q^2 a_{n-1} p^{n-2} + \dots + q^{n-1}
      a_1 p + q^n a_0}{q^n} = 0
    \]
    therefore $q \mid p^n$
    \footnote{
      This is because we have
      \[
      p^n = q \left(-a_n p^{n-1} - q a_{n-1} p^{n-2} - \dots - q^{n-2}
      a_1 p - q^{n-1} a_0\right)
      \]
      and
      \[
      \left(-a_n p^{n-1} - q a_{n-1} p^{n-2} - \dots - q^{n-2}
      a_1 p - q^{n-1} a_0\right) \in A
      \]
    }
    which is in contradiction with
    $\left(p, q\right) = 1$. As result we have that $q$ is invertible
    and therefore $x \in A$.
    \footnote{
      $q^{-1} \in A$ and $x = \frac{p}{q} = p q^{-1} \in A$.
    }
    \end{proof}
\end{remark}

\subsection{Ring of integers in a number field}

\begin{definition}[Number field]
  Let $K$ is a finite extension of $\mathbb{Q}$ i.e.
  $\left[K:\mathbb{Q}\right] < \infty$. In the case $K$ is a number
  field. 
  \label{def:numberfield}
\end{definition}

Let $K$ is a \mynameref{def:numberfield} and $\left[K:\mathbb{Q}\right]
= N$.
\begin{definition}[Ring of integers]
  Let $K$ is a \mynameref{def:numberfield}. The ring of integers
  $O_K \subset K$ is the integral closure of $\mathbb{Z}$
  in $K$. 

  Note: We know that integral closure of $\mathbb{Z}$ in $\mathbb{Q}$
  is $\mathbb{Z}$ but now we consider the closure in $K$ but not in
  $\mathbb{Q}$. 
  \label{def:ringintegers}
\end{definition}

\begin{property}
  \begin{enumerate}
  \item $\forall \alpha \in K, \exists d \in \mathbb{Z} \setminus
    \{0\}$ such that $d\alpha \in O_K$.
  \item If $\alpha \in O_K$ then $P_{min}\left(\alpha,
    \mathbb{Q}\right) \in \mathbb{Z}\left[X\right]$.
  \end{enumerate}
  \begin{proof}
    For the first part lets
    $P_{min}\left(\alpha,
    \mathbb{Q}\right) = X^m + a_{m-1}X^{m-1} + \dots + a_1 X + a_0
    \in \mathbb{Q}\left[X\right]$.

    $\exists d \in \mathbb{Z}$ (the common denominator) such that
    $\forall i: d a_i \in \mathbb{Z}$. So
    $b_i = d^{m-i}a_i \in \mathbb{Z}$ for any $i$. Therefore
    \[
    \left(d \alpha\right)^m + b_{m-1} \left(d \alpha\right)^{m-1} +
    \dots + b_0 = 0.
    \]
    Thus $d \alpha \in O_K$.

    The second part is also easy. If we have such
    $\alpha \in O_K$, it is a
    root of some \mynameref{def:monicpolynomial}
    $Q \in \mathbb{Z}\left[X\right]$.
    Then the $P_{min} \mid Q$. So $Q = P_{min} R$.
    If we pick $P_{min}$ to be monic, then by an
    argument very similar to that of the \mynameref{lem:gauss},
    we conclude  that both $P_{min}, R \in \mathbb{Z}\left[X\right]$.
    \footnote{
      I.e. we can always write
      \[
      Q = m n P_1 R_1 
      \]
      where $P_1, R_1 \in \mathbb{Z}\left[X\right]$. Choose $p$ is a
      prime divisor of $m n$ we can write
      \[
      \bar{P_1}\bar{R_1} = \bar{Q} = 0
      \]
      where $\bar{Q} = Q \mod p$, $\bar{P}_1 = P_1 \mod p$,
      $\bar{R_1} = R_1 \mod p$ thus $\bar{P}_1$ or $\bar{R}_1$ are
      equal to 0. Let $\bar{R}_1 = 0$ thus all coefficients of $R_1$ are
      divided by $p$ i.e. $R_1 = \frac{R_2}{p}$ where
      $R_2 \in \mathbb{Z}\left[X\right]$. Therefore
      \[
      Q = \frac{mn}{p}P_1 R_2
      \]
      Continue this way we can conclude that
      \[
      Q = P_s R_t, 
      \]
      where $P_s, R_t \in \mathbb{Z}\left[X\right]$. As soon as $Q$ is
      monic then both $P_s$ and $R_t$ are monic. Using the fact that
      $P_s = z P_{min}$ where $z \in \mathbb{Z}$ we have
      $P_s\left(\alpha\right) = 0$ and therefore we can
      conclude that $P_s$ is the minimal polynomial. 
    }
  \end{proof}
  \label{property:lec9_2}
\end{property}

\section{Norm and trace}

\subsection{Norms and traces}
(The material was given inside the proof of theorem \ref{thm:lec9_1}
and can be considered as a recall. The remarks
\ref{rem:lec9_embedding} and \ref{rem:lec9_add} are not parts of the
lectures and were given for better understanding the material)

\begin{remark}[$K$ embedding of $E$ into the algebraic closure
    $\bar{K}$]
  Let $K \subset E \subset \bar{K}$. When we say about $K$ embedding
  of $E$ into the algebraic closure $\bar{K}$ we assume
  $\sigma \in Hom_K\left(E, \bar{K}\right)$ i.e. $\sigma$ is a
  \mynameref{def:homomorphism} of \mynameref{def:kalgebra}s i.e. the map
  that preserves the structure and especially $\sigma(K) = K$ i.e.
  $\forall k \in K: \sigma(k) = k$.

  If $E$ is a normal extension then all such homomorphisms have the
  same image accordingly theorem \ref{thm:lec5_3}. The image is $E$
  and therefore the homomorphisms can be considered as automorphisms
  i.e. $Gal\left(E/K\right) = Hom_K\left(E, \bar{K}\right)$.
  \label{rem:lec9_embedding}
\end{remark}

\begin{definition}[Norm]
  Let $K \hookrightarrow E$ - finite separable field extension. Let $\alpha \in
  E$.   Define the norm of alpha with respect to this extension as
  \[
  \mathrm {N}_{E/K}\left(\alpha\right) =
  \prod_{\sigma_i: E \hookrightarrow \bar{K}} \sigma_i\left(\alpha\right)
  \]
  i.e. we took a product by all $K$ embeddings of $E$ into the
  algebraic closure of $K$ (see remark \ref{rem:lec9_embedding}).
  And we also assume that $E$ is
  finite and as result $i = 1, \dots, r$.
  \label{def:norm}
\end{definition}

\begin{definition}[Trace]
  Let $K \hookrightarrow E$ - finite separable field extension. Let $\alpha \in
  E$.   Define the norm of alpha with respect to this extension as
  \[
  \mathrm {Tr}_{E/K}\left(\alpha\right) =
  \sum_{\sigma_i: E \hookrightarrow \bar{K}} \sigma_i\left(\alpha\right)
  \]
  i.e. we took a sum by all $K$ embeddings of $E$ into the
  algebraic closure of $K$ (see remark \ref{rem:lec9_embedding}). And
  we also assume that $E$ is finite and as result $i = 1, \dots, r$. 
  \label{def:trace}
\end{definition}

In the definitions \ref{def:norm} and \ref{def:trace} we assume that
the extension $E$ is \mynameref{def:separableextension}. If the
extension is not separable then you have to take it to the power
equal to the pure inseparable degree of $E/K$,
\footnote{
  \cite{bib:lang} p. 284 gives the following definitions for norm and
  trace in not separable case:
  \begin{eqnarray}
  \mathrm {N}_{E/K}\left(\alpha\right) =
  \left(
  \prod_{\sigma_i: E \hookrightarrow \bar{K}}
  \sigma_i\left(\alpha\right)
  \right)^{\left[E:K\right]_i},
  \nonumber \\
  \mathrm {Tr}_{E/K}\left(\alpha\right) =
  \left[E:K\right]_i
  \sum_{\sigma_i: E \hookrightarrow \bar{K}} \sigma_i\left(\alpha\right),
  \nonumber
  \end{eqnarray}
  where $\left[E:K\right]_i$ is \mynameref{def:inseparabledegree}.
}
but for simplicity,
we shall suppose that everything is separate.

\begin{remark}
  We have $\mathrm {N}_{E/K}\left(\alpha\right) \in K$ and
  $\mathrm {Tr}_{E/K}\left(\alpha\right) \in K$
  \label{rem:lec9_add}
  \begin{proof}
    Let $g \in Gal\left(E/K\right)$. We have to prove that
    $g\left(\mathrm {N}_{E/K}\left(\alpha\right)\right) =
    \mathrm {N}_{E/K}\left(\alpha\right)$. In the case
    $\mathrm {N}_{E/K}\left(\alpha\right) \in K$ because
    $E^{Gal\left(E/K\right)} = K$.

    The $g$ just permutes the homomorphisms $Hom_K\left(E,
    \bar{K}\right)$ i.e.
    $\left|g Hom_K\left(E, \bar{K}\right)\right| =
    \left|Hom_K\left(E, \bar{K}\right)\right|$. If this is not the
    truth then $\exists \sigma_i, \sigma_j, \sigma_k \in Hom_K\left(E,
    \bar{K}\right)$ such that
    $g\sigma_i = \sigma_k, g \sigma_j = \sigma_k$ and
    $\sigma_i \ne \sigma_j$.
    Therefore
    \[
    \sigma_j = g^{-1} \sigma_k = g^{-1} g \sigma_i = \sigma_i
    \]
    that is contradiction.
    By  the field definition
    (\ref{def:field}) the product in
    \mynameref{def:norm} does not depend on the order and the
    result become the same after the permutation.

    The same result is for \mynameref{def:trace}.
  \end{proof}
\end{remark}

\begin{property}
  \begin{enumerate}
  \item $\mathrm{N}_{E/K}: E^\times \to K^\times$
    \footnote{
      $E^\times = E \setminus \{0\}$ and
      $K^\times = K \setminus \{0\}$
    }
    is multiplicative
    i.e. homomorphism of groups.
    $\mathrm{Tr}_{E/K}: E \to K$ is additive, $K$-linear
    i.e. homomorphism of $K$-vector spaces.
    \footnote{
      The property statement is the truth i.e.
      there really should be $K$ (not $\bar{K})$(see remark
      \ref{rem:lec9_add}).
    }    
  \item If $E=K\left(\alpha\right)$, $n = \left[E:K\right]$ and
    $P_{min}\left(\alpha, K\right) = X^n + a_1 X^{n-1} + \dots +
    a_{n-1} X + a_{n}$ then $\mathrm{N}_{E/K}\left(\alpha\right) = (-1)^n
    a_n$ and $\mathrm{Tr}_{E/K}\left(\alpha\right) = -a_1$.
  \item If we have the tower of extensions $K \subset F \subset E$
    then
    \[
    \mathrm{N}_{E/K} =
    \mathrm{N}_{F/K} \circ
    \mathrm{N}_{E/F}
    \]
    and the same for trace
    \[
    \mathrm{Tr}_{E/K} =
    \mathrm{Tr}_{F/K} \circ
    \mathrm{Tr}_{E/F}
    \]
    \footnote{      
      Or in other words
      \[
      \mathrm{N}_{E/K}\left(\alpha\right) =
      \mathrm{N}_{F/K} \left(
      \mathrm{N}_{E/F}\left(\alpha\right)
      \right)
      \]
      and
      \[
      \mathrm{Tr}_{E/K}\left(\alpha\right) =
      \mathrm{Tr}_{F/K} \left(
      \mathrm{Tr}_{E/F}\left(\alpha\right)
      \right).
      \]
      Lets also note \cite{bib:KeithConradTraceNorm2} that the
      following expression does not make sense:
      \[
      \mathrm{Tr}_{E/F} \left(
      \mathrm{Tr}_{F/K}\left(\alpha\right)\right)
      \]
      because $\mathrm{Tr}_{E/F}: E \to F$ and
      $\mathrm{Tr}_{F/K}: F \to K$ (see remark
      \ref{rem:lec9_add}).
      The same is valid for \mynameref{def:norm}.  
    }
    \item Consider $T: E \times E \xrightarrow[(x, y) \to
      \mathrm{Tr}_{E/K}\left(xy\right)  ]{} K$. This is a
      non-degenerate $K$-bilinear form (see
      definition \ref{def:nondegeneratebf}) 
    \item $\alpha$ integral over $\mathbb{Z}$. Then
      $\mathrm{N}_{E/\mathbb{Q}}\left(\alpha\right),
      \mathrm{Tr}_{E/\mathbb{Q}}\left(\alpha\right)$ are integers.
      \footnote{
        We have $K = \mathbb{Q}$ in the property.
      }
  \end{enumerate}
  \label{property:lec9_norm_trace}
  \begin{proof}
    The first property is obvious from the definition.
    \footnote{
      Let $\alpha, \beta \in E$ then
      \begin{eqnarray}
        \mathrm{N}_{E/K}\left(\alpha \beta\right) =
          \prod_{\sigma_i: E \hookrightarrow \bar{K}}
          \sigma_i\left(\alpha \beta\right) =
          \nonumber \\
          =
          \prod_{\sigma_i: E \hookrightarrow \bar{K}}
          \sigma_i\left(\alpha\right)
          \sigma_i\left(\beta\right) 
          =
          \mathrm{N}_{E/K}\left(\alpha\right)
          \mathrm{N}_{E/K}\left(\beta\right).
        \nonumber
      \end{eqnarray}
      With remark \ref{rem:lec9_add} we can get that
      \[
      \mathrm{N}_{E/K}: E^\times \to K^\times
      \]

      If we take $a,b \in K, \sigma \in Hom_K\left(E, \bar{K}\right)$
      then one can get that
      $\sigma\left(a \alpha\right) = a
      \sigma\left(\alpha\right)$. This is because $a$ is fixed under
      every embedding of $E$ over $K$ (\cite{bib:lang} page 286), see
      also remark \ref{rem:lec9_embedding}.
      
      Therefore one can get
      \[
      \mathrm{Tr}_{E/K}\left(a \alpha + b \beta\right)=
      a \mathrm{Tr}_{E/K}\left(\alpha\right) +
      b \mathrm{Tr}_{E/K}\left(\beta\right)
      \]
      I.e. there is a $K$-linear map of $E$ to $K$
      (see remark \ref{rem:lec9_add}).
    }

    The second one uses the following fact:
    $\sigma_i\left(\alpha\right)$ are roots of $P_{min}\left(\alpha,
    K\right)$. The \mynameref{def:norm} is a product and it's assigned
    to its constant term ($a_n$) and the sum is the first coefficient 
    term ($a_1$) (see also example \ref{ex:lec8_generalequation}).

    The third property is somewhat less trivial, so this follows from,
    the fact that if $\tau_1, \dots, \tau_k$ are $K$ embeddings of $F$
    into $\bar{K}$ and, $\mu_1, \dots, \mu_s$ are $F$ embeddings of
    $E$ into $\bar{K}$ then the embedings of $E$ into $\bar{K}$ are
    just the compositions $\{\tau_j \mu_i\}$.
    \footnote{
      We have $K \subset F \subset \bar{K}$, and (see remark
      \ref{rem:lec9_embedding}):
      \[
      \begin{cases}
        \tau_j: F \to \bar{K} \text{ such that }
        \forall k \in K: \tau_j(k) = k, \\
        \mu_i: E \to \bar{K}  \text{ such that }
        \forall f \in F: \mu_i(f) = f.
      \end{cases}
      \]
      Lets extend homomorphism $\tau_j: F
      \to \bar{K}$ to the automorphism
      $\bar{K} \to \bar{K}$ and denote the result with the same
      $\tau_j$ (\cite{bib:lang} p. 285). 
      As result we have $\tau_j \mu_i: E \to \bar{K}$. We also have
      $\forall k \in K \subset F$
      \(
      \mu_i(k) = k
      \) (as soon as $k \in F$) and therefore
      $\tau_j(\mu_i(k)) = \tau_j(k) = k$. Thus
      $\tau_j \mu_i \in Hom_K\left(E, \bar{K}\right)$.
    }

    For the 4th property. Indeed if $x \in \ker T$,
    \footnote{
      $T: E \times E \to K$
    }
    that means 
    $\mathrm{Tr}_{E/K}\left(xy\right) = 0, \forall y \in E$ (see
    definition \ref{def:nondegeneratebf}), but this
    can't be a case when $xy \in K \setminus \{0\}$ by definition
    \ref{def:trace}
    \footnote{
      and by taking into consideration the following fact
      (see remark \ref{rem:lec9_embedding}):
      if $xy \in K$
      then $\sigma_i(xy) = xy$ 
    }
    $\mathrm{Tr}_{E/K}\left(xy\right) =
    \left[E:K\right] xy$.
    \footnote{
      we proved that $T$ has a trivial kernel i.e. only $x = 0$ is in
      the $\ker T$. There is one of definitions of the non-degenerate
      $K$-bilinear form (see definition \ref{def:nondegeneratebf}) 
    }

    For the 5th property we know that
    \begin{eqnarray}
    \mathrm{Tr}_{E/\mathbb{Q}}\left(\alpha\right) =
    \mathrm{Tr}_{\mathbb{Q}(\alpha)/\mathbb{Q}}\left(
    \mathrm{Tr}_{E/\mathbb{Q}(\alpha)}\left(\alpha\right)
    \right) =
    \nonumber \\
    =
    \mathrm{Tr}_{\mathbb{Q}(\alpha)/\mathbb{Q}}\left(
    \left[E:\mathbb{Q}(\alpha)\right] \alpha
    \right) =
    \left[E:\mathbb{Q}(\alpha)\right]
    \mathrm{Tr}_{\mathbb{Q}(\alpha)/\mathbb{Q}}\left(
     \alpha
    \right)
    \nonumber
    \end{eqnarray}
    but
    \(
    \mathrm{Tr}_{\mathbb{Q}(\alpha)/\mathbb{Q}}\left(
    \alpha
    \right) \in \mathbb{Z}
    \)
    because
    \(
    \mathrm{Tr}_{\mathbb{Q}(\alpha)/\mathbb{Q}}\left(
    \alpha
    \right)
    \) is a coefficient of $P_{min}\left(\alpha, \mathbb{Q}\right) \in
    \mathbb{Z}\left[X\right]$.
    \footnote{
      Property \ref{property:lec9_2} says that $P_{min}\left(\alpha,
      \mathbb{Q}\right) \in \mathbb{Z}\left[X\right]$. But 2d item of
      the property \ref{property:lec9_norm_trace} says that the
      \mynameref{def:trace} is a coefficient of the polynomial i.e.
      \(
      \mathrm{Tr}_{\mathbb{Q}(\alpha)/\mathbb{Q}}\left(
      \alpha
      \right) \in \mathbb{Z}
      \)
    }
  \end{proof}
\end{property}

Why such names are used? Consider the following map
(multiplication by $a$)
\[
f_a: E \xrightarrow[x \to a x]{} E
\]
then the $\mathrm{Tr}_{E/K}\left(a\right)$ is exactly the trace of the
linear map (i.e. sum of diagonal elements of the linear map matrix in
a basis) and the $\mathrm{N}_{E/K}\left(a\right)$ is the
determinant
\footnote{
  Consider an easy case when $E = K\left(a\right)$. In the case
  $a$ is a root of $P_{min}\left(a, K\right) = a_n + a_{n-1} X + \dots +
  a_1 X^{n-1} + X^{n}$.
  We have the following basis $1, a, \dots, a^{n-1}$. The basis is
  transformed by multiplication via the following rules
  \[
  \begin{cases}
    1 \to a, \\
    a \to a^2, \\
    \vdots \\
    a^{n-1} \to a^n = -a_n - a_{n-1} a - \dots -
    a_1 a^{n-1}.
  \end{cases}
  \]
  Therefore the \mynameref{def:endomorphism} matrix can be written as
  follows: 
  \[
  M = 
  \begin{bmatrix}
    0 & 1 & 0 & \cdots & 0 & 0 \\
    0 & 0 & 1 & \cdots & 0 & 0 \\
    \vdots & \vdots & \vdots & \ddots & \vdots & \vdots \\
    0 & 0 & 0 & \cdots & 0 & 1 \\
    -a_n & -a_{n-1} & -a_{n-2} & \cdots & -a_2 & -a_1 
  \end{bmatrix}
  \]
  It can be easy seen (with 2d item of
  the property \ref{property:lec9_norm_trace}) that
  \[
  \mathrm{Tr}\left(M\right) = -a_1 = 
  \mathrm{Tr}_{E/K}\left(a\right)
  \]
  and
  \[
  \det\left(M\right) = 
  (-1)^n a_n = 
  \mathrm{N}_{E/K}\left(a\right).
  \]  
}
. Now this $f_a$ is a a $K$-linear map. It's an
\mynameref{def:endomorphism} of a vector space $E/K$, and
the $\mathrm{Tr}_{E/K}\left(a\right)$ is
the trace of this endomorphism, and the
$\mathrm{N}_{E/K}\left(a\right)$ is the determinant of this endomorphism.  

\subsection{Theorem about rings of integers}
\begin{theorem}
  $O_k$ is a finitely generated (see definition \ref{def:fgmodule})
  $\mathbb{Z}$-module that is a \mynameref{def:freemodule} of
  rank (see definition \ref{def:rankfreemodule}) $n$, where $n=
  \left[K:\mathbb{Q}\right]$. 
  \label{thm:lec9_1}
  \begin{proof}
    If $e_1, \dots, e_n$ is a $\mathbb{Q}$-basis of $K$ then
    $\forall i \exists d_i \in \mathbb{Z} \setminus
    \{0\}$ such that $d_i e_i \in O_K$ (see property
    \ref{property:lec9_2}). Therefore $O_K$ contains a free
    $\mathbb{Z}$-submodule of rank $n$
    \footnote{
      This is because $d_1 e_1, \dots, d_n e_n$ are linearly
      independent and form a basis of a free $\mathbb{Z}$-module. The
      number $n$ is the cardinality of the basis.

      It can be proved by contradiction i.e. let there exists a set
      $\{c_i\}$ such that $\exists j: c_j \ne 0$ and
      \[
      \sum_{i=1}^n c_i d_i e_i = 0,
      \]
      as soon as $d_j \ne 0$ then $k_j = c_j d_j \ne 0$ and
      \[
      \sum_{i=1}^n k_i e_i = 0,
      \]
      i.e. $\{e_i\}$ are not linearly independent. This is in
      contradiction with the initial conditions.
    }.

    What is the $\mathbb{Z}$-module this is a finitely generated
    \mynameref{def:fgagroup} and we know a lot of things about such
    groups. The \mynameref{def:fgagroup} is the same as
    finitely generated $\mathbb{Z}$-module. Any such group is isomorphic
    to (see theorem \ref{thm:fgagroup})
    \[
    \mathbb{Z}^n \oplus A,
    \]
    where $A$ is a finite group (torsion part). A subgroup
    $B \subset \mathbb{Z}^n$ is itself a free module ($B \cong \mathbb{Z}^m$)
    of rank $m \le n$.

    We have to show that $O_K \subset A$ where $A$ is a free
    $\mathbb{Z}$-submodule of rank $n = \left[K:\mathbb{Q}\right]$.
    Let $e_1, \dots, e_n$ is a $\mathbb{Q}$-basis of $K$ (as above)
    contained in $O_K$. Consider the following map
    ($T: K \times K \to \mathbb{Q}$):
    \[
    (x,y) \to \mathrm{Tr}_{K/\mathbb{Q}}\left(xy\right) 
    \]
    this is
    \mynameref{def:nondegeneratebf} (see 4th property
    \ref{property:lec9_norm_trace}) therefore $\exists v_1, \dots,
    v_n$ - \mynameref{def:dualbasis} ($\mathbb{Q}$-basis of $K$)
    and we have the property that
    $\mathrm{Tr}_{K/\mathbb{Q}}\left(e_i v_j\right) = \delta_{ij}$.
    \footnote{
      Lets consider the following map $T: K \times K \to
      \mathbb{Q}$. We can use it to construct a linear map by the
      following rule: $\forall x \in K$ we have $f_x(y) =
      T(x,y): K \to \mathbb{Q}$.
      We have $f_{x+y} = f_x + f_y$ and $f_{ax} = a f_x$
      i.e. the set $\{f_x\} = K^\ast$ is the linear space.
      The map $K \xrightarrow[x \to f_x]{} K^\ast$ is
      \mynameref{def:surjection} by the $f_x$ construction. The map is
      also \mynameref{def:injection} as soon as the map $T: K \times K \to \mathbb{Q}$ is
      \mynameref{def:nondegeneratebf} and as result the $f_x(y) =
      T(x,y)$ has a trivial kernel (if $x \ne 0$): $\ker f =
      \{0\}$. Therefore we can conclude that $K^\ast$ is
      \mynameref{def:dualspace}. I.e. exists a set of elements 
      $\{f_{v_j}\} \subset K^\ast$  which
      form the \mynameref{def:dualbasis} to $\{e_i\}$ i.e. $f_{v_j}(e_i) =
      \delta_{ij}$. Each element $f_{v_j}$ of dual space corresponds
      to $v_j \in K$. For such $v_j$ we have:
      \[
      \delta_{ij} = f_{v_j}(e_i) = T(v_j, e_i) =
      \mathrm{Tr}_{K/\mathbb{Q}}\left(v_j e_i\right).
      \]
    }

    We claim that $\mathbb{Z}$ submodule generated by $v_1, \dots,
    v_n$ contains $O_K$. Indeed let $\alpha \in O_K$ and write
    \(
    \alpha = \sum \alpha_i v_i, \alpha_i \in \mathbb{Q}
    \). We can do it because $\{v_i\}$ is a $\mathbb{Q}$ basis of
    $K$. But one can see that $\alpha_i \in \mathbb{Z}$ because
    $\alpha_i = \mathrm{Tr}_{K/\mathbb{Q}}\left(\alpha e_i\right)$ (by
    definition of $v_j$).
    \footnote{
      \begin{eqnarray}
        \mathrm{Tr}_{K/\mathbb{Q}}\left(\alpha e_i\right) =
        \mathrm{Tr}_{K/\mathbb{Q}}\left(\sum_{j=1}^n\alpha_j v_j
        e_i\right) =
        \nonumber \\
        = \sum_{j=1}^n\alpha_j
        \mathrm{Tr}_{K/\mathbb{Q}}\left( v_j
        e_i\right) = 
        \sum_{j=1}^n\alpha_j \delta_{ij} = \alpha_i
        \nonumber
      \end{eqnarray}
    }
    Since $\alpha$ and $e_i$ are elements of
    $O_K$ then $\alpha e_i \in O_K$ too. Therefore
    $\mathrm{Tr}_{K/\mathbb{Q}}\left(\alpha e_i\right) \in
    \mathbb{Z}$. So $\alpha_i \in \mathbb{Z}$ and this one is what we
    want to proof. We have expressed any element of $O_K$ as a
    combination of $v_i$ with integral coefficients. So $O_K$ is contained
    in a $\mathbb{Z}$ submodule, generated by $\{v_i\}$.  
  \end{proof}
\end{theorem}

\section{Reduction modulo a prime}

Let $P \in \mathbb{Z}\left[X\right]$ is an irreducible polynomial with integer
coefficients. $K$ is a \mynameref{def:splittingfield} of $P$ over
$\mathbb{Q}$ and $n = \left[K:\mathbb{Q}\right]$. Let
$G = Gal\left(P\right) \eqdef Gal\left(K/\mathbb{Q}\right)$. We denote
roots of $P$ as $\alpha_1, \dots, \alpha_n$ and they are elements of
$O_K$. $G$ acts on the set of roots, and on $O_K$. We will denote
$O_K$ as $A$. Let $p$ is a prime number and we will consider $A/pA$.
As we have seen
\footnote{
  see proposition \ref{prop:lec4_prop2} where $M = A, A=\mathbb{Z}$
  and $I = p\mathbb{Z}$.
}
\[
A/pA \cong A \otimes_{\mathbb{Z}} \mathbb{Z}/p\mathbb{Z} =
A \otimes_{\mathbb{Z}} \mathbb{F}_p
\]
there $A \otimes_{\mathbb{Z}} \mathbb{F}_p$ is a 
$n$-dimension vector space over $\mathbb{F}_p$
\footnote{
  As soon as $A=O_K$ is a free
  $\mathbb{Z}$-module of rank $n$ (see theorem \ref{thm:lec9_1}) then
  proposition 
  \ref{prop:lec4_Addon} gives us that $A \otimes_{\mathbb{Z}}
  \mathbb{F}_p$ is a free $\mathbb{F}_p$-module with rank equal
  to $n$.
}.
Maximal ideals
of $A/pA$ are in one-to-one correspondence with maximal ideals of $A$
containing $p$. As we know (see theorem
\ref{thm:structurefinitekalgebra}) there are only finitely many
maximal ideal in a finite algebra over a field. Therefore $A$ also has
finitly many 
maximal ideals $J_1, \dots, J_r$ containing $p$. Our group $G$ acting
on $A$ must permute these maximal ideals in some way.
\footnote{
  Let $J$ is an \mynameref{def:ideal} of $A$, then $A J = J$
  (i.e. $\forall a \in A, j \in J: aj \in J$). Let also $g \in G$ then
  $g\left(A\right) = A$ (see \cite{bib:lang} p. 341) and
  \[
  A g\left(J\right) = g\left(A\right) g\left(J\right) =
  g\left(AJ\right) = g\left(J\right)
  \]
  i.e. $g\left(J\right)$ is also an ideal of $A$.
  \label{note:lec9_permute_ideal}
}

Lets consider a subgroup $D_i \subset G$ which stabilizes $J_i$ (see
definition \ref{def:stabilizersubgroup}) i.e.
\[
D_i = \{g \in G \mid g J_i = J_i\}.
\]
Let also $k_i = A/J_i$ - this is a field
\footnote{
  see theorem \ref{thm:maxideal}
}
and there is a finite
extension of $\mathbb{F}_p$.
\footnote{
  We have that $A$ is a $\mathbb{F}_p$-algebra. $J_i$ is a
  \mynameref{def:maxideal} of $A$ then by remark
  \ref{rem:lec5_quotientkalgebra} we have that $k_i = A/J_i$ is also
  $\mathbb{F}_p$-algebra i.e. $k_i$ is a field extension of
  $\mathbb{F}_p$ (see definition \ref{def:fextension2}). The extension
  is finite as soon as $A$ is finite.
}
Then there exists a natural homomorphism
\(
D_i \to Gal\left(k_i/\mathbb{F}_p\right)
\).
Since $D_i$ stabilizes $J_i$ and it acts on the residual classes of
modulo $J_i$ so there is a homomorphism of $D_i$ into the
\mynameref{def:galoisgroup}.
\footnote{
  We can notice that both $k_i$ and $D_i$
  are related to $J_i$ because any $g^{(k)} \in
  Gal\left(k_i/\mathbb{F}_p\right)$ is an automorphism of $k_i$
  i.e. it should preserve $O_{k_i} = J_i$ (as soon as it is an
  injection). Or in other words any element of
  $Gal\left(k_i/\mathbb{F}_p\right)$ acts in the same way as elements
  of $D_i$ (preserves $J_i$). See also theorem \ref{thm:lec9_2} about
  the map structure (it's surjection and in several situations is also
  bijection). 
}


\begin{theorem}
  \begin{enumerate}
  \item $G$ acts transitively (see definition \ref{def:transitive}) on
    $\{J_1, \dots, J_r\}$ and the map
    $D_i \to Gal\left(k_i/\mathbb{F}_p\right)$ is a
    \mynameref{def:surjection} i. e.
    $D_i \twoheadrightarrow Gal\left(k_i/\mathbb{F}_p\right)$
    \item If the reduction $\bar{P} = P \mod p$ has no multiple roots
      then the map $D_i \to Gal\left(k_i/\mathbb{F}_p\right)$ is
      bijection and $k_i$ is a splitting field of the reduction
      $\bar{P}$. 
  \end{enumerate}
  \label{thm:lec9_2}
  \begin{proof}
    For the first part. Suppose that for some $i$ and $\forall g \in G,
    g\left(J_1\right) \ne J_i$ i.e. suppose that there is not a
    \mynameref{def:transitive}. By \mynameref{thm:chineseremainder}
    $\exists x \in A$ such that
    $x \equiv 0 (\mod J_i),  x \equiv 1 (\mod g\left(J_1\right))
    \forall g \in G$.
    \footnote{
      We have that all $J_i$ are relatively prime (see note
      \ref{note:lec9_primeideal} below) and as result 
      the map $\pi\left(a\right)$ from (\ref{eq:lec4_crt_map}) is
      \mynameref{def:surjection} 
      i.e. for any residuals and therefore for the chosen ones
      ($\equiv 0 \mod J_i$ and $\equiv 1 \mod J_1$) exists $x$ that
      produces the residuals.
      \label{note:lec9_crt}
    }
    Consider a product of all such things:
    \[
    a = \prod_{g} g x
    \]
    it's an integer $a \in \mathbb{Z}$.
    \footnote{
      This is because $\prod_{g} g x =
      \mathrm{N}_{A/\mathbb{Q}}\left(x\right)$, but by 5th item of
      property \ref{property:lec9_norm_trace} one can get that such
      \mynameref{def:norm}
      (in the definition of norm we have homomorphism but not
      automorphism but accordingly remark \ref{rem:lec9_embedding} it
      is not important as soon as $A$ is a normal extension of
      $\mathbb{Q}$)  
      is an integer.       
    }
    But since $x \in J_i$
    (remember that $\equiv 0 \mod J_i$)
    then $a$
    is also in $J_i$:
    \footnote{
      $a = \prod_{g} g x = x \prod_{g \ne id} g x$ and by ideal
      definition \ref{def:ideal} we can conclude that $a \in J_i$.
    }
    $a \in \mathbb{Z} \cap J_i = (p)$
    \footnote{
      We have $J_i$ is a \mynameref{def:maxideal} and $A$ is a
      \mynameref{def:pid}. Therefore by lemma \ref{lem:primeideal_is_maximal},
      $J_i$ is a \mynameref{def:primeideal}
      that contains $p  
      \in \mathbb{Z}$, but the prime ideal in $\mathbb{Z}$ that
      contains $p$ is $(p)$ (see lemma \ref{lem:primeideal_in_Z}).
      \label{note:lec9_primeideal}
    }
    - the ideal generated by the prime
    number $p$. So one has $a \in J_1$ since all $J_i$, and
    especially $J_1$, contains $p$. But this is impossible as soon as
    $J_1$ is a \mynameref{def:primeideal} (see note
    \ref{note:lec9_primeideal}). That is because if we have 
    $\prod_k x_k \in J_1$ ($x_k = g_k(x)$)
    then $\exists i$ such that $x_i \in J_1$ but
    there is not a case in our construction.
    \footnote{
      We have $x
      \in J_i$ and $x_i \in J_1$ but $x_i$ is obtained from $x$ by
      applying an element from $G$ i.e. $\exists g_i \in G$ such that
      $x_i = g_i(x)$. But if we consider a reverse element: $g =
      g^{-1}_i$ then $g(x_i) = x \in 
      J_i$ that is in contradiction with $g(J_1) \ne J_i$.
    }

    We still need to proof that $D_i \twoheadrightarrow
    Gal\left(k_i/\mathbb{F}_p\right)$.
    \footnote{
      i.e. that there is a
      \mynameref{def:surjection} or in other words that $\forall \bar{g}_i
      \in Gal\left(k_i/\mathbb{F}_p\right), \exists d_i \in D_i$ such that
      $d_i \to \bar{g}_i$.
    }
    We may assume that $i=1$. By the
    \mynameref{thm:primitiveelement} $\exists z \in
    \mathbb{F}_p$ such that $k_1 = \mathbb{F}_p\left(z\right)$
    i.e. $z$ generates $k_i/\mathbb{F}_p$.
    By \mynameref{thm:chineseremainder} $\exists y \in A$ such that
    $y \in J_i, i \ne 1,  y \equiv z (\mod J_1)$.
    \footnote{
      I.e. using the same argument as in note \ref{note:lec9_crt} one
      can conclude that $\exists y \in A$ such that its residual are
      the following $\forall i \ne 1$: $\equiv 0 \mod J_i$ and $\equiv z
      \mod J_1$.
    }
    Consider polynomial
    $Q = \prod_{g \in G} \left(X - g\left(y\right)\right)$. There is a
    polynomial with integral coefficients i.e.
    $Q \in \mathbb{Z}\left[X\right]$. This is because we know that
    coefficients are $G$ invariant
    i.e. $Q \in
    \mathbb{Q}\left[X\right]$
    \footnote{
      $Q = P_{min}\left(y, \mathbb{Q}\right)$.
    }
    moreover the roots
    are \mynameref{def:integralelement}s over $\mathbb{Z}$
    \footnote{
      as soon as $y \in A = O_K$.
    }
    and therefore the polynomial arguments are in $\mathbb{Z}$.
    \footnote{
      Ekaterina also said that ??? 
      as soon as
      $\mathbb{Z}$ is \mynameref{def:integrallyclosed}.
    }

    Lets study $\bar{Q} = Q \mod J_1$. If $g \notin D_1$ then $\exists
    i$ such that $g\left(J_i\right) = J_1$
    \footnote{
      If $g \notin D_1$ then $g^{-1} \notin D_1$ i.e.
      $g^{-1}\left(J_1\right) \ne J_1$ i.e. as soon as $g \in G$
      permutes the ideals (see note \ref{note:lec9_permute_ideal})
      then $\exists i$ such that 
      $g^{-1}\left(J_1\right) = J_i$ i.e. $g\left(J_i\right) = J_1$.
    }
    and particularly $g(y) \in
    J_1$. Therefore for such $g$ we have
    \[
    \overline{X -g\left(y\right)} = X -g\left(y\right) \mod J_1 = X. 
    \]
    So we have for $\bar{Q} \in \mathbb{F}_p\left[X\right]$
    \[
    \bar{Q} = \prod_{g \in G \setminus D_1} X \prod_{g \in D_1}
    \left(X - \overline{g\left(y\right)}\right),
    \]
    but
    \(
    \prod_{g \in D_1}
    \left(X - \overline{g\left(y\right)}\right)
    \) has $z$ as a root
    \footnote{
      Let $g = id \in D_1$ then
      $\overline{g\left(y\right)} = y \mod J_1 = z$        
    }
    and $D_1$ acts transitively on its roots.

    Now recall that $z$ generates $k_1$ (i.e. $k_1 =
    \mathbb{F}_p\left(z\right)$). Thus an element of 
    $Gal\left(k_1/\mathbb{F}_p\right)$ is determined by the image of
    $z$.
    \footnote{
      Because we have bijection $g \to g(z)$. 
    }
    And we have an element of $D_1$ which sends $z$ to any
    possible image of it. But this means that
    \(
    D_1 \twoheadrightarrow
    Gal\left(k_1/\mathbb{F}_p\right)
    \)

    For the second part of the theorem we assume that $\bar{P}$ has no
    multiple roots. So $\alpha_1, \dots, \alpha_n$ -roots of $P$ and
    $\bar{\alpha}_1, \dots, \bar{\alpha}_n$ -roots of $\bar{P}$ where
    $\bar{\alpha}_i = \alpha_i \mod J_1$.

    Lets $g \in D_1$ acts as $id$ on $k_1$. Then, of course,
    $g\left(\bar{\alpha}_i\right) = \bar{\alpha}_i$.
    But $g\left(\alpha_i\right) \in \{\alpha_1, \dots, \alpha_n\}$ and
    it can not be different from $\alpha_i$ since they will have
    diffeent reduction $\mod J_1$. So
    $\forall i,  g\left(\alpha_i\right) = \alpha_i$ and therefore $g =
    id$.
    \footnote{
      This proves injection of the map
      $D_1 \to
      Gal\left(k_1/\mathbb{F}_p\right)$
      (see lecture 6 note \ref{note:lec6_staff_comment}).
    }
    Thus conclusion that $D_1 \cong
    Gal\left(k_1/\mathbb{F}_p\right)$.

    By the same argument
    \footnote{
      i.e.
      $\forall g \in Gal\left(k_1/\mathbb{F}_p\left[\bar{\alpha}_1,
        \dots, \bar{\alpha}_n\right]\right)$ and
      $\forall x \in \mathbb{F}_p\left[\bar{\alpha}_1,
        \dots, \bar{\alpha}_n\right]$ we have $g(x) = x$ i.e.
      $g = id$.
    }
    \[
    Gal\left(k_1/\mathbb{F}_p\left[\bar{\alpha}_1,
      \dots, \bar{\alpha}_n\right]\right) = id
    \]
    therefore
    $k_1 = \mathbb{F}_p\left[\bar{\alpha}_1,
      \dots, \bar{\alpha}_n\right]$ i.e. $k_1$ is a splitting field.
  \end{proof}
\end{theorem}

\section{Finding elements in Galois groups}
How can we apply the above material to study Galois groups? One uses
this theorem to construct elements of a certain type in the 
Galois group to show that the Galois group is large.

So let $P \in \mathbb{Z}\left[X\right]$ be an
irreducible polynomial and suppose that there is a prime
$p \in \mathbb{Z}$ such that $\bar{P} = P \mod p$ is also irreducible.
Then $Gal\left(P\right)$ contains a subgroup that is isomorphic to 
$Gal\left(\bar{P}\right)$.
\footnote{
  See theorem \ref{thm:lec9_2} there
  $D_i \cong Gal\left(k_i/\mathbb{F}_p\right)$
}
But we know Galois group of finite fields
and we conclude that this Galois group contains an $n$ cycle.
This is because $Gal\left(\bar{P}\right)$ is cyclic generated by $n$
cycle (see corollary \ref{cor:lec3_2}).

Sometimes, there is no such prime, but of course, a variant of this
argument exists also in other cases. Suppose, for instance that $P$ is
irreducible of degree 5 and that 
$\bar{P} = R_2 R_3$ where $R_i$ is irreducible of degree $i$.
Then the same argument, gives that
$Gal\left(P\right)$ contains the permutation $(1,2)(3,4,5)$.
\footnote{
  The $Gal\left(P\right)$ contains a subgroup of 2-cycle (associated
  with the first 2 roots) and a
  subgroup of 3-cycle (associated with the last 3 roots).
}

And in this way one can construct elements of particular type in the
Galois group and use this to show that those groups are very large.  
