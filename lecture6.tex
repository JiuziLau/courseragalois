%% -*- coding:utf-8 -*-
\chapter{Galois correspondence and first examples}
We state and prove the main theorem of these lectures: the Galois
correspondence. Then we start doing examples (low degree,
discriminant, finite fields, roots of unity).

\section{Some further remarks on normal extension. Fixed field}

Some definitions from previous lecture. $L$ over $K$ is
\indexref{def:galoisextension} if and only if it is a
\indexref{def:separableextension} and \indexref{def:normalextension} or
in other words $L$ is a \indexref{def:splittingfield} of a family of
separable irreducible polynomials over $K$. We also seen (see theorem
\ref{thm:lec5_4}) that in the case of finite extension
$\left[L:K\right] < \infty$ the number of automorphisms
$\left|Aut\left(L/K\right)\right| = \left[L:K\right]$.

There are several remarks on \indexref{def:normalextension}s which show
that the extensions behave sometimes differently compare to other
types of extensions. Especially we have seen for that an extension 
$L$ over $M$ over $K$ was finite or algebraic or separable or purely
inseparable if, and only if, it was true for $L$ over $M$ and $M$ over
$K$. So, for a normal extensions, this is not the case anymore.

\begin{remark}
  Let we have a tower of extensions $K \subset L \subset M$. If $M$ is
  normal over $K$ then of course the $M$ is normal over $L$. It is
  clear because if $M$ is a splitting field of a family of polynomials
  over $K$ the one can just consider them as being polynomials over
  $L$ and say that $M$ is a splitting field of a family of polynomials
  over $L$.

  But $L$ does not have to be normal over $K$ (see example
  \ref{ex:lec6_1}). 
\end{remark}

\begin{example}
  Consider
  \[
  \mathbb{Q} \subset
  \mathbb{Q}\left(\sqrt[4]{2}\right)
  \subset
  \mathbb{Q}\left(\sqrt[4]{2}, i\right)
  \]

  We have $\mathbb{Q}\left(\sqrt[4]{2}, i\right)$ to be a splitting
  field for polynomial $X^4 -2$ but
  $\mathbb{Q}\left(\sqrt[4]{2}\right)$ is just a
  \indexref{def:stemfield} (not \indexref{def:splittingfield}) for this
  polynomial. And as result $\mathbb{Q}\left(\sqrt[4]{2}\right)$ is
  not a normal over $\mathbb{Q}$.  
  \label{ex:lec6_1}
\end{example}

\begin{remark}
  A quadratic extension
  \footnote {
    Extensions with degree 2. 
  }
  is normal. This is by formula for roots of a
  quadratic equation.
  \footnote {
    Because a if we have 2 roots $x_1, x_2 \in L \supset K$ then there
    exists the following relation $x_2 = k_1 + k_2 x_1$ where $k_1,
    k_2 \in K$. I.e. if we know one root then the other is easy
    computed and located in the same extension as the first one. 
  }

  If $P$ quadratic over $K$ has 1 root in $L$ then its another root is
  also in $L$.
\end{remark}

\begin{remark}
  One often has $K \subset L \subset M$ with $L$ normal over $K$, $M$
  normal over $L$ but $M$ not normal over $K$ (see example
  \ref{ex:lec6_2}). 
\end{remark}

\begin{example}
  Consider
  \[
  \mathbb{Q} \subset
  \mathbb{Q}\left(\sqrt{2}\right)
  \subset
  \mathbb{Q}\left(\sqrt[4]{2}\right)
  \]

  We have $\mathbb{Q}\left(\sqrt{2}\right)$ normal over $\mathbb{Q}$
  as well as $\mathbb{Q}\left(\sqrt[4]{2}\right)$ normal over
  $\mathbb{Q}\left(\sqrt{2}\right)$ because they both are quadratic
  extensions. But 
  $\mathbb{Q}\left(\sqrt[4]{2}\right)$ is not normal over $\mathbb{Q}$
  (as it was mentioned in example \ref{ex:lec6_1})
  \label{ex:lec6_2}
\end{example}

We also seen at last lecture the following definition
(see definition \ref{def:setinvariants}):
\begin{definition}[Fixed field]
  If $L$ is a field and $G \subset Aut\left(L\right)$ then
  \[
  L^G = \left\{
  x \in L \mid \forall g \in G: g x = x 
  \right\}
  \]  
  is a fixed field
  \label{def:fixedfield}
\end{definition}

If we have a sub-field $K \subset L$ then we can consider the
following group of automorphisms of $L$ over $K$:
$Aut\left(L/K\right)$ in the case if $L$ is normal because otherwise
the group will be too small to give information 
about $L$. But in the normal case it makes sense to consider  
the group of automorphisms of $L$ over $K$.

We have seen (see (\ref{eq:lec5_2})) that if $L$ is separable over $K$
then 
\[
L^{Aut\left(L/K\right)} = K
\]
This is because of the group of automorphisms was permuting the roots
over the minimal polynomial of $x$ over $K$ (see item
\ref{rem:item:lec5_onnormalext_3} in remark
\ref{rem:lec5_onnormalext}). 

We also have seen (see theorem \ref{thm:artin}) that if $G$ is finite
the $L$ is \indexref{def:galoisextension} over $L^G$ and
$\left[L:L^G\right] = \left|G\right|$.

And now we are going to summarize all these in a theorem which is in
fact the main subject of this lecture course and this theorem is
called the Galois correspondence. 

\section{The Galois correspondence}

Let $L$ over $K$ be a \indexref{def:galoisextension}. By definition the
group automorphisms $Aut\left(L/K\right)$ is called
\indexref{def:galoisgroup} and denoted as $Gal\left(L/K\right)$
\begin{theorem}[Galois correspondence]
  \label{thm:galoiscorrespondence}
  There are several statements:
  \begin{enumerate}
  \item If $L$ is finite over $K$ then there is a
    \indexref{def:bijection} between a sub-extension $F$
    ($K \subset F \subset L$) and a subgroup $H \subset
    Gal\left(L/K\right)$. The correspondence is the following
    \begin{eqnarray}
      F \rightarrow Gal\left(L/F\right)
      \nonumber \\
      L^H \leftarrow H
      \nonumber
    \end{eqnarray}
  \item The following statement are equivalent (if and only if)
    \begin{enumerate}
    \item $F$ is Galois over $K$ \label{thm:galoiscorrespondence:item2a}
    \item $\forall g \in Gal\left(L/K\right) g\left(F\right) = F$
      \label{thm:galoiscorrespondence:item2b}
    \item $Gal\left(L/F\right)$ is a \indexref{def:normalsubgroup} in
      $Gal\left(L/K\right)$ 
    \end{enumerate}
    In this case  $g$ goes to $g$ restricted to $F$: $g \to
    \left.g\right|_F$ this is a \indexref{def:surjection}
    $Gal\left(L/K\right) \twoheadrightarrow Gal\left(F/K\right)$ and
    the kernel is $Gal\left(L/F\right)$.
    \footnote{
      The fact is explained at the end of the proof of the theorem and
      is used in the claim \ref{claim:galoisquotient} validation. 
    }
  \end{enumerate}
  \begin{proof}
    \begin{enumerate}
    \item Most work have been done before. What have we got by now?
      $L^{Gal\left(L/F\right)} = F$ (see (\ref{eq:lec5_2})). We also
      have $H \subset Gal\left(L/L^H\right)$.
      \footnote{
        Ekaterina called the fact obvious but let try to explain it.
        From (\ref{eq:lec5_2}) we have
        $L^{Gal\left(L/L^H\right)} = L^H$ thus
        $\forall h \in H$ it should also satisfy $h \in
        Gal\left(L/L^H\right)$ therefore $H \subset
        Gal\left(L/L^H\right)$. 
      }
      \indexref{thm:artin} gives us
      $\left[L:L^H\right] = \left|H\right|$ but with theorem
      \ref{thm:lec5_4} we also have
      $\left[L:L^H\right] = \left|Gal\left(L/L^H\right)\right|$ so one
      must have $H = Gal\left(L/L^H\right)$ as soon as the subset $H$
      has the same cardinality as the set $Gal\left(L/L^H\right)$ itself.

      This means that the maps that we have in the theorem :
      $F \rightarrow Gal\left(L/F\right)$ and
      $L^H \leftarrow H$ are mutually inverse
      \footnote{
        We have the following maps:
        \[
        F \to Gal\left(L/F\right) \to L^{Gal\left(L/F\right)} = F
        \]
        I.e. $\forall F$ such that $K \subset F \subset L$ we can
        construct $H = Gal\left(L/F\right)$ then we can construct
        $L^H$ such that (by theorem \ref{thm:artin})
        $K \subset L^H \subset L$.
      }
      and if a map is invertible then there is a \indexref{def:bijection}.
    \item We should proof equivalence of the following statements:
      \begin{enumerate}
      \item $F$ is Galois over $K$ \label{item:galoiscorrespondence1}
      \item $\forall g \in Gal\left(L/K\right), g\left(F\right) = F$
        \label{item:galoiscorrespondence2}
        \footnote{
          $g \in Gal\left(L/K\right)$ operates on $L \supset F$.
        }
      \item $Gal\left(L/F\right) \triangleleft Gal\left(L/K\right)$
        \label{item:galoiscorrespondence3}
      \end{enumerate}      
    \end{enumerate}

    Lets show that \ref{item:galoiscorrespondence1} implies
    \ref{item:galoiscorrespondence2}. Fix $x \in F$ then the minimal
    polynomial $P_{min}\left(x, K\right)$ splits in $L$ but it has a
    root in $F$, thus it should have all roots in $F$ by normality.
    I. e. as soon as $F$ is a \indexref{def:normalextension},
    $P_{min}\left(x, K\right)$ splits in $F$ (see theorem
    \ref{thm:lec5_3}). This means, of course, 
    that any map from Galois group preserves $F$ since it premutes the
    roots: $\forall g \in Gal\left(L/K\right)$ $g$ permutes the roots
    of $P_{min}\left(x, K\right)$ and that is the true for any $x \in
    F$ therefore $g\left(F\right) \subset F$ since $F$ is generated
    (consists of) such roots.
    \footnote{
      We have the following fact: $x \in F$ then also $g(x) \in F$
      i.e. $g(F) \subset F$. For equality $g\left(F\right) = F$ we
      have to proof that $\forall g \in Gal\left(L/K\right): F \subset
      g\left(F\right)$. Really let $g \in Gal\left(L/K\right)$ then
      $g^{-1} \in Gal\left(L/K\right)$ and
      $g^{-1}(F) \subset F$ therefore $F = id(F) =
      g\left(g^{-1}(F)\right) \subset 
      g(F)$ i.e. $F \subset g(F)$.  
    }

    Lets show that \ref{item:galoiscorrespondence2} implies
    \ref{item:galoiscorrespondence1}. If $g\left(F\right) \subset F$
    then all roots of $P_{min}\left(x, K\right)$, $x \in F$, are in $F$
    since $g$ permutes those roots or, in other words, since Galois
    group acts transitively on
    roots of an irreducible polynomial
    \footnote{
      see also 2d remark \ref{rem:lec5_onnormalext} that says that if
      $F$ is normal then $Gal(L/F)$ acts
      transitively on the roots of any irreducible polynomial $P \in
      F\left[X\right]$. 
    }
    therefore $F$ is normal by
    definition.
    \footnote{
      More clear explanation is below. We have to prove that $F$ is
      normal and separable. Normality: $\forall P_{min}\left(x,
      K\right)$ splits in $F$ thus the theorem \ref{thm:lec5_3} gives
      us the required normality. About separability. We have $K
      \subset F \subset L$. $L$ is separable over $K$ as soon as $L$
      is Galois extension. Thus theorem \ref{thm:lec3_3} gives us $F$
      separability. 
    }

    Lets show that \ref{item:galoiscorrespondence1} and
    \ref{item:galoiscorrespondence2} are equivalent to
    \ref{item:galoiscorrespondence3} i.e. let $g \in G$,
    $g\left(F\right) \subset L$ then if
    $h \in Gal\left(L/F\right)$ is such that $\left.h\right|_F = id$
    then $\left.g h g^{-1}\right|_{g\left(F\right)} = id$.
    \footnote{
      The map $g^{-1}$ acts as follows $g^{-1}: g(F) \to F$ i.e. $h$
      in $\left.g h g^{-1}\right|_{g\left(F\right)}$ operates on $F$
      only therefore
      \[
      \left.g h g^{-1}\right|_{g\left(F\right)} =
      \left.g h_F g^{-1}\right|_{g\left(F\right)}  =
      \left.g g^{-1}\right|_{g\left(F\right)} = id
      \]
    }
    This means
    that $g h g^{-1} \in Gal\left(L/g\left(F\right)\right)$ (see
    (\ref{eq:lec5_2})) so the 
    statement $g\left(F\right) = F$ is the same to say
    \[
    g Gal\left(L/F\right) g^{-1} = Gal\left(L/F\right)
    \]
    So apply this to all $g \in Gal\left(L/K\right)$ one can get that
    $Gal\left(L/F\right)$ is a \indexref{def:normalsubgroup} of
    $Gal\left(L/K\right)$ by the definition of
    \indexref{def:normalsubgroup}. 

    Finally if all this statements
    (\ref{item:galoiscorrespondence1}
    $\Longleftrightarrow$
    \ref{item:galoiscorrespondence2}
    $\Longleftrightarrow$
    \ref{item:galoiscorrespondence3}
    ) are true then we can consider map (make sense by
    \ref{item:galoiscorrespondence2}) 
    \[
    \phi: Gal\left(L/K\right)
    \xrightarrow[g \to \left.g\right|_F]{}
    Gal\left(L/F\right).
    \]
    This is a \indexref{def:surjection} by theorem \ref{thm:lec2_3}
    \footnote{
      We have that $\forall g_F \in Gal\left(L/F\right)$ $g_F$ is a
      homomorphism that operates as follows $g_F: F \to \bar{K}$
      (note that by the theorem \ref{thm:lec5_3} all homomorphisms on
      a normal extension $F$ have the same image namely $F$)
      and
      therefore it can be extended (by theorem 
      \ref{thm:lec2_3}) to homomorphism $g: L \to \bar{K}$
      i.e. $\forall g_F \in  Gal\left(L/F\right), \exists g \in
      Gal\left(L/K\right)$ such that $\phi(g) = g_F$. Or in other
      words $\phi$ is a \indexref{def:surjection}.
    }
    and the kernel $\ker \phi = Gal\left(L/F\right)$ by
    definition because the kernel consists of things which are identity on $F$. 
  \end{proof}
\end{theorem}

\begin{remark}
  If $L$ over $K$ is not finite then Galois correspondence is not
  \indexref{def:bijection} i.e. the maps which are in the theorem still
  make sense, but they will not be mutually inverse bijections and we
  shall see an example (see section \ref{sec:lec6_finitefield} about it).
  \label{rem:lec6_gcnotbijection}
\end{remark}

\section{First examples (polynomials of degree 2 and 3)}

\begin{example}[Degree 2]
  Let $\left[L:K\right] = 2$ and $char K \ne 2$. The extension $L$ is
  generated by a root of 
  quadratic polynomial i.e. $x \in L \setminus K$ then
  $P_{min}\left(x, K\right)$ is quadratic and if we look at the
  formula for the root of the equation we will see that the extension
  is generated by a root of discriminant $\Delta$:
  $L = K\left(\sqrt{\Delta}\right)$.

  What can we say about the $Gal\left(L/K\right)$. It consists of 2
  elements and there is only one cyclic group of 2 elements
  \footnote{
    There is only one group of 2 elements \cite{wiki:finitegroup}
  }
  :
  $\mathbb{Z}/2\mathbb{Z}$. Therefore
  \[
  Gal\left(L/K\right) \cong \mathbb{Z}/2\mathbb{Z}.
  \]
  The elements of the group is identity and an element that exchanges
  the 2 roots i.e. permutes $\sqrt{\Delta}$ and $-\sqrt{\Delta}$.
\end{example}

\begin{example}[Degree 3]
    Let $\left[L:K\right] = 3$ and $char K \ne 3$ (separable
    extensions) then $L$ is generated by $x$ - root of 
    degree 3 polynomial $P$ and there are 2 cases
    \begin{enumerate}
    \item $P$ splits in $L$ therefore $L$ is a Galois extension and
      $Gal\left(L/K\right)$ is the Galois group but the
      Galois group of 3 elements must be cyclic
      \footnote{
        There is only one group of 3 elements 
        and it is cyclic \cite{wiki:finitegroup}.
      }
      i.e.
      $Gal\left(L/K\right) \cong \mathbb{Z}/3\mathbb{Z}$ - cyclic
      group of order 3.
    \item $P$ does not split in $L$ then there exists
      $M = K\left(x_1, x_2, x_3\right)$ - splitting field where
      $x_{1,2,3}$ are roots of $P$ and $L = K\left(x_1\right)$. $M$ is
      Galois extension and the Galois group is embeded into a group of
      permutation of 3 elements (because Galois group permutes the roots):
      $Gal\left(M/K\right) \hookrightarrow S_3$.

      As soon $L \subsetneq M$ then $\left[M:K\right] > 3$ so
      $Gal\left(M/K\right) = S_3$. In particular
      $\left[M:K\right] = \left|S_3\right| = 6$
    \end{enumerate}

    If you see a cubic polynomial how will you decide is its Galois
    group is cyclic or $S_3$? This is determined by a
    \indexref{def:discriminant} of polynomial which is a subject of
    next section (see example \ref{ex:lec6_discriminant3degree}
    and proposition \ref{prop:lec6_1}).

    \label{ex:lec6_degree3}
\end{example}

\section{Discriminant. Degree 3 (cont'd). Finite fields}

\subsection{Discriminant}

\begin{definition}[discriminant]
  Let $P \in K\left[X\right]$. The polynomial has the following roots
  in $\bar{K}$: $x_1, x_2, \dots, x_n$. The following product is
  called discriminant:
  \[
  \Delta = \prod_{i < j} \left(x_i - x_j\right)^2
  \]
  \label{def:discriminant}
\end{definition}

If we take group $G = Gal\left(P\right)$ then
$G \subset S_n$ and any permutation preserves $\Delta$ and as result
we have $\Delta \in K$ (see (\ref{eq:lec5_2})).

Lets take a root of discriminant (we have to choose some roots order
for this operation) then
\[
\sqrt{\Delta} = \prod_{i < j} \left(x_i - x_j\right)
\]
this quantity is preserved only by even
\footnote{
  You can see from definition \ref{def:paritypermutation} and example
  \ref{ex:paritypermutation} that for even permutations each insertion
  is equivalent to a root exchange. If the number of such exchange is
  even then we can return the changed root of a discriminant to its original
  form with event steps. Each step changes the sign and as result the
  even steps will not change the sign.
}
(and not by odd) permutations.

\begin{proposition}
  Let $G = Gal\left(P\right)$ - \indexref{def:galoisgroup} then $G
  \subset A_n$
  \footnote {
    $A_n$ is a group of even permutations
  }
  if and only if $\sqrt{\Delta} \in K$.
  \begin{proof}
    Since if the Galois group is even then, this will be preserved by
    an element of Galois group and so will be in $K$ and conversely, if
    it is an element of $K$, then it must be preserved by the Galois
    group, but we know it is preserved only by even permutations. 
  \end{proof}
  \label{prop:lec6_1}
\end{proposition}

If we return to our example \ref{ex:lec6_degree3} we can get the
following one
\begin{example}[Discriminant of polynomial degree 3]
  Lets compute the discriminant for the following polynomial:
  $X^3 + p X + q$.
  \footnote {
    $X^2$ element can be always hidden via a variable change. Thus the
    polynomial can be considered as a common case for cubic polynomials.
  }

  The discriminant easy to compute:
  \footnote{
    The below explanation was taken from
    \cite{mathstackexchange:discriminant3}. 
    The discriminant is a polynomial of degree 6. $p$ can be
    represented as a polynomial of degree 2. $q = - X^3 - p X$ is a
    polynomial of degree 3. The discriminant therefore can be
    represented as follows $\Delta = a p^3 + b q^2$ where $a,b$ are
    numbers. Let $p = -1, q = 0$ then the polynomial $X^3 - X$ has 3
    roots: $-1, 0, 1$. It is easy to compute discriminant
    $\Delta = (0 + 1)^2(0 - 1)^2(1+1)^2 = 4$ therefore $a =  -4$.
    For the case $p = 0, q = -1$ we will get the polynomial $X^3 - 1$
    with roots $1, \zeta, \zeta^2$. We also have
    \[
    \zeta^2 + \zeta + 1 = 0
    \]
    therefore
    \[
    \left(\zeta - 1\right)^2 = \zeta^2 - 2 \zeta + 1 = - 3 \zeta
    \]
    and
    \[
    \left(\zeta + 1\right)^2 = \zeta^2 + 2 \zeta + 1 =  \zeta
    \]    
    thus
    \begin{eqnarray}
    \Delta = \left(\zeta - 1\right)^2\left(\zeta^2 -
    1\right)^2\left(\zeta^2 - \zeta\right)^2 =
    \nonumber \\
    = \left(\zeta - 1\right)^2 \left(\zeta - 1\right)^2
    \left(\zeta + 1\right)^2 \zeta^2 \left(\zeta - 1\right)^2 =
    \nonumber \\
    = \left(- 3 \zeta\right)^3 \zeta^3 = -27.
    \nonumber
    \end{eqnarray}
    As result $b=-27$.
  }
  \(
  \Delta = -4 p^3 - 27 q^2
  \).
  So if $\Delta$ is a square in $K$ then $Gal\left(P\right) \cong A_3$
  (cyclic of order 3)
  \footnote{
    See example \ref{ex:s3group} about groups $S_3$ and
    \indexref{def:alternatinggroup} $A_3$.
  }
  . If not then $Gal\left(P\right) \cong S_3$ (non
  commutative group of 6 elements).

  What can we say about sub-extensions for the two cases? Let $M$ is a
  splitting field of $P$ over $K$ then for the first case there is no
  any sub-extension. For the second case there are several
  sub-extensions (they are determined by sub-groups of the Galois
  group: $S_3$ for our case). Especially we have 3 sub-extension of
  degree 3:
  \footnote{
    because $x_1, x_2, x_3$ are roots of a cubic polynomial
  }
  $K\left(x_1\right)$, $K\left(x_2\right)$ and
  $K\left(x_3\right)$, fixed by non-normal sub-groups of order 2 -
  because $M$ is degree 2 over $K\left(x_{1,2,3}\right)$.
  \footnote{
    $K\left(x_i\right) = M^{Gal\left(M/K\left(x_i\right)\right)}$ and
    $Gal\left(M/K\left(x_i\right)\right)$ has order 2, as soon as
    $\left[M:K\left(x_i\right)\right] = 2$,
    i.e.
    $Gal\left(M/K\left(x_i\right)\right) \cong \mathbb{Z}/2\mathbb{Z}$
    - transposition of roots
  }
  And we have
  one quadratic sub-extension 
  (of degree 2) fixed by $A_3 \subset S_3$ this is
  $K\left(\sqrt{\Delta}\right)$.
  \footnote{
    The extension should be $K\left(\alpha\right)$ where $\alpha$ is a
    root of a quadratic polynomial and $\alpha \notin
    K$ but $\alpha \in M$. $\sqrt{\Delta}$ is a good choice for the
    sub-extension 
    generator because $\sqrt{\Delta} \notin K$ but $\sqrt{\Delta} \in
    M$ and $\sqrt{\Delta}$ - is a root of a quadratic polynomial
    i.e. it generates a quadratic extension. In the case
    $K\left(\sqrt{\Delta}\right) =
    M^{Gal\left(M/K\left(\sqrt{\Delta}\right)\right)} = M^{A_3}$ i.e.
    $Gal\left(M/K\left(\sqrt{\Delta}\right)\right) = A_3$.
  }

  \indexref{thm:galoiscorrespondence} says us that there are no other
  sub-extensions. Because those sub extensions correspond objectively
  to subgroups of the \indexref{def:galoisgroup}. And in this case, it
  does not have so many subgroups. These are just three subgroups of order 2
  generated by transpositions, and one subgroup of order 3 generated
  by a three cycle.  
  \label{ex:lec6_discriminant3degree}
\end{example}

\subsection{Finite fields. An infinite degree example}
\label{sec:lec6_finitefield}
We have seen that theory of finite fields is easy. Especially all
\indexref{def:galoisgroup}s are cyclic (see corollary
\ref{cor:lec3_2}). I.e. we have the field $\mathbb{F}_{p^n}$ over
$\mathbb{F}_{p}$. The Galois group is cyclic and generated by
Frobenius map (see remark \ref{rem:frobeniushomomorphism}) which is
$F_p: x \to x^p$.

More interesting are infinite extensions of a finite field, for
instance the \indexref{def:algebraicclosure}.
Thus consider $\bar{\mathbb{F}}_p$ as an extension of
$\mathbb{F}_p$. If we take an invariant generated by Frobenius $F_p$
\footnote {
  The group generated by one element $F_p$ is denoted as $\left<F_p\right>$.
}
then
\footnote{
  It required some explanation. $\left<F_p\right>$ consists of powers
  of Frobenius map: $F_p, F_{p^2}, \dots, F_{p^n}, \dots$, we also
  have that $\bar{\mathbb{F}}_p^{F_{p^n}} = \mathbb{F}_{p^n}$
  (This is because $\forall x \in \mathbb{F}_{p^n}:
  F_{p^n}\left(x\right) = x^{p^n} = x$). Therefore
  \[
  \bar{\mathbb{F}}_p^{\left<F_p\right>} =
  \cap_n \mathbb{F}_{p^n} = \mathbb{F}_p
  \]
}
\[
\bar{\mathbb{F}}_p^{\left<F_p\right>} = \mathbb{F}_p
\]
but
\[
Gal\left(\bar{\mathbb{F}}_p/\mathbb{F}_p\right) \ne
\left<F_p\right>
\]
therefore there is no bijective correspondence between sub-fields and
sub-groups. In particular the \indexref{thm:galoiscorrespondence} is
not \indexref{def:bijection} (as it was mentioned at remark
\ref{rem:lec6_gcnotbijection})

So how to see that the Galois group is not cyclic:
$Gal\left(\bar{\mathbb{F}}_p/\mathbb{F}_p\right) \ne
\left<F_p\right>$?

Really a smaller group is not cyclic.
\footnote{
  We have a theorem that a subgroup of a cyclic group is cyclic
  (see \indexref{thm:fundamentaltheoremofcyclicgroup}).
  This
  means that if the whole group is cyclic then the smaller group has
  to be a cyclic too.
}
Lets look at the following:
\[
\mathbb{F}_p \subset \mathbb{F}_{p^2} \subset \dots
\mathbb{F}_{p^{2^n}} \subset \dots
\]
and let
\[
L = \cup \mathbb{F}_{p^{2^n}}
\]
We claim that $Gal\left(L/\mathbb{F}_p\right)$ is not cyclic. Consider
the following number $a_n = 1+ 2 + 4 + \dots + 2^n$ then
$\forall x \in \mathbb{F}_{p^{2^n}}$
\begin{equation}
  F_p^{a_n}\left(x\right) = F_p^{a_m}\left(x\right)
  \label{eq:lec6_finitefieldinfinite1}
\end{equation}
for any $m > n$.
\footnote{
  If $F_p^{2^{n+l}} = id$ (see below in the text) then
  \begin{eqnarray} 
  F_p^{a_m}\left(x\right) = F_p^{a_n + 2^{n + 1} + 2^{n+2} + \dots +
    2^m}\left(x\right) =
  \nonumber \\
  = F_p^{a_n}\left(F_p^{2^{n + 1} + 2^{n+2} + \dots +
    2^m}\left(x\right)\right) =
  F_p^{a_n}\left(id\left(x\right)\right) =
  F_p^{a_n}\left(x\right).
  \nonumber
  \end{eqnarray}
}
This is because the Frobenius map $F_p$ sends $x$ to $x^p$ is an
identity on $\mathbb{F}_p$ therefore $F_p^{2^{n+l}}$ is identity on
$\mathbb{F}_{p^{2^n}}$ $\forall l \ge 0$.
\footnote{
  For example consider $F_p^{2^{n+1}}$ and let $x \in x
  \in \mathbb{F}_{p^{2^n}}$. We have
  \begin{eqnarray}
  F_p^{2^{n+1}}\left(x\right) =
  F_p^{2^n \cdot 2}\left(x\right) =
  F_p^{2^n + 2^n}\left(x\right) =
  \nonumber \\
  =
  F_p^{2^n}\left(F_p^{2^n}\left(x\right)\right) =
  F_p^{2^n}\left(x^{2^n}\right) =
  F_p^{2^n}\left(x\right) = x^{2^n} = x.
  \nonumber
  \end{eqnarray}
}
This implies
that
there exists an automorphism $\phi: L \to L$ such that
$\forall n \ge 0$
\footnote{
  ??? We can say that $\phi$ maps $L$ to $L$. If equation
  (\ref{eq:lec6_finitefieldinfinite}) holds for an arbitrary large $N$
  (i.e. informally $\mathbb{F}_{p^{2^N}}$ very
  close  to $L$)
  then it also holds $\forall n < N$ because
  (\ref{eq:lec6_finitefieldinfinite1})
}
\begin{equation}
  \left.\phi\right|_{\mathbb{F}_{p^{2^n}}} = F^{a_n}
  \label{eq:lec6_finitefieldinfinite}
\end{equation}
but $\forall k \in \mathbb{Z}$ $F_p^k \ne \phi$.
\footnote{
  i.e. $\phi \notin \left<F_p\right>$
}
One can look at
$\phi$ as $\phi = F_p^{1+2+4+ \dots + 2^n + \dots}$ but this is, of
course, very informal.
\footnote{
  Using (\ref{eq:lec6_finitefieldinfinite}) ``informally'' one can get
  \[
  \phi = \left.\phi\right|_L =
  \lim_{n \to \infty}\left.\phi\right|_{\mathbb{F}_{p^{2^n}}} =
  \lim_{n \to \infty} F^{a_n} = 
  F^{1
    +2 +4 + \dots + 2^n + \dots}
  \]  
}
The rigorous conclusions we can draw from this
is that our Galois group is not a cyclical group generated by the
Frobenius map i.e.
$Gal\left(\bar{\mathbb{F}}_p/\mathbb{F}_p\right) \ne
\left<F_p\right>$. And also, that we don't have a bijective Galois
correspondents like we have for finite field extensions i.e no
bijective correspondents  between sub-groups 
of the Galois group and sub-extensions. Indeed the fixed field of the
$F_p$ and and the whole Galois group coincide.  

\section{ Roots of unity: cyclotomic polynomials}
Consider a number $n$ that is prime to characteristic (see section
\ref{sec:fieldcharacteristic}) of $K$:
$\left(n, char\left(K\right)\right) = 1$ and consider the polynomial
$P_n = X^n - 1$ (if $\left(n, char\left(K\right)\right) = 1$
then the
polynomial has no multiple roots)
\footnote{
 if $\left(n, char\left(K\right)\right) = 1$ then $P_n' = n X^{n-1}
 \not \equiv 0 \mod char\left(K\right)$ and as result
 $gcd\left(P_n, P_n'\right) = 1$ i.e. $P_n$ does not have multiple
 roots. 
}
. Thus the polynomial has exactly $n$
roots which form a cyclic (see definition\ref{def:cyclicgroup})
multiplicative subgroup of $\bar{K}^\times$ 
(see definition \ref{def:multiplicativegroup})
\footnote{
  $\exists x \in \mu_n$ such that $x^n = 1$ i.e. $x$ is a root of $P_n$.
}
denoted by $\mu_n$. So $\mu_n$ is just the group of $n$ roots of unity
in $\bar{K}^\times$.

\begin{definition}[Primitive roots of unity]
  There are root of unity of degree $n$ such that not root of unity of
  degree $d < n$.
  \label{def:primitiverootsofunity}
\end{definition}

The set of \indexref{def:primitiverootsofunity} is denoted as
$\mu_n^*$. All elements of $\mu_n$ are powers of a single one:
$\forall x \in \mu_n, \exists a \in \mathbb{N}: x = \zeta^a$ for some
$\zeta \in \mu_n$. And primitive roots of unity form the following set
$\left\{\zeta^a\right\}$ where $\left(a, n\right) = 1$.
\label{sec:lec6_5_primitiveroot}
\footnote{
  The fact require a proof. Consider case $\left(a, n\right) = k >
  1$. Thus $\frac{n}{k}, \frac{a}{k} \in \mathbb{Z}$ and we can
  get 
  \[
  \left(\zeta^a\right)^{\frac{n}{k}} =
  \left(\zeta^{\frac{a}{k}}\right)^n = 1
  \]
  therefore $\zeta^a$ is a root of $X^{\frac{n}{k}} - 1$ and as soon
  as $\frac{n}{k} < n$ we conclude that $\zeta^a \notin \mu_n^\ast$
  (but $\zeta^a \notin \mu_n$) i.e. $\zeta^a$ is not a
  \indexref{def:primitiverootsofunity}.
}
The number of
such primitive roots is determined by \indexref{def:eulerfuction}:
$\left|\mu_n^*\right| = \phi\left(n\right)$.
\footnote{
  That is by definition of \indexref{def:eulerfuction} that counts the
  positive integers up to a given integer $n$ that are relatively
  prime to $n$  
}

\begin{definition}[$n$-th cyclotomic polynomial]
  The polynomial
  \[
  \Phi_n = \prod_{\alpha \in \mu_n^*}\left(X - \alpha\right) \in \bar{K}\left[X\right]
  \]
  is called $n$-th cyclotomic polynomial.
  \label{def:cyclotomicpolynomial}
\end{definition}

\begin{example}[$n$-th cyclotomic polynomial]
  \[
  \Phi_1 = X - 1
  \]
  \[
  \Phi_2 = \frac{X^2 - 1}{X-1} = X + 1
  \]
  \[
  \Phi_3 = \frac{X^3 - 1}{X-1} = X^2 + X + 1
  \]
  \[
  \Phi_4 = \frac{X^4 - 1}{\left(X-1\right)\left(X + 1\right)} = X^2 + 1
  \]
  \[
  \Phi_5 = X^4 + X^3 + X^2 + 1
  \]  
\end{example}

\begin{proposition}
  \begin{enumerate}
    \item 
      \(
      P_n = \prod_{d \mid n} \Phi_d
      \)
    \item $\Phi_n$ has coefficients in prime fields (see section
      \ref{sec:fieldcharacteristic}): $\mathbb{Q}$ if $char K = 0$ or
      $\mathbb{F}_p$ if $char K = p$
    \item If $char K = 0$ then $\Phi_n \in
      \mathbb{Z}\left[X\right]$. If $char K = p$ then $\Phi_n$ is the
      reduction $\mod{p}$ of the \indexref{def:cyclotomicpolynomial}
      over $\mathbb{Z}$.
  \end{enumerate}
  \begin{proof}
    The proof was marked as an exercise in the lectures
    
    The first item proof is the following. The $X^n - 1$ has $n$ roots and
    lets $\zeta$ is one of the 
    root. Let $d=ord\left(\zeta\right)$ i.e. $\zeta^d = 1$ from other side
    $\zeta^n = 1$ i.e. $d \mid n$. Therefore $\zeta \in \mu_d^\ast$
    i.e. $\zeta$ is a root of $\Phi_d$. Thus any root $X^n - 1$ will
    also be a root of $\prod_{d \mid n} \Phi_d$.
    From other side for any root $\zeta'$ of $\prod_{d \mid n}
    \Phi_d$ exists $d$ such that $\zeta'$ will be a root of $\Phi_d$
    and as soon as $d \mid n$ then $\exists l \in \mathbb{Z}: n = d
    \cdot l$ therefore $\left(\zeta'\right)^n =
    \left(\left(\zeta'\right)^d\right)^l = 1$ i.e. $\zeta'$ is also
    a root of $X^n - 1$. The both polynomial
    are \indexref{def:monicpolynomial}s and therefore are the same.
    
    Lets consider case
    $char(K) = 0$ (for the second and third parts) and lets proof that
    in the case $\Phi_d$ has coefficients in 
    $\mathbb{Z}$. Lets proof by induction. The case $n$ is trivial
    because 
    $\Phi_1 = X - 1$. Let for any $d < n$ we have that $\Phi_d$ has
    integer coefficients. Then using
    \[
    P_n = \prod_{d \mid n} \Phi_d
    \]
    we can get
    \[
    \Phi_n = \frac{X^n-1}{\prod_{d \mid n, d \ne n} \Phi_d}
    \]
    using the fact that $\prod_{d \mid n, d \ne n} \Phi_d$ has integer
    coefficients (by induction hypothesis) we can conclude that
    $\Phi_n$ has coefficients in $\mathbb{Q}$. Then using
    \indexref{lem:gauss} we can conclude that $\Phi_n \in
    \mathbb{Z}\left[X\right]$.

    Finally using
    \[
    P_n = X^n - 1 = P_n \mod p = \prod_{d \mid n} \left(\Phi_d \mod p\right)
    \]
    one can conclude that if $char K = p$ then $\Phi_n$ is the
    reduction $\mod{p}$ of the \indexref{def:cyclotomicpolynomial}
    over $\mathbb{Z}$.
    
  \end{proof}
  \label{prop:lec6_cyclotomic}
\end{proposition}

\section{Irreducibility of cyclotomic polynomial.The Galois group}

\begin{theorem}
  Let $char(K) = 0$, then $\Phi_n$ is irreducible in
  $\mathbb{Q}\left[X\right]$
  (it amounts to the same to say that this is irreducible in
  $\mathbb{Z}\left[X\right]$ as we know).
  \begin{proof}
    We have to prove that all \indexref{def:primitiverootsofunity} have
    the same minimal polynomial over $\mathbb{Q}$. It must be $\Phi_n$
    by degree reason.

    Let fix one primitive root $\zeta$ and all others have the form
    $\zeta^a$ where $a$ is prime to $n$: $\left(a,n\right)=1$.
    \footnote{
      as it was mentioned at the section \ref{sec:lec6_5_primitiveroot}.
    }
    If we
    can show these for $a$ 
    prime then we can also show this for all $a$.
    \footnote{
      Let $a = l \cdot k$ where $l$ and $k$ are prime. We have (as it
      will be proved later) that $\nu = \zeta^l$ and $\zeta$ are roots of the
      same minimal polynomial. $\nu^k$ will also be a root of that
      minimal polynomial (if $\nu$ is a root and $k$ is prime than
      $\nu^k$ will also be a root of the same polynomial), but
      $\nu^k = \zeta^{k \cdot l} = \zeta^a$.
    }
    Thus we may
    assume that $a$ is a prime number $l$
    and suppose
    \[
    P_{min}\left(\zeta, \mathbb{Q}\right)
    \ne
    P_{min}\left(\zeta^l, \mathbb{Q}\right).
    \]
    Then $\Phi_n = f \cdot g$ where $f$ has $\zeta$ as a root and $g$
    has $\zeta^l$ as a root. This is true in
    $\mathbb{Q}\left[X\right]$ but also as we seen
    \footnote{
      $\Phi_n \in \mathbb{Z}\left[X\right]$ (see proposition
      \ref{prop:lec6_cyclotomic}). \indexref{lem:gauss} says that
      if $\Phi_n$ is reducible than they factors are also in
      $\mathbb{Z}\left[X\right]$.
    }
    in
    $\mathbb{Z}\left[X\right]$. So we have $g\left(\zeta^l\right) = 0$
    thus we can define $g_l\left(X\right) = g\left(X^l\right)$ then
    $g_l$ will have $\zeta$ as a root. But $g_l \equiv g^l \mod l$.
    \footnote{
      Using $g\left(X\right) = a_n X^n + \cdots + a_1 X + a_0$ we can get
      \begin{eqnarray}
        g^l\left(X\right) = \left(a_n X^n + \cdots + a_1 X +
        a_0\right)^l =
        \nonumber \\
        = a_n {X^l}^n + \cdots + a_1 {X^l} +
        a_0 + l \cdot \left( \dots \right) \equiv g_l \mod l
        \nonumber
      \end{eqnarray}
    }
    Thus in
    modulo $l$ $\Phi_n$ has $\zeta$ as a multiple root. This is
    impossible because $\Phi_n$ divides $P_n$ and this does not have
    multiple roots whenever $l$ prime to $n$. 
  \end{proof}
  \label{thm:lec6_2}
\end{theorem}

\begin{remark}
  Statements of theorem \ref{thm:lec6_2} are not true if $char(K) >
  0$. I.e. over $\mathbb{F}_p$ $\Phi_n$ is not always irreducible.

  For instance $\Phi_8 = X^4 + 1$ is reducible over $\mathbb{F}_p$ for
  any $p$. For instance $p = 2$
  \footnote{
    $1$ is a root because $1^4 + 1 = 2 \equiv 0 \mod 2$. Therefore
    $\Phi_8 = X^4 + 1$ is reducible over $\mathbb{F}_2$
  }
  In fact it splits in $\mathbb{F}_{p^2}$.
  This is because if $p$ is odd then $8\mid p^2-1$ so the
  \indexref{def:multiplicativegroup} 
  $\mathbb{F}_{p^2}^\times$ contains a cyclic subgroup of order 8 which is
  exactly the group of 8 roots of unity.
\end{remark}

The main theorem about cyclotomic extensions is the following
\begin{theorem}
  The splitting field $L$ of $P_n$ over $K$ is $K\left(\zeta\right)$,
  where $\zeta$ is a root of $\Phi_n$.

  $\forall g \in Gal\left(L/K\right)$ acts by $g: \zeta \to
  \zeta^{a_g}$ where $\left(a_g, n\right) = 1$.

  $Gal\left(L/K\right)  \hookrightarrow
  \left(\mathbb{Z}/n\mathbb{Z}\right)^\times$ and this is an isomorphism
  whenever $\Phi_n$ is irreducible over $K$ (for example $K
  =\mathbb{Q}$). 
  \begin{proof}
    \begin{enumerate}
      \item All $n$-th roots of unity are powers of $\zeta$ so they
        are in  $K\left(\zeta\right)$.
      \item Thus any $g \in Gal\left(L/K\right)$ induces an
        automorphism of $\mu_n \subset L$ and all such automorphisms
        are raising a root to a power that is prime to $n$.
        \footnote{
          This is because $\forall g \in Gal\left(L/K\right)$
          $g$ permutes roots of the irreducible polynomial
          $\Phi_n$. The roots a generated by means of $\zeta \to
          \zeta^{a_g}$, where $\left(a_g, n\right) = 1$. Therefore
          we can associate a root permutation (i.e. $g$) with the
          following map $\zeta \to
          \zeta^{a_g}$, where $\left(a_g, n\right) = 1$.
        }
      \item $Gal\left(L/K\right)  \hookrightarrow
        Aut\left(\mu_n\right) \cong 
        \left(\mathbb{Z}/n\mathbb{Z}\right)^\times$. That
        \footnote{
          $Gal\left(L/K\right)  \hookrightarrow
          Aut\left(\mu_n\right)$
        }
        is because if a $g$ is an identity on $\mu_n$ then it is also
        identity on $L$ because $\mu_n$ generates $L$ over $K$
        \footnote{
          We have to proof there that $Gal\left(L/K\right)  \hookrightarrow
          Aut\left(\mu_n\right)$ is an \indexref{def:embedding}
          i.e. \indexref{def:injection} that preserves the group
          structure. The injection was proved before because for
          $g_1,g_2 \in Gal\left(L/K\right)$ such that $g_1 \ne g_2$ we
          have two different root exchange $\zeta \to \zeta^{a_{g_1}}$ and
          $\zeta \to \zeta^{a_{g_2}}$. About structure preserving if
          $f: g \to \left(\zeta \to \zeta_a\right)$ then easy check
          that $f\left(g_1 g_2\right) = f\left(g_1 \left) f\right(
          g_2\right)$ and that there is a homomorphism. By the way any
          embedding should preserve the identity element that was
          shown by Ekaterina.

          Comments from staff about the identity preserving: The proof
          of injectivity in the lecture goes as follows: take
          arbitrary element $g \in Gal(L/K)$. We prove, that if the
          image of $g$ is the identity in $(\mathbb{Z}/n)*$,
          then $g$ is itself the identity in $Gal(L/K)$.
          \label{note:lec6_staff_comment}
        }
       \item If $\Phi_n$ is irreducible then there is an isomorphism
         because of cardinality: $\left[L:K\right] = \deg \Phi_n =
         \phi\left(n\right)$. But
         $\phi\left(n\right) =
         \left|\left(\mathbb{Z}/n\mathbb{Z}\right)^\times\right|$. So in
         the case the embedding must be isomorphism. 
    \end{enumerate}    
  \end{proof}
  \label{thm:lec6_3}
\end{theorem}
