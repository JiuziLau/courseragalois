%% -*- coding:utf-8 -*-
\chapter{Structure of finite K-algebras continued}

We apply the discussion from the last lecture to the case of field
extensions. We show that the separable extensions remain reduced after
a base change: the inseparability is responsible for eventual
nilpotents. As our next subject, we introduce normal and Galois
extensions and prove Artin's theorem on invariants.

\section{Structure of finite K-algebras, examples (cont'd)}

Last time we have seen that a finite $K$-algebra $A$
($\left[A:K\right] < \infty$) has only finitely many maximal ideals
$m_1, \dots, m_r$ and the following equation holds (see theorem
\ref{thm:structurefinitekalgebra}):
\[
A \cong A/m_1^{k_1} \times \dots \times A/m_r^{k_r}
\]
This is a general form of \nameref{thm:chineseremainder}.

\begin{example}
  Let
  \[
  A = K\left[X\right]/\left(F\right)
  \]
  And the polynomial $F$ is not necessary
  irreducible so let's decompose into a product of irreducible
  factors:  $F= P_1^{k_1} \dots P_r^{k_r}$.
  Then by the  \nameref{thm:chineseremainder}
  \footnote{
    See also remark \ref{rem:lec5_4} and theorem
    \ref{thm:structurefinitekalgebra}
  }
  one can get
  \[
  A \cong
  K\left[X\right]/\left(P_1\right)^{k_1} \times \dots
  \times K\left[X\right]/\left(P_r\right)^{k_r},
  \]
  where $K\left[X\right]/\left(P_i\right)^{k_i} = A/m_i^{k_i}$ and
  $m_i = \left( P_i \mod F \right)$
  \footnote{
    Using definition \ref{def:quotientring} one can get that
    $P_i \in K\left[X\right]$ corresponds to
    $P_i \mod F$ in $A = K\left[X\right]/\left(F\right)$ therefore 
    we have $\left(P_i\right)$ is a \nameref{def:maxideal} for
    $K\left[X\right]$ and 
    \[
    A/\left(P_i\right)^{k_i} =
    K\left[X\right]/\left(P_i \mod F \right)^{k_i}.
    \]
    but $P_i \mod F  = P_i$ and as result
    \[
    K\left[X\right]/\left(F\right) \cong
    K\left[X\right]/\left(P_1\right)^{k_1} \times \dots
    \times K\left[X\right]/\left(P_r\right)^{k_r}.
    \]
  }
  - an ideal.
  \label{ex:lec5_1}
\end{example}

\begin{definition}[Nilpotent element]
  Let $A$ is a \nameref{def:ring} than $x \in A$ is nilpotent if $x \ne 0$ but
  $\exists k: x^k = 0$.
  \footnote{
    Alternative definition from \cite{wiki:nilpotent}: An element,
    $x$, of a ring, $R$, is called nilpotent if there exists 
    some positive integer, $n$, such that $x^n = 0$.
  }
  \label{def:nilpotent}
\end{definition}

\begin{definition}[reduced]
  $K$-algebra $A$ is reduced if it has no \nameref{def:nilpotent}s.
  Or in other words
  \footnote{
    Let we have an $i$-th element of the product $\prod A/m_i^{k_i}$
    with $k_i > 1$ and $m_i  = (p)$ when 
    $p \in A/m_i^{k_i}$ and $p \ne 0 = p^{k_i}$ i.e. $p$ is a
    nilpotent. 
  }
  if in the decomposition
  \[
  A \cong A/m_1^{k_1} \times \dots \times A/m_r^{k_r}
  \]
  $\forall i: k_i = 1$.
  Or 
  \footnote{
    $A/m_i$ is a field as soon as $m_i$ is a \nameref{def:maxideal}
  }
  if $A$ is a product of fields. 
  \label{def:reduced}
\end{definition}

\begin{definition}[local]
  \nameref{def:ring} $A$ is called local if it has only one
  \nameref{def:maxideal} i.e. $A \cong A/m^k$. 
  \label{def:local}
\end{definition}

If $A$ is local then all elements of $A$ are nilpotents
i.e. any element of $A$ is a identity,zero or nilpotent
\footnote{
  As it was mentioned in \cite{wiki:localring}, a nonzero ring in
  which every element is either a unit or nilpotent is a local
  ring, but not reverse, as it was pointed on the lectures. There is
  also an example $A \cong A/\{0\}$ where $A$ is a \nameref{def:field}
  and $\{0\}$ is the only maximal ideal for the fields (see example
  \ref{ex:maxideal}). In this case there are many non nilpotents
  different from identity and zero but the ring ($K$-algebra) is local. 
}
\footnote{
  Comment from Staff on the issue: It looks confusing because Katya
  immediately applies this definition to the case of finite algebras
  without explicitly mentioning. 

  You are right that it is not true in general. For example, discrete
  valuation rings are local and have no zero divisors at all. (Also your
  counterexample is not quite correct, because a field is exactly the 
  case when every element is either invertible or nilpotent).  
  
  But if a finite algebra has only one maximal ideal, then by the
  structural theorem it consists of nilpotents. 
}
.

Most of our last examples were examples of reduced $K$-algebras such
as
\[
\mathbb{C} \otimes_{\mathbb{R}} \mathbb{C} =
\mathbb{C} \times \mathbb{C}
\]
or
\[
\mathbb{Q}\left(\sqrt{2}\right) \otimes_{\mathbb{Q}}
\mathbb{Q}\left(i\right) =
\mathbb{Q}\left(i, \sqrt{2}\right)
\]
that is a field and if we start producing similar examples then mostly
they are reduced. Well, why? Because in fact the presence of
nilpotents has to do with inseparability. The presence of nilpotents
reflects inseparability.

So let me give you one more example: tensor product of
extensions which is not reduced. Let $K$ be a field of characteristic
$p$, for instance $\mathbb{F}_p$. Consider a field of rational
functions over $K$
\footnote{
  As it was shown in example \ref{ex:lec3_1} (part
  \ref{ex:lec3_notperfect}) it's not a  \nameref{def:perfectfield} and
  as result of theorem \ref{thm:lec3_4} is not separable.
}
: $K\left(X\right)$. We will consider
$K\left(X\right)$ as an extension of $K\left(X^p\right)$ (or with new
variable $Y = X^p$ - $K\left(Y\right)$. We will be interested in
\(
K\left(X\right)
\otimes_{K\left(Y\right)}
K\left(X\right)
\) where $X$ is a $p$th root of $Y$ so
\footnote{
  as soon as
  $K\left(X\right) = K\left[T\right]/\left(T^p - Y\right)$
}
\begin{eqnarray}
K\left(X\right)
\otimes_{K\left(Y\right)}
K\left(X\right)
\cong
\nonumber \\
\cong
K\left(X\right)
\otimes_{K\left(Y\right)}
K\left[T\right]/\left(T^p - Y\right)
\cong
\nonumber \\
\cong
K\left(X\right)\left[T\right]/\left(T^p - Y\right) =
\nonumber \\
=
K\left(X\right)\left[T\right]/\left(T^p - X^p\right) =
K\left(X\right)\left[T\right]/\left(T - X\right)^p
\nonumber
\end{eqnarray}
where $T$ is another variable. As result we have got a ring with
nilpotents for example $T - X$ and of course the reason is that our
extension $K\left(X\right)$ is pure inseparable extension (see
definition \ref{def:inseparabledegree}) of $K\left(Y\right)$.  

\section{Separability and base change}
What is the reason for such a mysterious connection between 
presence of nilpotents and separability? If $L$ is separable over $K$
then the number of \nameref{def:homomorphism}s
$\left|Hom_K\left(L, \bar{K}\right)\right|$ is maximal and equal to
degree $\left[L : K\right]$ but in general it is less or equal to the
degree.  This is of course clear, because  if we have a polynomial
with distinct roots, then it's stem field for instance  has exactly
this number of  homomorphisms into the-algebraic closure and this
number is equal to the number of roots. So if some roots coincide,  
then the number of homomorphisms diminishes. 

Lets also recall \nameref{thm:basechange}. If $L$ and $E$ are
extensions of $K$ and $L$ is finite over $K$ then
\[
Hom_K\left(L, E\right) \cong
Hom_E\left(
L \otimes_K E, E
\right).
\]
In the formula, $L \otimes_K E$ is a finite $E$-algebra denoted as $A$
below.
\begin{remark}
  The remark is not a part of the lectures but it is important to
  understand the below content.

  We have that $A = L \otimes_K E \cong E \otimes_K L$ is a free $E$
  module as soon as $L$ is a free $K$ module and with proposition
  \ref{prop:lec4_Addon} we have that $\left[A : E\right] < \infty$
  (as soon as $\left[L : K\right] < \infty$)
  and
  as result with theorem \ref{thm:structurefinitekalgebra} one can get
  that there are finitely many maximal ideals $m_i$ and
  \[
  A \cong A/m_1^{k_1} \times \dots \times A/m_r^{k_r}
  \]
  \label{rem:lec5_own1}
\end{remark}

\begin{definition}
  With \nameref{thm:chineseremainder} theorem we have
  \[
  A \cong A/m_1^{k_1} \times \dots \times A/m_r^{k_r}
  \]
  Reduced algebra $A_{red}$ is defined by the following equation
  \[
  A_{red} = A/m_1 \times \dots \times A/m_r
  \]
  \label{def:reducedalgebra}
\end{definition}

We have that
\footnote{
  i.e. nilpotents become zeros in the $A_{red}$.
}
\[
A_{red} = A /\eta\left(A\right)
\]
where $\eta\left(A\right)$ is an \nameref{def:ideal} of
nilpotents in $A$.

It is clear that
\[
Hom_E\left(A, E\right) =
Hom_E\left(A_{red}, E\right)
\]
because all homomorphism into a field must be zero on all nilpotents
\footnote{
  It requires some clarification. Consider a homomorphism
  $\phi \in Hom_E\left(A, E\right)$. $\forall x,y \in A, \phi(xy) =
  \phi(x)\phi(y)$. Let $x \in \eta\left(A\right)$ i.e. $x$ is a
  nilpotent then $x \ne 0_A, x^k = 0_A$. We have
  \[
  0_E = \phi(x^k) = \phi(x)^k
  \]
  i.e. $\phi(x) =  0_E$. Therefore all nilpotents go to zero and,
  instead of $A$ (as the set the $\phi$ acts on), we can consider
  $A_{red}$. As result, we will get that $\phi(0_{A_{red}}) = 0_E$ and all
  other properties of homomorphism are also hold, for instance
  $\forall \bar{x},\bar{y} \in A_{red}: \phi(\bar{x}+\bar{y}) =
  \phi(\bar{x}) + \phi(\bar{y})$. Really
  $\bar{x} = x + \eta(A), \bar{x} = x + \eta(A)$ and
  \[
  \phi(\bar{x}+\bar{y}) =
  \phi(x+y) = \phi(x) + \phi(y) =
  \phi(\bar{x}) + \phi(\bar{y})
  \]
  as soon as $\phi(\eta(A)) = 0_E$.
}. 

So again, we see that if there are nilpotents in the tensor product,
then there is somehow fewer space for homomorphisms. Because if A is
not reduced, then the dimension
\[
\left[A_{red} : E\right] < \left[A : E\right].
\]
So the maximal number of homomorphisms, so let's say the slogan
``Maximal number of homomorphisms'' is attained when $A$ is reduced
and all quotients
\begin{equation}
  A/m_i \cong E
  \label{eq:lec5_1}
\end{equation}
because those quotients are of
course extensions of $E$.
\footnote{
  as it was mentioned above $A = L \otimes_K E$ is a finite
  $E$-algebra i.e. $A$ is a $E$-extension. The $A/m$ is also
  $E$-algebra (see remark
  \ref{rem:lec5_quotientkalgebra}) i.e. also $E$-extension.

  From other hand we consider $Hom\left(A/m_i, E\right)$ i.e. there is
  a homomorphism (injection) from $A/m_i \to E$ and as result
  $A/m_i \cong E$. These arguments are also used in the text below for
  the fact explanation. 
}
In general,    
those quotients are extensions of $E$ (see remark
  \ref{rem:lec5_quotientkalgebra}). We also have
\[
A \cong A/m_1 \times \dots \times A/m_r
\]
but $Hom\left(A/m_i, E\right) = \{0\}$ if $\left[A/m_i: E\right] > 1$.  
This is because a field homomorphism is always injective. A field
homomorphism a homomorphism of fields which are extensions of $E$  an
$E$-homomorphism is injective. So you cannot map an $E$-vector space of
dimension greater than 1 into an $E$-vector space of dimension 1.

Lets take $E = \bar{K}$ then automatically we will get
$A/m_i \cong E$ because an algebraically closed field does not have a
non trivial finite extension.

So what have we had (see also example \ref{ex:lec5_conrad})?
\[
A = L \otimes_K \bar{K},
\]
\begin{equation}
  A_{red} = \prod_{i=1}^r \bar{K}.
  \label{eq:lec5_AredIsProd}
\end{equation}
The following one $A = A_{red}$ is the same to $r$ is maximal and
equal to $\left[L:K\right] = \left[A: \bar{K}\right]$.
\footnote{
  It is because there are $\left[L:K\right] = r$ roots of a
  polynomial and all the roots are in $\bar{K}$. Each root $\alpha_i$
  forms a polynomial $X - \alpha_i$ which creates an ideal $m_i =
  \left(X - \alpha_i\right)$. We also have
  $L = K\left(\alpha_1, \dots, \alpha_r\right)$ and
  \[
  A = L \otimes_K \bar{K} \cong \bar{K} \otimes_K L =
  \bar{K} \otimes_K K\left(\alpha_1, \dots, \alpha_r\right).
  \]
  It expands to (see example \ref{ex:lec4_1})
  \begin{eqnarray}
    A \cong \bar{K} \otimes_K \frac{K\left[X\right]}{(X-\alpha_1)
      \dots (X-\alpha_r)} \cong
    \nonumber \\
    \cong
    \frac{\bar{K}\left[X\right]}{\left(X - \alpha_1\right)}
    \times
    \dots
    \times
    \frac{\bar{K}\left[X\right]}{\left(X - \alpha_r\right)} 
    \cong \bar{K} \times \dots \times \bar{K}.
    \nonumber
  \end{eqnarray}
}
In the case
\[
r = \left|Hom_{\bar{K}}\left(A, \bar{K}\right)\right| = 
\left|Hom_K\left(L, \bar{K}\right)\right|
\]
So this explains why seperability is the same thing as the absence of
nilpotents. So let me formulate it as a theorem.

\begin{theorem}
  Let $L$ is a finite extension over $K$ then
  \begin{enumerate}
  \item $L$ is separable if and only if $L \otimes_K \bar{K}$
    is \nameref{def:reduced}.
    $L$ is pure inseparable if and only if $L \otimes_K \bar{K}$
    is \nameref{def:local}
  \item $L$ is separable if and only if for all algebraic extension
    $\Omega$, $L \otimes_K \Omega$ is reduced.
    $L$ is pure inseparable if and only if for all algebraic extension
    $\Omega$, $L \otimes_K \Omega$ is local.
  \item If $L$ is separable then the map
    \[
    \phi: L \otimes_K \bar{K} \to \bar{K}^n
    \]
    which sends
    \[
    \phi\left(l \otimes k\right) =
    \left(
    k \phi_1\left(l\right),
    \dots,
    k \phi_n\left(l\right)
    \right)
    \]
    where $\phi_i$ are distinct homomorphisms from $L$ to $\bar{K}$, 
    is an isomorphism.
  \end{enumerate}
  \begin{proof}
    \begin{enumerate}
    \item $L$ separable is the same thing that the algebra
      $A = L\otimes_K \bar{K}$ has $\left[L:K\right]$ factors
      \footnote{
        If we have $L = K\left(\alpha\right)$
        (see theorem \ref{thm:lec3_3}) then
        there exists a minimal polynomial $P_{min}\left(\alpha,
        K\right)$ of degree $r = \left[L:K\right]$. The polynomial
        splits and has $r$ roots: $\alpha_1, \dots, \alpha_r$. Thus we
        have $r$ maximal ideals
        \(
        m_1 = \left(X - \alpha_1\right), \dots,
        m_r = \left(X - \alpha_r\right)
        \).
        For each maximal ideal we have $A/m_i \cong A$ (see example
        \ref{ex:lec1_fieldquotionisomorphism}) thus with theorem
        \ref{thm:structurefinitekalgebra} and (\ref{eq:lec5_1}) we have
        \[
        A \cong \prod_{i=1}^r \bar{K}
        \]
      }
      $\bar{K}$ which is the same as $A$ is \nameref{def:reduced}
      since $\dim_{\bar{K}} A = \left[L:K\right]$.
      \footnote{
        From (\ref{eq:lec5_AredIsProd}), if $A = A_{red}$, one can get
        $\dim_{\bar{K}} A = \dim_{\bar{K}} A_{red} = \dim_{\bar{K}}
        \prod_{i = 1}^r \bar{K} = r = \left[L:K\right]$.
      }

      $L$ is pure inseparable: this means that exists only one
      homomorphism of $L$ into $\bar{K}$ i.e. $A$ has only one
      $\bar{K}$-homomorphism into $\bar{K}$ thus only one factor and
      as result $A$ is \nameref{def:local}.
    \item If $\Omega$ is an algebraic extension then
      \footnote {
        i.e. we can setup an \nameref{def:injection} and structure
        preserving map $f: L \otimes_K \Omega \to L \otimes_K
        \bar{\Omega}$ (see definition \ref{def:embedding}) 
      }
      \[
      L \otimes_K \Omega \hookrightarrow L \otimes_K \bar{\Omega} =
      L \otimes_K \bar{K}.
      \]
      There is a sub-ring and so one easily checks, that a sub-ring of
      a reduced algebra is reduced and same for local.  
    \item Leave as an excises
      \footnote{
        We have that
        $\forall l \otimes k \in L \otimes_K \bar{K}, \exists
        \bar{k}^n = \left(
        k \phi_1\left(l\right),
        \dots,
        k \phi_n\left(l\right)
        \right) \in \bar{K}^n$.

        From other side $\{\phi_i\}$ - distinct homomorphisms and can
        be considered as distinct \nameref{def:character}s. Therefore
        by theorem \ref{thm:dedekind} the homomorphisms are linearly
        independent i.e. form a basis in $\bar{K}^n$. I.e.
        any $\bar{k}^n \in \bar{K}^n$ can be represented as
        \[
        \bar{k}^n = k
        \left(
        \phi_1\left(l\right),
        \dots,
        \phi_n\left(l\right)
        \right)
        \]
        where $k \in \bar{K}$ and $l \in L$.
      }
    \end{enumerate}
  \end{proof}
  \label{thm:lec5_1}
\end{theorem}
\begin{remark}
In general for modules $M$, $N$ and $P$ over a ring $R$ \textbf{not true} that
if $M \hookrightarrow N$ ($M$ is a sub module of $N$) then
$M \otimes_R P \hookrightarrow N \otimes_R P$. But this become the
truth if $R$ is a field and as result $M,N,P$ are
\nameref{def:vectorspace}s. So, for my field extensions, I can say
that if I have an extension and then I take a base change, then it
remains an extension, but  you should not think that the same thing is
true for arbitrary modules over a ring.  
\end{remark}

\begin{example}
  The example is not a part of lectures and was taken from
  \cite{bib:KeithConradSeparability2}.
  Consider extension $\mathbb{Q}\left(\sqrt{2}\right)$ over
  $\mathbb{Q}$. Since
  \[
  \mathbb{Q}\left(\sqrt{2}\right) \cong
  \mathbb{Q}\left[X\right]/\left(X^2 - 2\right)
  \]
  tensoring with $\mathbb{Q}$ gives
  \begin{eqnarray}
    \mathbb{Q}\left(\sqrt{2}\right) \otimes_{\mathbb{Q}}
    \bar{\mathbb{Q}} \cong
    \bar{\mathbb{Q}} \otimes_{\mathbb{Q}}
    \mathbb{Q}\left(\sqrt{2}\right)
    \cong
    \nonumber \\
    \cong
    \bar{\mathbb{Q}} \otimes_{\mathbb{Q}}
    \mathbb{Q}\left[X\right]/\left(X^2 - 2\right)
    \cong
    \nonumber \\
    \cong
    \bar{\mathbb{Q}}\left[X\right]/
    \left(
    \left(X - \sqrt{2}\right)
    \left(X + \sqrt{2}\right)
    \right)
    \cong
    \nonumber \\
    \cong
    \bar{\mathbb{Q}}\left[X\right]/\left(X - \sqrt{2}\right)
    \times
    \bar{\mathbb{Q}}\left[X\right]/\left(X - \sqrt{2}\right)
    \cong
    \bar{\mathbb{Q}} \times \bar{\mathbb{Q}}
    \nonumber
  \end{eqnarray}
  We used the following fact (see example
  \ref{ex:lec1_fieldquotionisomorphism}) 
  \[
  \bar{\mathbb{Q}}\left[X\right]/\left(X \pm \sqrt{2}\right)
  \cong
  \bar{\mathbb{Q}}
  \]  
  \label{ex:lec5_conrad}
\end{example}

\section{Primitive element theorem}

\begin{definition}[Idempotent]
  The element $x$ is called idempotent if $x \cdot x = x$
  \label{def:idempotent}
\end{definition}

\begin{theorem}[Primitive element]
  Let $L$ is a finite \nameref{def:separableextension} of $K$ then it
  has only finitely many sub extensions i.e. $E$ such that $K \subset
  E \subset L$.
  \begin{proof}
    So, let's base change to $\bar{K}$
    \footnote {
      see proof of theorem \ref{thm:lec5_1} (second part of it).
    }  
    :
    \(
    E \otimes_{K} \bar{K} \hookrightarrow
    L \otimes_{K} \bar{K}
    \)
    \footnote{
      See also proof of theorem \ref{thm:lec5_1}.
    }.
    We also have (see also equation (\ref{eq:lec5_AredIsProd}))
    \[
    E \otimes_{K} \bar{K} \cong \bar{K}^m
    \]
    and
    \[
    L \otimes_{K} \bar{K} \cong \bar{K}^n
    \]
    are reduced $\bar{K}$ sub-algebras generated by
    \nameref{def:idempotent}s namely by
    \[
    e_i = \left(0,0, \dots, 1, \dots, 0\right),
    \]
    where $1$ is in $i$-th place
    \footnote{
      this is because the typical element $k \in \bar{K}^n$ has the
      following form
      \[
      k = \left(k_1, k_2, \dots, k_n\right) = \sum_{i=1}^n k_i e_i,
      \]
      where $k_i \in \bar{K}$. 
    }.

    On the other hand $L \otimes_K \bar{K} \cong \bar{K}^n$ has only
    finitely many \nameref{def:idempotent}s because 
    $\left(a_1, \dots, a_i, \dots, a_n\right)$ is an idempotent if and
    only if all $a_i$ are 0 or 1 and therefore there
    are only finitely many ways to choose $m$ idempotents out of them,
    \footnote{
      we have the following equation
      \[
      \bar{K}^m \hookrightarrow \bar{K}^n
      \]
    }
    so there is only finitely many ways to generate a subalgebra.  
  \end{proof}
  \label{thm:primitiveelement}
\end{theorem}

\begin{corollary}[Primitive element theorem]
  $\exists \alpha \in L$ such that $L = K\left( \alpha \right)$
  whenever $L$ is finite and separable.
  \begin{proof}
    And this is easy to see, of course, because
    if $L$ and $K$ are infinite, then 
    $L$ cannot be a union, a finite union of proper subextension. 
    A vector space over an infinite field is not a finite union of
    proper subspaces.   For instance a plane is not a finite union of
    lines.
    \footnote{
      It will require some additional explanations. Took
      $\alpha \in L$ such that $P_{min}\left(\alpha, K\right)$ has
      maximal degree. If $K\left(\alpha\right) = L$ we complete and
      found the primitive element. If not then let
      $\beta \in L \setminus K\left(\alpha\right)$. Consider the
      following element $\gamma_a = \alpha + a \beta$ where $a \in
      K$. For any $a \in K$ exists $K\left(\gamma_a\right)$ such that
      $K \subset K\left(\gamma_a\right) \subset L$. We have
      $\left|K\right| = \infty$ but the 
      number of sub-extensions is limited by theorem
      \ref{thm:primitiveelement} therefore $\exists a, b \in K$ such that
      $a \ne b$ and $K\left(\gamma_a\right) =
      K\left(\gamma_b\right) = K\left(\gamma\right)$ where $\gamma =
      \gamma_a$. 

      We have $\gamma_a - \gamma_b = (a-b) \beta \in K\left(\gamma\right)$, i.e.
      $\beta \in K\left(\gamma\right)$. Therefore
      $\alpha = \gamma_a -a \beta = \gamma -a \beta \in
      K\left(\gamma\right)$. We also 
      have (as soon as $\beta \notin 
      K\left(\alpha\right)$) $K \subset K\left(\alpha\right)
      \subsetneq K\left(\gamma\right) 
      \subset L$. Thus $\left[K\left(\gamma\right):K\right] >
      \left[K\left(\alpha\right):K\right]$. Therefore
      (see proposition \ref{prop:dimextension})
      \[
      \deg\left(P_{min}\left(\gamma, K\right)\right) >
      \deg\left(P_{min}\left(\alpha, K\right)\right)
      \]
      that is in contradiction with $\alpha$ choose.

      Note that \cite{bib:KenBrownPrimitiveElementTheorem} has another,
      more known, proof for the fact and prove that if
      $L = K\left(\alpha, \beta\right)$ then $\exists \lambda \in K$
      such that $\gamma = \alpha + \lambda \beta$ is the primitive
      element i.e. $L = K\left(\gamma\right) = K\left(\alpha,
      \beta\right)$. 
    }
    
    If $L$ and $K$ are \nameref{def:finitefield}s, then we have
    already described this situation completely. We have described all
    finite extensions and have seen that they are generated by one element.
    \footnote{
      As it was mentioned in the proof of corollary \ref{cor:lec3_1}
      we can take $\alpha = \mbox{ generator of } K^\times$. For more info
      see corollary \ref{cor:lec3_1}.
    }
  \end{proof}
  \label{col:primitiveelement}
\end{corollary}

\section{Examples. Normal extensions}

\subsection{Examples}

\begin{example}[Primitive element]
  \[
  \mathbb{Q}\left(\sqrt{2}, \sqrt{3}\right) =
  \mathbb{Q}\left(\sqrt{2} + \sqrt{3}\right).
  \]
  We have
  $\left[\mathbb{Q}\left(\sqrt{2}, \sqrt{3}\right) :
    \mathbb{Q}\right] = 4$ so all sub-extensions are quadratic.
  \footnote{
    As soon as extension has degree $4 = 2 \cdot 2$ then a
    sub-extension should have 
    degree 2 and the minimal polynomial should be quadratic.
  }
  As no
  quadratic polynomial has $\alpha = \sqrt{2} + \sqrt{3}$ for a root
  \footnote{
    Quadratic polynomials have very simple formula for roots with
    only one square (discriminant) and it is not possible to get 2
    squares with it 
  },
  $\alpha$ generates $\mathbb{Q}\left(\sqrt{2}, \sqrt{3}\right)$.

  This must be a primitive element, generates our field. 
  It is not contained in any proper subextension.

  There is another proof (not part of the lectures) that shows that
  $\beta = \sqrt{2} + \sqrt{3}$ is a primitive element i.e.
  $\sqrt{2}, \sqrt{3}$ are generated by $\beta$. Really
  $\beta^2 = 5 + 2 \sqrt{2}\sqrt{3}$ i.e.
  \[
  \sqrt{2}\sqrt{3} = \frac{\beta^2 - 5}{2}.
  \]
  From other side
  \[
  \sqrt{2} \beta = \sqrt{2}\left(\sqrt{2} + \sqrt{3}\right) =
  2 + \sqrt{2}\sqrt{3} = \frac{\beta^2 - 1}{2}.
  \]
  Therefore
  \[
  \sqrt{2} = \frac{\beta^2 - 1}{2 \beta}
  \]
  and
  \[
  \sqrt{3} = \beta - \frac{\beta^2 - 1}{2 \beta}.
  \]
  \label{ex:lec5_primitiveelement}
\end{example}

\begin{example}[Extension which cannot be generated by a single element]
  So, take $K$ equal to $\mathbb{F}_p$ and consider
  $K\left(x,y\right)$ as an
  extension of $K\left(x^p,y^p\right)$. It has degree $p^2$
  \footnote {
    $K \subset K\left(x^p\right) \subset K\left(x^p,y^p\right)$ and
    $\left[K\left(x^p\right): K\right] = p$ as well as
    $\left[K\left(x^p, y^p\right): K\left(x^p\right)\right]$ thus with
    theorem \ref{thm:mulformuladegrees}
    \[
    \left[K\left(x^p, y^p\right): K\right] = p^2.
    \]
  }
  i.e.
  \[
  \left[
    K\left(x,y\right) : K\left(x^p,y^p\right)
    \right] = p^2.
  \]
  We have $\forall \alpha \in K\left(x,y\right) \setminus
  K\left(x^p,y^p\right)$ is of degree $p$ over
  $K\left(x^p,y^p\right)$. This is because
  $\alpha^p \in K\left(x^p,y^p\right)$
  \footnote{
    $\alpha = k_1 x + k_2 y$, where $k_1, k_2 \in K =
    \mathbb{F}_p$. Using remark \ref{rem:frobeniushomomorphism} we can
    get
    \[
    \alpha^p = k_1^p x^p + k_2^p y^p \in K\left(x^p, y^p\right).
    \]
  }
  . So, no element like these can
  generate our extension.  
\end{example}

\subsection{Normal extensions}

\begin{definition}[Normal extension]
  A normal extension of $K$ is a \nameref{def:splittingfield} of a
  family of polynomials
  \footnote {
    There is a set of polynomials (can be only one polynomial in the
    set) and the polynomials not necessary to be irreducible. Example
    $\mathbb{Q}/\mathbb{Q}$ - is a normal 
    extension because there is a set of polynomials split in it :
    $\left\{X-1, X^2-1\right\}$. Note that any irreducible polynomial
    (another definition below) that has a root in it also splits, for
    instance $X - a$, where $a \in \mathbb{Q}$ is an irreducible, has
    a root $a \in \mathbb{Q}$ and splits in it. From other side an
    arbitrary polynomial ( for example $X^3-1$ ) can have a root in
    $\mathbb{Q}$ but does not necessary split in it. 
  }
  in $K\left[X\right]$.
  \footnote{
    Another good definition of a normal extension
    \cite{wiki:normalextrus} can also be used. Normal extension $E/K$ is
    such algebraic extension in which every irreducible polynomial
    $P(X) \in K\left[X\right]$ that has a single root in $E$ splits in
    $E$.   
  }
  \label{def:normalextension}
\end{definition}

\begin{remark}[Normal extension]
  So, take a bunch of polynomials in $K$ and we adjoin all their roots
  to $K$, and this is what is called a normal extension.
  For instance, a \nameref{def:splittingfield} of one polynomial is
  also a normal extension. 
\end{remark}

\begin{theorem}
  The following conditions are equivalent for an extension $L$ of $K$:
  \begin{enumerate}
  \item $\forall x \in L$ $P_{min}\left(x, K\right)$ splits in $L$.
  \item $L$ is \nameref{def:normalextension}
  \item All \nameref{def:homomorphism}s from $L$ to $\bar{K}$ have the
    same image.
    \footnote{
      And the image is $L$ because $id$ is also a
      \nameref{def:homomorphism}
      ??? There are 2 staff's comments about the claim:
      \begin{itemize}
      \item You can't naturally identify $L$ with a particular
        subfield of $\overline{K}$ (unless $L$ is given as a
        subfield of $\overline{K}$ a priori). But from the lecture
        2.5 we know, that there exists an embedding $L
        \hookrightarrow \overline{K}$ since $L$ is algebraic
        over $K$. And the theorem says that images of all such
        embeddings are equal if $L$ is normal over $K$. 
      \item My previous answer was a little bit misleading. I meant,
        if you have an arbitrary algebraic extension of $K$, there
        is no canonical embedding into $\overline{K}$. Consider, for
        example, a field $\mathbb{Q}/(x^3-2)$. It has several
        embeddings into $\overline{\mathbb{Q}}$,
        namely, $\mathbb{Q}(\sqrt[3]{2})$, $\mathbb{Q}(e^{2\pi
          i/3}\sqrt[3]{2})$ and $\mathbb{Q}(e^{2\pi
          i/3}\sqrt[3]{2})$. But if we consider the splitting field
        of $x^3-2$, then it's image in $\overline{\mathbb{Q}}$
        is $\mathbb{Q}(\sqrt[3]{2}, e^{2\pi i/3})$. So, yes, if we
        identify roots of polynomials which define $L$ with some
        elements of $\overline{K}$ (up to a permutation for each
        polynomial), then the image is $L$ itself. 
      \end{itemize}
    }
  \item The \nameref{def:group} of \nameref{def:automorphism}s
    $Aut\left(L/K\right)$ acts transitively (see definition
    \ref{def:transitive}) on this set of 
    homomorphisms $Hom_K\left(L, \bar{K}\right)$.
  \end{enumerate}
  \label{thm:lec5_3}
  \begin{proof}
    1 implies 2: Take
    $\left(P_i\right)_{i \in I} = \left\{P_{min}\left(x, K\right) \mid
    x \in L\right\}$ - the set of polynomials
    (i.e. the family of polynomials).
    $L$ will be a splitting field of the set $\left(P_i\right)_{i \in
      I}$ and therefore (by definition \ref{def:normalextension}) $L$ is normal.

    2 implies 3: Let
    $S = \left\{\mbox{roots of } P_i, i \in I \mbox{ in } L\right\}$
    and
    $S' = \left\{\mbox{roots of } P_i, i \in I \mbox{ in }
    \bar{K}\right\}$
    \footnote{
      $S$ and $S'$ are 2 sets of roots
    }
    then any homomorphism
    $\phi: L \to \bar{K}$ sends $S$ to $S'$, but $S$ generates $L$
    over $K$, so $\phi\left(S\right)$ determines $\phi\left(L\right)$
    \footnote{
      As soon as $\phi(S) = S'$ then we have one set of roots and as
      result one image (that is generated by the set) for
      homomorphisms. Or more concrete, let $l \in L, S = \{s_i\}, S' =
      \{s'_i\}$. We have $l = \sum l_{ik}s_i^k$ where $l_{ik} \in K$.
      For the homomorphism $\phi$ we have
      $\phi(l) = \sum l_{ik}{s'}_i^k$ i.e. the image $\phi(L)$ consists
      of elements of the following form: $\sum l_{ik}{s'}_i^k$. For any
      homomorphism we will have the same form for the image or in
      other words - any homomorphism has the same image.
    }
    .

    3 implies 4: Let $j, j' \in Hom_K\left(L, \bar{K}\right)$
    (i.e. we took 2 homomorphisms)
    then
    they send $L$ isomorphically to its image $L'$. So, these are
    isomorphisms from $L$ to $L'$. So

    \begin{tikzpicture}[descr/.style={fill=white,inner sep=2.5pt}]
      \matrix (m) [matrix of math nodes, row sep=3em,
        column sep=3em]
              { & L' & \\
                L & & L\\ };
              \path[->,font=\scriptsize]
              (m-2-1) edge node[descr] {$ j' $} (m-1-2)
              (m-1-2) edge node[descr] {$ j^{-1} $} (m-2-3)
              (m-2-1) edge node[descr] {$ j^{-1}j' $} (m-2-3);
    \end{tikzpicture}
    
    take $j^{-1} \cdot j' \in Aut \left(L/K\right)$ and it sends $j$
    to $j'$
    \footnote{
      We have got that $\forall j, j' \in Hom_K\left(L,
      \bar{K}\right), \exists g \in Aut\left(L/K\right)$
      such that $g(j) = j'$, for instance $g = j^{-1}j'$ and
      \begin{equation}
        g(j) \equiv j g = j j^{-1} j' = j'.
        \label{eq:lec5_groupaction}
      \end{equation}
        Therefore group
      automorphisms acts transitively by the definition
      \ref{def:transitive}
    }.

    4 implies 1:  I have this \nameref{def:transitive} and I have to prove
    that any minimal polynomial splits. Consider
    $P_{min}\left(x, K\right)$. $\alpha_1, \dots, \alpha_n$ - roots in
    $\bar{K}$. Then I have map $K\left(x\right) \to
    K\left(\alpha_i\right)$ that extends to
    $j_i: L \xrightarrow[x \to \alpha_i]{} \bar{K}$. This is by
    theorem \nameref{thm:lec2_3}.
    $\exists \theta_i \in Aut\left(L/K\right)$ such that
    $j_1 \theta_i = j_i$
    \footnote{
      as it was mentioned at the equation (\ref{eq:lec5_groupaction})
    }
    thus
    $\alpha_i \in j_1\left(L\right)$
    \footnote{
      $\theta_i: L \to L$ thus $j_1 \theta_i: L \to j_1(L)$
    }
    or all roots are in
    $j_1\left(L\right)$ and the polynomial
    $P_{min}\left(x, \bar{K}\right)$ splits over $j_1\left(L\right)$
    but this means that it splits over $L$
    \footnote{
      If $x = \alpha_1$ then $j_1 = id$ and $j_1(L) = L$
    }
  \end{proof}
\end{theorem}

\section{Galois extensions}

Now we are ready to give a definition for central object of Galois
theory

\begin{definition}[Galois extension]
  A Galois extension is a \nameref{def:normalextension} and
  \nameref{def:separableextension}.
  \label{def:galoisextension}
\end{definition}

\begin{theorem}
  Let $L$ be a finite over $K$ then the number of automorphisms
  $Aut\left(L/K\right)$ is less or equal to degree
  $\left[L:K\right]$:
  \[
  \left|Aut\left(L/K\right)\right| \le \left[L:K\right].
  \]
  The equality holds if and only if $L$ is
  \nameref{def:galoisextension}.
  \label{thm:lec5_4}
  \begin{proof}
    We know that the group of automorphisms $Aut\left(L/K\right)$ acts
    freely (see definition \ref{def:freeaction}) on the set
    $Hom_K\left(L, \bar{K}\right)$,
    \footnote{
      Note that the number of distinct roots is equal to the number of
      distinct homomorphisms (see proposition \ref{prop:lec3_2}). Thus
      we can set an association between 
      roots and homomorphisms. Proposition \ref{prop:stemfield}
      says that the stem 
      field isomorphism (that translated to the considered
      automorphism) is unique determined by it's value on the roots that
      it exchanges. I.e. the the action of $Aut\left(L/K\right)$ on
      the set of roots (and as result on the set $Hom_K\left(L,
      \bar{K}\right)$) is exactly that one defined for 
      \nameref{def:freeaction}.

      See also first and second remarks at \ref{rem:lec5_onnormalext}.
    }
    so the number of 
    automorphisms $\left|Aut\left(L/K\right)\right|$ is equal to the
    number of \nameref{def:orbit} of this action which is less or
    equal
    \footnote{
      in ideal case each element has an unique orbit and the number of
      elements and the number of orbits are equal.
    }
    to the cardinality of the set it self:
    $\left|Hom_K\left(L, \bar{K}\right)\right|$. The equality holds
    whenever (if and only if) \nameref{def:action} is
    \nameref{def:transitive}.
    \footnote{
      This follows directly from the definition of
      \nameref{def:transitive} i.e. $\forall h_1, h_2 \in
      Hom_K\left(L, \bar{K}\right)$ $\exists a \in
      Aut\left(L/K\right)$  such that $a(h_1) = h_2$.
    }
    We just seen in 
    theorem \ref{thm:lec5_3} that this means that $L$ is normal over
    $K$. So we have
    \[
    \left|Aut\left(L/K\right)\right| \le
    \left|Hom_K\left(L, \bar{K}\right)\right| \le
    \left[L:K\right].
    \]
    The first inequality become equality if $L$ is normal and the
    second one if $L$ is separable
    \footnote{
      see definitions \ref{def:separabledegree} and
      \ref{def:separableextension}. 
    }, thus
    \[
    \left|Aut\left(L/K\right)\right| \le \left[L:K\right]
    \]
    and equality holds if $L$ is both normal and separable i.e. if
    it's \nameref{def:galoisextension}.
  \end{proof}
\end{theorem}

\begin{definition}[Set of invariants]
  If group $G$ acts on a set $X$ then $X^G = \left\{x
  \in X \mbox{ such that } gx = x, \forall g \in G\right\}$ - the
  set of invariants.
  \label{def:setinvariants}
\end{definition}

\begin{remark}[on normal extensions]
  If $L$ is normal over $K$ then
  \begin{enumerate}
  \item If we have an \nameref{def:isomorphism} of sub-extensions
    ($K \subset L_1, L_2 \subset L$)
    $\phi: L_1 \cong L_2$ then it extends to an
    \nameref{def:automorphism} of $L$.  To see this, we embed $L$ into
    an algebraic closure $\bar{K}$. And remark that $\phi$ extends to a
    map from $L_1$ to $\bar{K}$,
    \footnote{
      Ekaterina really said that but may be more better to say that
      $\phi$ extends to a map from $L$ to $\bar{K}$ ???
    }
    but all those maps have the same
    image (see theorem \ref{thm:lec5_3}), namely $L$
    \footnote{
      Consider $L_1 \subset L_2 \subset L \subset \bar{K}$ we have a
      homomorphism $\phi_1: L_1 \to L_2$ (our isomorphism) that
      accordingly theorem \ref{thm:lec2_3}, can be extended to
      homomorphism $L_1 \to \bar{K}$ and the last one can be extended
      to $\tilde{\phi_1}: L \to \bar{K}$. We can also consider
      $L_2 \subset L_1 \subset L \subset \bar{K}$ and homomorphism
      $\phi_2: L_2 \to L_1$ that can extended (using the same
      approach) to  $\tilde{\phi_2}: L \to \bar{K}$. Both
      $\tilde{\phi_{1,2}}$ has the same image $L$ as soon as $L$ -
      normal extension (see theorem \ref{thm:lec5_3}). Thus we have an
      \nameref{def:automorphism}. 
    }.  
  \item The group of automorphisms $Aut\left(L/K\right)$ acts
    transitively on the roots of any irreducible polynomial $P \in
    K\left[X\right]$. Again, an isomorphism of stem fields extends to
    an automorphism of $L$
    \footnote{
      In the case we have \nameref{def:transitive} i.e. for any roots
      $x,y$ we have an isomorphism of stem fields $\phi$ such that
      $\phi(x) = y$ (see proposition \ref{prop:stemfield}). This
      isomorphism extends to an automorphism accordingly the
      prev. remark.  
    }.
  \item If the group $Aut\left(L/K\right)$ fixes (see definition
    \label{rem:item:lec5_onnormalext_3}
    \ref{def:fixedpoint}) some element $x \notin K$ then $x$ is pure
    inseparable (see definitions 
    \ref {def:deginseppol} and \ref{def:degsepelem}) i.e.
    $P_{min}\left(x, K\right)$ has a 
    single root $x$. Thus if $L$ is
    \nameref{def:galoisextension} (i.e. is separable) then
    the set of elements which are fixed  by the automorphisms  of $L$
    over K is just K itself: 
    $L^{Aut\left(L/K\right)} = K$
    (see definition \ref{def:setinvariants}).
    \footnote{
      Another explanation is the following. Let $x \in L \setminus K$
      then $g \in Aut\left(L/K\right)$ such that $g \ne id$ permutes
      roots of irreducible $P_{min}\left(x, K\right)$ and $g(x) = x'
      \ne x$. The only thing that works for 
      $\forall g \in Aut\left(L/K\right)$ is $K$ i.e.
      $\forall k \in K$ and $\forall g \in Aut\left(L/K\right)$ we
      have $g(k) = k$. This is because $P_{min}\left(x, K\right) = X
      - k$ i.e. only one root possible.
    }
  \end{enumerate}
  \label{rem:lec5_onnormalext}
\end{remark}

\begin{definition}[Galois group]
  If $L$ is \nameref{def:galoisextension} then Galois group $G =
  Gal\left(L/K\right)$ is the group of automorphisms $Aut\left(L/K\right)$. 
  \label{def:galoisgroup}
\end{definition}

Thus we can write
\begin{equation}
  L^{Gal\left(L/K\right)} = K.
  \label{eq:lec5_2}
\end{equation}

\section{Artin's theorem}

Motivated by (\ref{eq:lec5_2}) let formulate and proof the important
theorem
\begin{theorem}[Artin]
  $L$ is a field and $G \subset Aut\left(L\right)$
  \begin{enumerate}
  \item If $G$ acts with finite orbits,  so, I mean all orbits of G
    are finite (i.e. $\forall x \in L: \left|Orb(x)\right| < \infty$),
    then $L$ is a \nameref{def:galoisextension} of $L^G$. 
  \item If $\left|G\right| = n < \infty$ then
    $\left[L : L^G\right] = n$ and $G$ is a \nameref{def:galoisgroup}
  \end{enumerate}

  \begin{remark}
    Well notice, that acting with finite orbits  and being finite is not
    the same thing. So, a short remark before giving a proof: notice
    that finite orbits does not mean finiteness because it's typical for
    Galois groups to act with finite orbits.  If we have some G, which
    is Galois of $L$ over $K$: $G = Gal\left(L/K\right)$, and
    $x \in L$, then $x$ is a root of a 
    polynomial of some finite degree  and it's splitting field is
    finite over $K$, so, the orbit of $x$ is also finite 
    because it's always sent to another root of the
    same polynomial and so consists of roots of  the
    $P_{min}\left(x, K\right)$. But of course the Galois group itself
    $Gal\left(L/K\right)$ can be infinite when $L$ is not finite over $K$. For
    instance, if  $K  = \mathbb{F}_p$ and $L = \overline{\mathbb{F}_p}$. It is
    very easy to compute all the Galois groups, and in fact we shall see
    shortly what is exactly this Galois group of $L$ over $K$ is infinite.
    \footnote{
      Section \ref{sec:lec6_finitefield} shows that
      $Gal\left(\bar{\mathbb{F}}_p/\mathbb{F}_p\right)$ is not cyclic,
      but theorem \ref{thm:lec3_2} says that if $G$ is finite than it
      should be cyclic. Thus we can conclude that
      $Gal\left(\bar{\mathbb{F}}_p/\mathbb{F}_p\right)$ is not finite
    }
  \end{remark}
  \begin{proof}
    \begin{enumerate}
    \item  Let me take $x$, well say, $x_1 \in L$ which is not
      $G$-invariant: $x_1 \in L \setminus L^G$ and
      $G$-\nameref{def:orbit} of $x$ 
      $Orb(x) = \{x_1, x_2, \dots, x_k\}$. The
      polynomial $P\left(X\right) = \prod_{i=1}^k\left(X - x_i\right)$
      is $G$-invariant
      \footnote{
        I.e. $\forall g \in G: g\left(P\left(X\right)\right) =
        P\left(X\right)$
      }
      . $G$ just permutes  the $x_i$, it permutes the factors of these
      polynomial, 
      so the polynomial is $G$-invariant. Therefore its coefficients are
      $G$-invariant and as result $P \in L^G\left[X\right]$ by
      definition \ref{def:setinvariants}. $L^G$ is a field of $G$
      invariants, and it is 
      separable. $P$ is separable, 
      because all $x_i$ are distinct ( there are distinct elements of the
      orbit ). And $L$ is splitting field of $P$, therefore $L$ is a
      \nameref{def:galoisextension} over $L^G$ by the
      \nameref{def:galoisextension} definition.
    \item We have $\left|G\right| = n$ then
      $\forall y \in L: \left|Orb\left(y\right)\right| \le n$. Take
      $x$ as above ($x \in L \setminus L^G$)
      $\left[L^G\left(x\right): L^G\right] \le n$.
      \footnote{
        $x$ is a root of the following polynomial $P\left(X\right) =
        \prod_{i=1}^k\left(X - x_i\right)$, where $x_i \in
        Orb(x)$. The polynomial has degree $\deg P =
        \left|Orb(x)\right| \le n$ and
        $\deg P_{min}\left(x, L^G\right) \le \deg P \le n$ therefore
        $\left[L^G\left(x\right): L^G\right] \le n$.
      }
      Claim that this
      implies $\left[L:L^G\right] \le n$. If I knew already, that $L$ is
      finite over $L^G$, this would be very easy, this would be just a
      direct consequence of \nameref{thm:primitiveelement} theorem. I
      would say that $L$ is generated by one element. I take this one element as
      my $x$ and I see that $L$ is of degree at most $n$ over $L^G$. But I
      don't know yet that $L$ is finite so I have to do some trick. So,
      proof of the claim:  take $x$ such that
      $\left[L^G\left(x\right):L^G\right]$ is maximal then take $y \in
      L$. $L^G\left(x,y\right)$ is finite over $L$
      \footnote{
        because $x,y$ are algebraic elements and using theorem
        \ref{thm:lec1_2} and proposition \ref{prop:lec1_1} we can
        conclude that $L^G\left(x,y\right)$ is finite over $L$.
      }
      and I can apply
      \nameref{thm:primitiveelement} theorem. Therefore
      $L^G\left(x,y\right) = L^G\left(z\right)$. But
      \[
      \left[L^G\left(x\right):L^G\right] \ge
      \left[L^G\left(z\right):L^G\right]
      \]
      thus $L^G\left(x\right) = L^G\left(z\right)$ so
      $y \in L^G\left(x\right)$ and since I can do this for any $y$, I
      eventually conclude that $L = L^G\left(x\right)$, and in
      particular, $\left[L: L^G\right] \le n$. Well, now if this is
      strictly less than $n$, then $L$ cannot have $n$ automorphisms
      over $L^G$ but $G \subset Aut\left(L/L^G\right)$ so this is a
      contradiction. Therefore $\left[L: L^G\right] = n$ and
      $G = Aut\left(L/L^G\right)$ (see theorem \ref{thm:lec5_4}).
    \end{enumerate}
  \end{proof}
  \label{thm:artin}
\end{theorem}
