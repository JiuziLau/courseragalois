%% -*- coding:utf-8 -*-
\chapter{Solvability by radicals, Abel's theorem. A few words on
  relation to representations and topology}

We finally arrive to the source of Galois theory, the question which
motivated Galois himself: which equation are solvable by radicals and
which are not? We explain Galois' result: an equation is solvable by
radicals if and only if its Galois group is solvable in the sense of
group theory. In particular we see that the "general" equation of
degree at least 5 is not solvable by radicals. We briefly discuss the
relations to representation theory and to topological coverings.

\section{Extensions solvable by radicals. Solvable groups. Example} 

\subsection{Extensions solvable by radicals}

Let $K$ is a field of characteristic 0: $char(K) = 0$. It is embedded
into its \nameref{def:algebraicclosure}.

\begin{definition}[Extension solvable by radicals]
  A finite extension $E$ of $K$ is solvable by radicals if
  $\exists \alpha_1, \dots, \alpha_r$ generating $E$ such that
  $\alpha_i^{n_i} \in K\left(\alpha_1, \dots, \alpha_{i-1}\right)$ for
  some $n_i \in \mathbb{N}$.
  \label{def:solvableextension}
\end{definition}

\begin{example}
  Let $K = \mathbb{Q}$, $E = \mathbb{Q}\left(\sqrt[3]{2 + 3
    \sqrt{7}}, \sqrt[5]{4 + 5 \sqrt{11}}\right)$. We have
    $\alpha_1 = \sqrt{7}, \alpha_2 = \sqrt{11},
    \alpha_3 = \sqrt[3]{2 + 3\sqrt{7}}, \alpha_4 = \sqrt[5]{4 + 5
      \sqrt{11}}$. 
\end{example}

\begin{definition}[Polynomial solvable by radicals]
  $P \in K\left[X\right]$ is called solvable by radicals if exists a
  $E$ - \nameref{def:solvableextension} and containing all roots of
  $P$. 
  \label{def:solvablepolynomial}
\end{definition}
So more precisely, it would say that the equation, $P = 0$ is solvable
by radicals.

\begin{property}
  \begin{enumerate}
  \item \nameref{def:compositeextension} of solvable by radicals is itself
    solvable by radicals
  \item If $L$ extension of $K$ is solvable by radicals (by definition
    $L$ should be finite extension of $K$) then exists a finite
    \nameref{def:galoisextension} $E$ containing $L$ and solvable by
    radicals.    
  \end{enumerate}
  \label{property:solvable}
  \begin{proof}
    For the first property (the proof is missing in the lectures): let
    $L = L_1 L_2$ where $L_1$ and $L_2$ are solvable i.e.
    $L_1 = K\left(\alpha_1, \dots, \alpha_n\right),
    L_2 = K\left(\beta_1, \dots, \beta_m\right)$ with $\alpha_i,
    \beta_j$ which satisfies properties from definition
    \ref{def:solvableextension}. In the case we can assume
    $L = L_1 L_2 = K\left(L_1 \cup L_2\right) = 
    K\left(\alpha_1, \dots, \alpha_n, \beta_1, \dots, \beta_m\right)$
    and all properties from definition
    \ref{def:solvableextension} will also be satisfied. Therefore
    the composite extension $L = L_1 L_2$ is also solvable.
    
    For the second property: Indeed take a composite of all images of $L$ in
    $\bar{K}$ or, in other words, images of $L$ by
    $Gal\left(\bar{K}/K\right)$.
    \footnote{
      Lets prove by induction. For the first step we have that
      $K\left(\alpha\right)$ is solvable therefore
      $\exists d = \min\{j \mid \alpha^j \in K\}$. Let
      $\zeta_d$ is a root of unity i.e. root of $X^d - 1$ then by
      proposition \ref{prop:lec7_1}, the
      $K\left(\zeta_d, \alpha\right) \supset K\left(\alpha\right)$ is
      a Galois extension and it has a finite degree ($\le d \cdot d$).

      Using induction hypothesis we have that if
      $K\left(\alpha_1, \dots, \alpha_i\right)$ is solvable then there
      exists a Galois extension of finite degree
      $F = K\left(\beta_1, \dots, \beta_k \right)$ that is solvable
      and $K\left(\alpha_1, \dots, \alpha_i\right) \subset F$.
      By \nameref{thm:galoiscorrespondence} (item
      \ref{thm:galoiscorrespondence:item2b}) we have that
      $\forall g \in G = Gal\left(\bar{K}/K\right)$:
      \[
      g\left(F\right) = F.
      \]
      Lets consider $F\left(\alpha_{i+1}\right)$. As we know
      (by induction hypothesis)
      $K\left(\alpha_1, \dots, \alpha_i, \alpha_{i+1}\right)$ is
      solvable i.e.
      \(
      \exists d = \min\{k \mid \alpha_{i+1}^k
      \in K\left(\alpha_1, \dots, \alpha_i\right)\}
      \)
      therefore $\alpha_{i+1}$ is a root of the following irreducible
      polynomial $X^d - a$ where $a = \alpha_{i+1}^d$.
      The Galois group $G = Gal\left(\bar{K}/K\right)$ permutes the
      roots of the polynomial and as result $\forall g \in G$
      \[
      L_g = g\left(F\left(\alpha_{i+1}\right)\right) =
      F\left(\beta_g\right),
      \]
      where $\beta_g = g\left(\alpha_{i+1}\right)$ is a root of $X^d -
      a$. I.e. $L_g$ is solvable. If we take a composite extension
      $L \cup_{g \in G} L_g$ then by first property of
      \ref{property:solvable} it will be solvable but as soon as
      $L$ is the image of all $g \in G$ then
      $\forall g \in G: g\left(L\right) = L$ and by
      \nameref{thm:galoiscorrespondence} (item
      \ref{thm:galoiscorrespondence:item2b})  the extension $L$ is
      Galois. $L$ is a splitting field of $X^d - a$ (because it
      contains all its roots) but by theorem \ref{thm:lec2_1} it has
      finite degree: $\left[L : F\right] \le d!$.
    }
  \end{proof}
\end{property}

\subsection{Solvable groups}
This shall be a brief reminder since this is not a course on group
theory, you are supposed to know some group theory already. So I
somehow I presume that you are familiar with this definition but I
will recall the definition of basic properties.

\begin{definition}[Solvable group]
  $G$ is called solvable if it has a filtration 
  i. e. $G = G_0 \supset G_1 \supset \dots \supset G_{r-1} \supset G_r
  = \left\{e\right\}$, such that $G_i$ is a
  \nameref{def:normalsubgroup} of 
  $G_{i-1}$ and the \nameref{def:quotientgroup} $G_{i-1}/G_i$ is
  an \nameref{def:abeliangroup}.
  \label{def:solvablegroup}
\end{definition}

\begin{example}[Group of permutations $S_3$]
  Consider $S_3$ - the group of permutations (see also example
  \ref{ex:s3group}). It's solvable because 
  $S_3 \supset A_3 \supset \left\{e\right\}$.

  We have $\left|S_3/A_3\right| = 2$ (see example
  \ref{ex:s3a3quotientgroup}) i.e. $S_3/A_3$ is cyclic of order 
  2. $\left|A_3\right| = 3$ i.e. $A_3$ - cyclic of order 3.
  \footnote{
    As it was mentioned at \cite{wiki:finitegroup} there is only one
    group of order 3. We also have (with theorem
    \ref{thm:cyclic_group_is_abelian}) that $A_3$ is
    \nameref{def:abeliangroup} 
  }
  \label{ex:lec8_s3}
\end{example}

\begin{example}[Group of permutations $S_4$]
  Consider $S_4$ - the group of permutations (see also example
  \ref{ex:s3group}). It's solvable because 
  $S_4 \supset A_4 \supset K \supset \left\{e\right\}$, where $K$ -
  is a subgroup
  \footnote{
    There is a well known Klein four group $V_4$ \cite{wiki:klein4group} -
    the only non-cyclic group of order 4. It also denoted as $K_4$. We
    have already seen it at example \ref{ex:lec7_cyclotomic8}.
  }
  of double transpositions
  (see example
  \ref{ex:permutation} for permutation cycles notation):
  \[
  K = \left\{
  e, (12)(34), (13)(24), (14)(23)
  \right\}.
  \]
  A double transposition is a product of two \nameref{def:transposition}s with
  distinct support, right, which permute the distinct elements.
  $A_4 \triangleleft S_4$, $\left|S_4/A_4\right| = 2$, i.e.
  $S_4/A_4$ is cyclic of order 2.
  \footnote{
    i.e. theorem
    \ref{thm:cyclic_group_is_abelian} gives us that $S_4/A_4$ is
    \nameref{def:abeliangroup}
  }

  $K \triangleleft A_4$, $\left|A_4/K\right| = 3$, i.e.
  $A_4/K$ is cyclic of order 3.
  \footnote{
    i.e. theorem
    \ref{thm:cyclic_group_is_abelian} gives us that $A_4/K$ is
    \nameref{def:abeliangroup}
  }
  
  $K$ is \nameref{def:abeliangroup} and
  $K \cong \mathbb{Z}/2\mathbb{Z} \times \mathbb{Z}/2 \mathbb{Z}$.

  So this shows that $S_4$ is solvable.
  \label{ex:lec8_s4}
\end{example}

\section{Properties of solvable groups. Symmetric group}

\begin{property}
  If $G$ is solvable and $H \subset G$ is a subgroup of $G$ then $H$
  is solvable.
  \begin{proof}
    Indeed $G_i \cap H$ gives a filtration with required property.
    \footnote{
      We have $H_i = G_i \cap H$ and as soon as
      $G_i \triangleleft G_{i-1}$ we also get
      $H_i \triangleleft H_{i-1}$. Really $\forall h \in H_{i-1}$ we
      have $h \in G_{i-1} \cap H \subset G_{i-1}$. Thus, using normality
      (see definition \ref{def:normalsubgroup})
      $G_i \triangleleft G_{i-1}$,
      \[
      h G_{i} = G_{i} h.
      \]
      The last equation also holds for any subset of $G_{i-1}$ and
      especially for $H_{i} = G_{i} \cap H \subset G_{i}$. I.e.
      $\forall h \in H_{i-1}: h H_i = H_i h$ or in other words,
      $H_i \triangleleft H_{i-1}$.

      We have that $G_{i-1}/G_i$ is an
      \nameref{def:abeliangroup} and now we have to prove that the
      \nameref{def:quotientgroup} 
      $H_{i-1}/H_i$ is abelian. $\forall h' \in H_{i-1}/H_i, \exists
      h_{i-1} \in H_{i-1}$ such that $h' = \left\{h_{i-1},
      H_i\right\}$. From other hand $h_{i-1} \in G_{i-1}$ and
      $H_i \subset G_i$ therefore we can associate with $h'$ the
      $g' =  \left\{h_{i-1}, G_i\right\} \in G_{i-1}/G_i$. I.e. we just
      got an injection $f: H_{i-1}/H_i \xrightarrow[h' \to g']{}
      G_{i-1}/G_i$. Obviously we have $\forall h',h'' \in H_{i-1}/H_i$
      the following relation
      \[
      f\left(h' h''\right) = g' g'' =
      f\left(h'\right) f\left(h''\right),
      \]
      where $g'' = \left\{h'', G_i\right\}$ and
      $g' g'' = \left\{h' h'', G_i\right\}$. Therefore $f$ is
      \nameref{def:homomorphism}. Thus $\forall a,b \in H_{i-1}/H_i$
      we have $f(ab) = f(a) f(b)$, but as soon as $G_{i-1}/G_i$ is
      abelian, $f(a) f(b) = f(b) f(a) = f(ba)$, or, using property of
      homomorphism, one can get that $ab=ba$ i.e. $H_{i-1}/H_i$ is an
      \nameref{def:abeliangroup} and as result $H$ is solvable.
    }
  \end{proof}
  \label{property:lec8_solvable1}
\end{property}

\begin{property}
  If $G$ is solvable and $H \triangleleft G$ is a normal subgroup of
  $G$ then $G/H$ is solvable.
  \begin{proof}
    Indeed consider a projection map
    \begin{equation}
      \pi: G \to G/H
      \label{eq:lec8_solvable_pi}
    \end{equation}
    then $\pi\left(G_i\right)$ gives a filtration $\left(G/H\right)_i$
    on $G/H$ with required properties.
    \footnote{
      We can associate each element $g \in G$ with
      $\bar{g} \in G/H$ via the following map:
      $\pi: G \xrightarrow[g \to \bar{g}]{} G/H$. I.e.
      $\pi(g) = \bar{g} = gH$. Thus we also have
      $\pi\left(G_i\right) = \bar{G}_i = G_i H$. Lets prove that
      $\bar{G}_i$ forms the required filtration i.e. that
      $\bar{G}_i \triangleleft \bar{G}_{i-1}$
      and $\bar{G}_{i-1}/\bar{G}_i$ is \nameref{def:abeliangroup}.

      For normality (see definition \ref{def:normalsubgroup}) prove we
      have that 
      ${G}_i \triangleleft {G}_{i-1}$ i.e.
      $\forall g \in {G}_{i-1}: g G_i = G_i g$ but
      $\forall \bar{g} \in \bar{G}_{i-1}$ we have
      \[
      \bar{g} \bar{G}_{i} = g H G_i H
      \]
      but $H \triangleleft G$ i.e.
      $\forall g' \in G_i \subset G: g' H = H g'$ therefore
      \[
      \bar{g} \bar{G}_{i} = g G_i H =
      G_i g H = G_i g H H = G_i H g H = \bar{G}_i \bar{g}
      \]
      that finished the normality proof.

      For the abelian quotient proof we have that
      ${G}_{i-1}/{G}_i$ is \nameref{def:abeliangroup} therefore
      $\forall a,b \in {G}_{i-1}: a b G_i = b a G_i$. Using the fact
      that $H$ is a \nameref{def:normalsubgroup} one can get
      \begin{eqnarray}
        \bar{a} \bar{b} \bar{G}_i =
        a H b H G_i H = a b G_i H = b a G_i H =
        \nonumber \\
        = b a G_i H H = b a H G_i H = b a H H G_i H =
        b H a H G_i H =
        \nonumber \\
        = \bar{b} \bar{a} \bar{G}_i,
        \nonumber 
      \end{eqnarray}
      i.e. $\bar{G}_{i-1}/\bar{G}_i$ is \nameref{def:abeliangroup}.
    }
  \end{proof}
  \label{property:lec8_solvable2}
\end{property}

\begin{property}
  If $H \triangleleft G$, $H$ and $G/H$ are solvable then $G$ is
  solvable.
  \label{property:lec8_solvable3}
  \begin{proof}
    Put togeter the filtration $H_i$ and
    $\pi^{-1}\left(\left(G/H\right)_j\right)$ (see
    (\ref{eq:lec8_solvable_pi}) for $\pi$ definition).
    \footnote{
      $\forall \bar{g} \in \bar{G} = G/H$ we can consider it as a set
      of elements: $\bar{g} = \{\forall h \in H: h g\}$ where
      $g \in G$.
      If we combine then we will get the following: $\cup_g \bar{g} =
      G$.
      We also have
      \begin{equation}
        G H = \cup_g \bar{g} H =
        \cup_g g H H = \cup_g g H = \cup_g \bar{g} = G.
        \label{eq:lec8_solvable3_note_bar}
      \end{equation}
      Thus from $\bar{G}_i \triangleleft \bar{G}_{i - 1}$ one can
      get $\forall \bar{g} \in \bar{G}_{i - 1}: \bar{g} \bar{G}_i =
      \bar{G}_i \bar{g}$ or in other words
      (with $G_{i - 1} = \cup_{\bar{g} \in \bar{G}_{i - 1}} \bar{g}$)
      $\forall g \in G_{i - 1}$ we have the following
      \[
      g G_{i - 1} = g H G_{i - 1} H =
      \bar{g} \bar{G}_{i-1} =
      \bar{G}_{i-1} \bar{g} =
      G_{i - 1} H g H =
      G_{i - 1} H H g = G_{i - 1} g,
      \]
      i.e.
      ${G}_i \triangleleft {G}_{i - 1}$.
      As result the filtration
      \[
      \{e\} = \bar{H} \triangleleft \dots \triangleleft
      \bar{G}_i \triangleleft \bar{G}_{i - 1} \triangleleft \dots
      \triangleleft \bar{G}_0 = \bar{G} 
      \]
      produces the following one
      \begin{equation}
      H \triangleleft \dots \triangleleft
      G_i \triangleleft G_{i - 1} \triangleleft \dots
      \triangleleft G_0 = G.
      \label{eq:lec8_solvable3_note}
      \end{equation}
      If $\bar{G}_{i - 1}/\bar{G}_{i}$ is abelian, i.e.
      $\forall \bar{a},\bar{b} \in \bar{G}_{i - 1}$:
      $\bar{a} \bar{b} \bar{G}_{i} = \bar{b} \bar{a} \bar{G}_{i}$, we
      can get the following
      $\forall a, b \in G_{i-1}$ (as soon as
      (\ref{eq:lec8_solvable3_note_bar}) gives us $G_i = G_i H$)
      \begin{eqnarray}
      a b G_i = a b G_i H = a b G_i H H =
      a b H G_i H = a b H H G_i H = a H b H G_i H =
      \nonumber \\
      =
      \bar{a} \bar{b} \bar{G}_{i} = \bar{b} \bar{a} \bar{G}_{i} =
      b H a H G_i H = b a H G_i H = b a G_i H = b a G_i
      \nonumber
      \end{eqnarray}
      i.e. the \nameref{def:quotientgroup} $G_{i-1}/G_i$ is an Abelian
      group.

      Finally combining (\ref{eq:lec8_solvable3_note}) with $H$
      solvability one can get
      \[
      \{e\} \triangleleft \dots
      \triangleleft
      H_j \triangleleft H_{j - 1} \triangleleft \dots
      \triangleleft H
      \triangleleft \dots \triangleleft
      G_i \triangleleft G_{i - 1} \triangleleft \dots
      \triangleleft G_0 = G
      \]
      i.e. $G$ is solvable.
    }
  \end{proof}
\end{property}

The following property is not a part of the lectures but it's required
for the property \ref{property:lec8_solvable4} proof
\begin{property}
  If $G$ is finite \nameref{def:abeliangroup} than $G$ is solvable and
  there exists a finite filtration with cyclic quotients for $G$.
  \begin{proof}
    Accordingly theorem \ref{thm:fgagroup}) each finite
    \nameref{def:abeliangroup} $G$ can be represented as
    \[
    G = K_1 \oplus K_2 \oplus \dots \oplus K_{n-1} \oplus K_n,
    \]
    where $K_i$ is a \nameref{def:cyclicgroup}. If we denote
    \[
    G_i = K_1 \oplus K_2 \oplus \dots \oplus K_{i} 
    \]
    when, by property \ref{property:directproduct}, we can get
    \[
    \{e\} = G_0 \triangleleft G_1 \triangleleft \dots \triangleleft
    G_n = G.
    \]
    The gotten filtration has cyclic quotients because
    (see property \ref{property:directproductquotient})
    $G_i/G_{i-1} = K_i$, where $K_i$ is a cyclic group.
  \end{proof}
  \label{property:lec8_solvable4add}
\end{property}

\begin{property}
  If $G$ is finite than $G$ is solvable (i.e. has a finite filtration with
  Abelian quotients) if and only if there exists a
  finite filtration with cyclic quotients.
  \label{property:lec8_solvable4}
  \begin{proof}
    This is just because a finite \nameref{def:abeliangroup} is just a
    product of cyclic groups (see see \nameref{thm:fgagroup}).
    \footnote{
       If exists a finite filtration of $G$ with cyclic quotient then
      the $G$ is solvable because \nameref{def:cyclicgroup} is
      abelian (see theorem \ref{thm:cyclic_group_is_abelian}).

      The other direction is not so simple. Lets $G$ is solvable then
      $G_i \triangleleft G_{i - 1}$ and $G_{i-1} / G_{i} = K$ -
      abelian. The property \ref{property:lec8_solvable4add} (that
      follows from \nameref{thm:fgagroup}) says us that there exists a
      finite filtration with cyclic quotients for $K$ i.e.
      \[
      \{e\} = K^{(n)} \triangleleft \dots \triangleleft
      K^{(j)} \triangleleft K^{(j - 1)} \triangleleft \dots \triangleleft
      K_0 = K
      \]
      where $K^{(j-1)}/K^{(j)}$ is cyclic. The \nameref{thm:correspondence}
      says us that $\forall K^{(j)}, \exists G_{i}^{(j)} \subset G_{i-1}$
      with the same properties i.e.
      $G_i^{(j)} \triangleleft G_i^{(j-1)}$ and
      $G_i^{(j-1)}/G_i^{(j)}$ is cyclic. As result we can complete
      each $G_i \triangleleft G_{i-1}$ with
      \[
      G_i = G_i^{(n)} \triangleleft \dots
      \triangleleft G_i^{(j)} \triangleleft G_i^{(j-1)} \triangleleft \dots
      \triangleleft G_i^{(0)} = G_{i-1}
      \]
      and the result filtration will have cyclic quotients.
      %% The $K$ is a finite \nameref{def:abeliangroup}. But by
      %% theorem \ref{thm:simple_subgroup_of_abelian} any finite abelian
      %% group $K$ has a simple (see \nameref{def:simplegroup}) subgroup
      %% $K_1$. The group $K/K_1$ is again abelian, so it has a simple
      %% subgroup $K_2/K_1$ we can continue until we reach $K$. Thus we
      %% get a set of subgroups
      %% of $G_i$ containing $G_{i-1}$ that we can
      %% insert in the original series; repeat for each $i$ and finally
      %% you get a composition series with abelian factors.        
    }
  \end{proof}
\end{property}

Lets also look at another definition of solvable group
\begin{definition}[Solvable group]
  $G$ is called solvable if the following sequence is finite:
  \[
  G
  \supseteq \left[G, G\right] = G^{(1)}
  \supseteq \left[G^{(1)}, G^{(1)}\right] = G^{(2)}
  \supseteq \dots \supseteq
  \left[G^{(n-1)}, G^{(n-1)}\right] = G^{(n)} = \left\{e\right\}
  \]
  where $G^{(i)} = \left[G^{(i-1)}, G^{(i-1)}\right]$ is the
  \nameref{def:commutatorsubgroup}.
  \label{def:solvablegroupadd}
\end{definition}

\begin{remark}
  Definitions of solvable group \ref{def:solvablegroupadd} and
  \ref{def:solvablegroup} are equivalent.
  \begin{proof}
    Our filtration with \nameref{def:commutatorsubgroup}s
    $G \supseteq G^{(1)} \supseteq \dots \supseteq G^{(n)} =
    \left\{e\right\}$
    is a filtration with abelian quotient because
    $G/\left[G, G\right]$ as well as 
    $G^{(i)}/\left[G^{(i)}, G^{(i)}\right] = G^{(i)}/G^{(i+1)}$ are
    \nameref{def:abeliangroup}s.
    \footnote{
      See definition \ref{def:abelianization} and theorem
      \ref{thm:about_quotient_group_and_commutatorsubgroup}
    }

    From the other hand if $G/H$ is an \nameref{def:abeliangroup} then
    $H \supset \left[G, G\right]$.
    \footnote{
      See theorem \ref{thm:about_quotient_group_and_commutatorsubgroup}
    }
    So if a finite filtration with
    abelian quotient exists then the filtration given by $G^{(i)}$ is
    also finite. It must terminate after a finite steps. So, this
    proves the equivalence.  
  \end{proof}
  \label{rem:lec8_solvable}
\end{remark}

\begin{theorem}[$S_n$ solvability]
  $S_n$ - the permutation of $n$ elements (see example
  \ref{ex:sngroup}) is not solvable for $n \ge 5$. 
  \begin{proof}
    It's easy to use definition \ref{def:solvablegroupadd}.
    Main steps are the following
    \begin{enumerate}
      \item we know that $\left[S_n, S_n\right] = A_n$ - subgroup of
        even permutations (see definition
        \ref{def:alternatinggroup}). It can be seen from the fact that
        any 3-cycle is a \nameref{def:commutatorsubgroup}
        \footnote{
          Let $n \ge 3$ and $(a,b,c) \in S_n$ is a 3-cycle then
          (see example \ref{ex:transposition_product})
          \begin{eqnarray}
            (a,b,c) = (a,b,c)^2 =
            \left((a,c)(a,b)\right)^2 =
            \nonumber \\
            = (a,c)(a,b)(a,c)(a,b) =
            (c,a)^{-1}(b,a)^{-1}(a,c)(a,b) =
            \nonumber \\
            =
            (a,c)^{-1}(a,b)^{-1}(a,c)(a,b) =
            \nonumber \\
            =
            \left[
              (a,c),(a,b)
              \right] \subset \left[S_n, S_n\right].
            \nonumber
          \end{eqnarray}
        }
        and 3-cycles
        generate $A_n$
        \footnote{
          \label{note:lec8_3cycle}
          Any even permutation can be represented in a form of product
          of transpositions (see theorem
          \ref{thm:permutationrepresent}). The number of
          transpositions should be 
          even. We have several cases there:
          \begin{enumerate}
            \item The transpositions with different elements produces
              2 3-cycles
              \[
              (a,b)(c,d) = (a,b,c)(b,c,d).
              \]
              Really
              \[
              (a,b,c)(b,c,d) = 
              \begin{array}{c}
                c \to d \\
                d \to b \to c \\
                b \to c \to a \\
                a \to b 
              \end{array} =
              (a,b)(c,d)
              \]
            \item The transpositions that have a same element produces
              a 3-cycle (see example \ref{ex:transposition_product})
              \[
              (a,c)(a,b) = (a,b,c)
              \]
          \end{enumerate}
          As soon as we have even number of transposition then we will
          always have a translation to a product of 3-cycles for any
          product of transposition pairs.
        }
      \item If $n \ge 5$ then $\left[A_n, A_n\right] = A_n$ thus the
        filtration generated by commutators will never terminate
        i.e. will never reach the unity ($\left\{e\right\}$) and will
        stabilize on $A_n$. How we can see it? We can remember that
        $\left[A_4, A_4\right] = K$ (see example \ref{ex:lec8_s4}) -
        the subgroup of double transpositions. $A_4 \hookrightarrow
        A_n$ in many different ways. Because you can pick any 4
        elements among our $n$ elements and just consider the
        permutations of 
        those 4 elements as a subgroup of permutations of $n$
        elements and then taking the commutators of those $A_4$, we
        see
        that all double transpositions are in the $\left[A_n,
          A_n\right]$ (\nameref{def:commutatorsubgroup} of $A_n$).
        \footnote{
          The double transposition consists of 4 elements. We can
          consider all permutations of the 4 elements only and ignore
          all others. For instance if $n = 6$ and we are interested in
          all double transpositions of $1,2,4,5$. In the case we have
          to look at the following permutations:
          \[
          \pi_{1,2,4,5} = \begin{pmatrix}
            1 & 2 & 3 & 4 & 5 & 6\\
            i_1 & i_2 & 3 & i_3 & i_4 & 6
          \end{pmatrix}.
          \]
          Note that we consider the even permutations only i.e.
          $\pi_{1,2,4,5} \in A_6$.
          From example \ref{ex:lec8_s4} we know that the commutator of
          such permutations contains all double transpositions for $1,2,4,5$:  
          $\left\{e, (12)(45), (14)(25), (15)(24) \right\}$ (the
          elements $3,6$ are not changed and disappeared in the
          commutator). Continue such way we can conclude that for any
          double transposition:
          \[
          (i_1,i_2)(i_3,i_4) \in \left[A_n, A_n\right]
          \]
        }
        But
        if $n \ge 5$, they generate $A_n$.
        \footnote{
          Really if $n \ge 5$ then for any 3 elements $a,b,c$ (which form a
          3-cycle $(a,b,c)$) exist 2 elements $e,d$ such that all
          elements $a,b,c,d,e$ are different. Using the
          $(e,d)(e,d) = id$ one can get that 2 double transpositions
          $(a,b)(d,e)$ and $(d,e)(b,c)$ generate the 3-cycle $(a,b,c)$:
          \[
          (a,b)(d,e)(d,e)(b,c) = (a,b)(b,c) = (a,b,c).
          \]

          I.e. any 3-cycle can be generated by 2 double
          transpositions. Therefore (see note \ref{note:lec8_3cycle})
          the whole $A_n$ is generated by 
          the double transpositions.

          Thanks Arnur Nigmetov for the hint.
        }
    \end{enumerate}
  \end{proof}
  \label{thm:lec8_sn_solvability}
\end{theorem}

\section{Galois theorem on solvability by radicals}

\begin{theorem}
  Let $P \in K\left[X\right]$. $P$ is a
  \nameref{def:solvablepolynomial} if and only if
  $Gal\left(P\right)$ is solvable. There
  $Gal\left(P\right)$ is (by definition) $Gal\left(F/K\right)$ where
  $F$ is a \nameref{def:splittingfield} of $P$ over $K$.
  \begin{proof}
    First of all lets proof that if $Gal\left(P\right)$ is solvable
    then $P$ is solvable. Let $n = \left[F:K\right]$ and consider
    $L = K\left(\zeta_n\right)$ where $\zeta_n$ - $n$-th root of 1.
    Let $M = FL$ is a \nameref{def:compositeextension}. So this is the
    splitting and field of $P$ of which we have adjoined all the
    $n$-th roots of unity.  Then $M$ is a
    \nameref{def:galoisextension}
    \footnote{
      Accordingly property \ref{property:lec7_2} the composition of
      two Galois extensions will also be Galois.
    }
    and
    $Gal\left(M/L\right) \hookrightarrow Gal\left(F/K\right)$.
    \footnote{
      See theorem \ref{thm:lec7_3} where $L_1 = F, L_2 = L, L_1 L_2 =
      M = FL$.
    }
    $\forall g \in Gal\left(M/L\right)$ leaves $F$ invariant.
    \footnote{
      $g(F) = F$ see \nameref{thm:galoiscorrespondence} (point
      \ref{thm:galoiscorrespondence:item2b}) 
    }
    If $g\mid_F = id$ then $g = id$. Then the image in fact of this
    map is $Gal\left(F/F \cap L\right)$.
    \footnote{
      See theorem \ref{thm:lec7_3} where $L_1 = F, L_2 = L, L_1 L_2 =
      M$.
    }    
    So $G = Gal\left(M/L\right)$
    is solvable
    \footnote{
      Using property \ref{property:lec8_solvable1} $G$ is solvable as
      soon as $G \subset Gal\left(F/K\right)$ and 
      $Gal\left(F/K\right)$ is solvable
    }
    i.e. 
    \[
    G = G_0 \supset G_1 \supset \dots \supset G_r = \left\{e\right\}
    \]
    and $G_i/G_{i+1}$ - cyclic
    \footnote{
      see property \ref{property:lec8_solvable4}
    }
    of order $n_i \mid n$.
    \footnote{
      As soon as $G_i, G_{i+1} \subset G$ then using
      \nameref{thm:lagrange} theorem one can get that
      $\left|G_i\right| \mid \left|G\right|$ as well as
      $\left|G_{i+1}\right| \mid \left|G\right|$ and therefore
      $\left|G_i/G_{i+1}\right| \mid \left|G\right|$. But
      $G \subset Gal\left(F/K\right)$, thus
      $\left|G\right| \mid \left|Gal\left(F/K\right)\right| = n$ and, as
      result, $\left|G_i/G_{i+1}\right|$ divides $n$.
    }
    And as soon as
    $n_i \mid n$ (very important), all $n$-th roots of 1 are in $M$
    (this is why we adjoin the $L$).

    Let $M_i = M^{G_i}$. We know $M_i \hookrightarrow M_{i+1}$ is a
    cyclic Galois extension of order $n_i \mid n$ and roots of 1 are
    in it, thus there is Kummer extension.
    \footnote{
      We have $M_i = M^{G_i}$ where
      \[
      M_0 = M^{G_0} = M^G = M^{Gal\left(M/L\right)} = L
      \]
      and
      \[
      M_r = M^{G_r} = M^{\{e\}} = M.
      \]
      Thus by \nameref{thm:galoiscorrespondence},
      as soon as
      $G_r \triangleleft G_{r-1} \triangleleft \dots \triangleleft G_1
      \triangleleft G_0 = G$,
      we have the
      following ``tower'':
      \[
      L = M_0 \subset M_1 \subset \dots \subset M_{r-1} \subset M_r = M.
      \]
      If we consider $M_i \subset M_{i+1}$ then we can get the
      following: $M_i$ contains all $n$-th roots of unity as soon as
      $K\left(\zeta_n\right) = L \subset M_i$; $M_{i+1}$ is
      \nameref{def:galoisextension} over $M_i$ with
      \nameref{def:galoisgroup} $G_i/G_{i+1}$
      (we have $G_i = Gal\left(M/M_i\right)$,
      $G_{i+1} = Gal\left(M/M_{i+1}\right)$ and therefore
      $G_i/G_{i+1} = Gal\left(M_{i+1}/M_i\right)$)
      which is cyclic of order
      $n_i \mid n$ and as
      result $M_{i+1}$ will be also be cyclic of the same order $n_i
      \mid n$. Therefore $M_{i+1}$ is a Kummer extension 
      (see section \ref{sec:kummerextension}) over $M_i$,
    }
    and therefore
    $M_{i+1} = M_i\left(\sqrt[n_i]{a_i}\right)$ (see proposition
    \ref{prop:lec7_2}).
    So $M = K\left(\zeta_n, \alpha_1, \dots, \alpha_r\right)$ where
    $\alpha_i = \sqrt[n_i]{a_i}$. Therefore $M$ is solvable by
    radicals.

    For another direction: if $P$ is solvable then $G$ is
    solvable. Let $E$ is solvable extension containing $F$. We may
    suppose (using property \ref{property:solvable}) that this is Galois.
    Then write
    $E = K\left(\alpha_1, \dots, \alpha_r\right)$ where
    $\alpha_i^{n_i} \in K\left(\alpha_1, \dots,
    \alpha_{i-1}\right)$. Then let $L = K\left(\zeta_n\right)$
    where $n = LCM\left(\left\{n_i\right\}\right)$ so
    $\forall n_i: n_i \mid n$. And
    take $M = LE$. We have
    $K\left(\alpha_1, \dots, \alpha_{i-1}\right) \hookrightarrow
    K\left(\alpha_1, \dots, \alpha_i\right)$ - cyclic extension of
    order $n_i$. We have $Gal\left(M/L\right)$ is solvable by this
    cyclic subgroups. $Gal\left(M/K\right)$ is also solvable since
    \[
    Gal\left(M/L\right) \subset Gal\left(M/K\right)
    \]
    and the quotient
    \[
    \frac{Gal\left(M/K\right)}{Gal\left(M/L\right)}
    \cong Gal\left(L/K\right)
    \]
    which is abelian.
    \footnote{
      as result the property \ref{property:lec8_solvable3} gives us
      the solvability of $Gal\left(M/K\right)$
    }
    $Gal\left(F/K\right)$ is a quotient
    \footnote{
      ??? subset
    }
    of $Gal\left(M/K\right)$ thus
    it is solvable too.
    \footnote{
      by the property \ref{property:lec8_solvable1}
    }    
  \end{proof}
  \label{thm:lec8_1}
\end{theorem}

\section{Examples of equations not solvable by radicals."General
  equation"}
As we can see there exist equations which are not solvable in
radicals.

\begin{example}[Not solvable polynomial of degree 5]
  Let $P \in \mathbb{Q}\left[X\right]$ is an irreducible polynomial
  with rational 
  coefficients of degree 5. It has 3 real roots (and 2 complex
  conjugate roots) as it shown on the picture.
  
  \begin{tikzpicture}
    \draw[->] (-1.5,0) -- (3,0) node[right] {$x$};
    \draw[->] (0,-1.5) -- (0,1.5) node[above] {$y$};
    \draw[scale=0.5,domain=-2.5:4.2,smooth,variable=\x,blue] plot
    ({\x},{0.01*(\x-1)*(\x-1)*(\x-1)*(\x-1)*(\x-1) - 0.5*(\x-1) + 0.01});
  \end{tikzpicture}

  We claim that $Gal\left(P\right) = S_5$. This is because
  \begin{enumerate}
    \item $Gal\left(P\right)$ contains the complex conjugation (we
      have 2 complex conjugated roots but \nameref{def:galoisgroup} is
      the group of automorphisms which exchange roots and the complex
      conjugation will exchange the 2 complex roots).
      The complex conjugation is the transposition of roots
      \footnote{
        $(1,2) = 1 \to 2 \to 1$
      }
    \item As soon as $P$ is irreducible then $Gal\left(P\right)$
      should act transitively (see definition \ref{def:transitive}) on
      roots (see theorem \ref{thm:lec5_3}). We have an irreducible
      polynomial. We can always send 
      one of its roots to another of its roots. We have this
      isomorphism of stem fields which extends to an automorphism of
      the splitting field. But, what is the subgroup of $S_5$, which adds
      transitively?

      $Gal\left(P\right) \subset S_5$ acts transitively. This means
      that $5 \mid \left|Gal\left(P\right)\right|$. That is because
      (see \nameref{thm:orbitstabilizertheorem})
      $\left|G\right| = \left|Orb\left(x\right)\right| \left|G_x\right|$
      ($Orb\left(x\right)$ - is the \nameref{def:orbit},
      $G_x$ - \nameref{def:stabilizersubgroup})
      but the orbit has 5 elements
      \footnote{
        $\left|Orb\left(x\right)\right| = 5$ because there are 5
        different roots that can be permuted by the Galois group.
      }
      and therefore 5 divides
      the cardinality of $G$. This means, by \nameref{cor:sylow} 
      theorems, that our group contains something of order 5. But only
      5-cycle has order 5 (see theorem
      \ref{thm:finite_group_of_prime_order}). 
      But a 5-cycle and transposition generate
      $S_5$ (see corollary \ref{cor:sn}).
      So $Gal\left(P\right) = S_5$.
  \end{enumerate}
  In fact, the same argument is valid for $S_p$ with every $p$ - prime. 
  I.e. applies to an arbitrary prime number $p$ instead of 5.

  So $Gal\left(P\right) = S_5$ - not solvable and therefore $P$ is not
  solvable by radicals.
  \label{ex:lec8_notsolvable1}
\end{example}

\begin{example}[General equation of degree $n$]
  What's the general equation. It is the following
  \[
  X^n - T_1 X^{n-1} + T_2 X^{n-2} + \dots + \left(-1\right)^n T_n,
  \]
  where $T_i$ is a variable. Where does it come from? Let
  $X_1, \dots, X_n$ are roots of a polynomial of degree $n$ when the
  polynomial itself is
  \begin{eqnarray}
  \left(X - X_1\right) \cdot \dots \cdot \left(X - X_n\right) =
  X^n - \left(\sum_i X_i \right) X^{n-1} +
  \nonumber \\
  +
  \left(\sum_{i,j} X_i X_j \right) X^{n-2} + \dots +
  \left(-1\right)^n \prod_i X_i,
  \nonumber
  \end{eqnarray}
  i.e.
  $T_1 = \sum_i X_i, T_2 = \sum_{i,j} X_i X_j, \dots, T_n = \prod_i
  X_i$.

  One has $K\left[T_1, \dots, T_n\right] \subset K\left[X_1, \dots,
    X_n\right]$ (multi-variable polynomial rings). We have the same
  also for field extensions:
  $K\left(T_1, \dots, T_n\right) \subset K\left(X_1, \dots,
  X_n\right)$.
  The $K\left(X_1, \dots, X_n\right)$ is algebraic and a
  splitting field for our general polynomial. So it has degree at most
  $n!$, i.e.
  $\left[K\left(X_1, \dots, X_n\right):K\right] \le n!$
  (see theorem \ref{thm:lec2_1}). On the other hand
  $K\left(T_1, \dots, T_n\right) \subset K\left(X_1, \dots,
  X_n\right)^{S_n}$
  \footnote{
    This is because $S_n$ permutes the roots, for instance $X_k \to X_l \to
    X_k$, but the equations for $T_i$ are not changed during the
    transformation. 
  }
  so degree of the extension is $n!$
  \footnote{
    We have that
    $S_n \subset Gal\left(K\left(X_1, \dots, X_n\right)/
    K\left(T_1, \dots, T_n\right)\right)$ and therefore
    \[
    \left[K\left(X_1, \dots, X_n\right):
      K\left(T_1, \dots, T_n\right)\right] \ge \left|S_n\right| = n!.
    \]
    Thus with $\left[K\left(X_1, \dots, X_n\right):K\right] \le n!$
    one can get that
    $\left[K\left(X_1, \dots, X_n\right):
      K\left(T_1, \dots, T_n\right)\right] = n!$
    and
    $\left[K\left(X_1, \dots, X_n\right):K\right] = n!$.
  }
  and
  \[
  K\left(T_1, \dots, T_n\right) = K\left(X_1, \dots,
  X_n\right)^{S_n}.
  \]
  In particular the Galois group is $S_n$ and our
  general polynomial is not solvable by radicals if $n \ge 5$.
  This is known as Abel theorem
  \label{ex:lec8_generalequation}
\end{example}

\section{Galois action as a representation. Normal base theorem}

Connection with group representations.
\begin{definition}[Group representation]
  Let $G$ is a finite group. $V$ is a \nameref{def:vectorspace} over
  $K$. Representation of $G$ is a \nameref{def:homomorphism}
  $\rho: G \to GL\left(V\right)$
  (where $GL\left(V\right)$ is the \nameref{def:glv} i.e. the group of
  \nameref{def:automorphism}s of the vector space $V$).
  \label{def:grouprepresentation}
\end{definition}

If $L$ is a finite extension of $K$ we can talk about it as about
$K$-vector space. So we have a representation of $G$ as
\nameref{def:galoisgroup} $Gal\left(L/K\right)$:
$\rho: G \to GL_K\left(L\right)$ - this is something that we have as
the definition because we define the Galois group as the group of
automorphisms of $L$ over $K$.

We can ask the question: what's kind of representation is the
$\rho$. We claim that $\rho$ is something that is called as
\nameref{def:regularrepresentation}.
\begin{definition}[Regular representation]
  Let a vector space $V$ has a basis indexed by elements of group $G$:
  $e_g$ where $g \in G$. $\rho_{reg}\left(h\right)$
  \footnote{
    where $h \in G$.
  }
  acts by
  permutations:
  \footnote{
    $\rho_{reg}\left(h\right)$ is an \nameref{def:automorphism} of $V$
    i.e.
    \[
    \rho_{reg}\left(h\right): V \xrightarrow[e_g \to e_{gh}]{} V.
    \]
  }
  \[
  \rho_{reg}\left(h\right) e_g = e_{hg}.
  \]
  \label{def:regularrepresentation}
\end{definition}

We claim that the representation of Galois group is the regular
representation. We have seen that (see proof of the theorem
\ref{thm:primitiveelement}) 
\[
L \otimes_K \bar{K} \cong \bar{K}^n.
\]

The $\bar{K}^n$ is a \nameref{def:directsum} of $n$
($n = \left|G = Gal\left(L/K\right)\right|$)
copies of $\bar{K}$.
The sum is indexed by the embeddings of $L$ into
$\bar{K}$.
\footnote{
  The embeddings are also noticed as the set of 
  homomorphisms $Hom_K\left(L, \bar{K}\right)$.
}
Pick one $j: L \hookrightarrow \bar{K}$ and all others can
be obtained by group \nameref{def:action} $j \circ g, g \in G$.
\footnote{
  This is because the \nameref{def:galoisgroup} acts transitively on
  the set of homomorphisms from $L$ to $\bar{K}$ ($Hom_K\left(L,
  \bar{K}\right)$), see theorem \ref{thm:lec5_3}. 
}
So
$\bar{K}^n$ has a basis indexed by $G$ and the \nameref{def:action} of
$G$ permutes the basis vectors. So $L \otimes_K \bar{K} \cong
\bar{K}^n \cong$ \nameref{def:regularrepresentation} of $G$ over
$\bar{K}$.
\footnote{
  Each item in sum $\bar{K}^n$ can be associated with a base vector,
  for instance the first item is related to $e_1 = (1,0,0,0,\dots)$,
  the second one is related to $e_2 = (0,1,0,0,\dots)$, etc. as result
  each homomorphism $j \in Hom_K\left(L,\bar{K}\right)$ can be
  associated with an item in sum $\bar{K}^n$ and as result with a base
  vector. But we have that $Gal\left(L/K\right)$ can be used to
  produce any $j \in Hom_K\left(L,\bar{K}\right)$. Therefore we can
  associate any base vector (via $j$) with an element of
  $Gal\left(L/K\right)$.
}
In particular $\exists x \in L \otimes_K \bar{K}$
such that $gx \mid_{g \in G}$
forms a basis of $L \otimes_K \bar{K}$ over $\bar{K}$.
\footnote{
  Galois group $G = Gal\left(L/K\right)$ acts on
  $L \otimes_K \bar{K}$ as $g \otimes id_{\bar{K}}$ where $g \in G$.
  (see for example proof of theorem \ref{thm:basechange})
  ??? let $l \in L$ such that $g(l)$ forms a basis in $L$ (???) then,
  accordingly proposition \ref{prop:lec4_Addon},
  $\{g(l) \otimes_K 1_{\bar{K}}\}$ forms a $\bar{K}$-basis.
} And, as result, the elements of $G$ are linearly independent in the space of
\nameref{def:endomorphism}s
$End_{\bar{K}}\left(L \otimes_K \bar{K}\right)$
\footnote{
  ??? see theorem \ref{thm:dedekind} about linearly independents of
  elements of a group $G$
}

\begin{theorem}[Normal base]
  $\exists x \in L$ such that
  $\left\{gx \mid g \in G\right\}$ is a $K$ basis of $L$.

  $G$ are
  linearly independent in the space of \nameref{def:endomorphism}s
  $End_{K}\left(L\right)$
  \begin{proof}
    First of all consider a case when $K$ is infinite. Let pick some
    basis $e_1, \dots, e_n$ - $K$-basis in $L$. $g_1, \dots, g_n \in
    G$. Let $x \in L$ then $g_1\left(x\right), \dots,
    g_n\left(x\right)$ is a basis if and only if matrix formed by
    $g_i\left(x\right)$ in the basis $e_j$ has non zero
    determinant.
    \footnote{
      We have $x = X_1 e_1 + \dots X_n e_n$ where $X_i \in K$.
      $\forall g_l \in G: g_l(x) = \sum_i X_i g_l(e_i)$
      because $g_l(X_i) = X_i$ (remember that $L^G = K$).
      For each $g_l(e_i)$ we can write
      \[
      g_l(e_i) = \sum_j k_{ij}^{(l)} e_j
      \]      
      We have $k_{ij}^{(l)}$ are predefined
      (because $\{g_l\}$ and $\{e_i\}$ are predefined)
      elements of $L$ and therefore $X_i$ can only be changed.

      We want $\{g_j(x)\}$ to be linearly independent set of elements
      from $L$. Therefore the equation
      \[
      c_1 g_1(x) + \dots c_n g_n(x) = 0,
      \]
      where $c_i \in K$, should holds only for $c_1 = c_2 = \dots =
      c_n = 0$.
      We also can write
      \begin{eqnarray}
      c_1 \sum_j \left(\sum_i X_i k_{ij}^{(1)} \right) e_j  +
      c_2 \sum_j \left(\sum_i X_i k_{ij}^{(2)} \right) e_j  +
      \dots
      +
      \nonumber \\
      +
      c_n \sum_j \left(\sum_i X_i k_{ij}^{(n)} \right) e_j  =
      \sum_j 0 e_j = 0
      \nonumber
      \end{eqnarray}
      The equation can also be rewritten in the matrix form
      \begin{eqnarray}
        \begin{bmatrix}
          \sum_i X_i k_{i1}^{(1)} & \sum_i X_i k_{i1}^{(2)} & \cdots &
          \sum_i X_i k_{i1}^{(n)} \\
          \sum_i X_i k_{i2}^{(1)} & \sum_i X_i k_{i2}^{(2)} & \cdots &
          \sum_i X_i k_{i2}^{(n)} \\
          \vdots & \vdots & \ddots & \vdots \\
          \sum_i X_i k_{in}^{(1)} & \sum_i X_i k_{in}^{(2)} & \cdots &
          \sum_i X_i k_{in}^{(n)} \\
        \end{bmatrix}
        \begin{bmatrix}
          c_1 \\
          c_2 \\
          \vdots \\
          c_n
        \end{bmatrix}
        =      
        \begin{bmatrix}
          0 \\
          0 \\
          \vdots \\
          0
        \end{bmatrix}
        \nonumber
      \end{eqnarray}
      The equation has non-trivial solutions only if
      \begin{equation}
      \det\left(
        \begin{bmatrix}
          \sum_i X_i k_{i1}^{(1)} & \sum_i X_i k_{i1}^{(2)} & \cdots &
          \sum_i X_i k_{i1}^{(n)} \\
          \sum_i X_i k_{i2}^{(1)} & \sum_i X_i k_{i2}^{(2)} & \cdots &
          \sum_i X_i k_{i2}^{(n)} \\
          \vdots & \vdots & \ddots & \vdots \\
          \sum_i X_i k_{in}^{(1)} & \sum_i X_i k_{in}^{(2)} & \cdots &
          \sum_i X_i k_{in}^{(n)} \\
        \end{bmatrix}      
        \right) \ne 0.
        \label{eq:lec8_note_det}
      \end{equation}
    }
    But this determinant is a polynomial
    in the
    coefficient, which is not identically zero.
    \footnote{
      The determinant(\ref{eq:lec8_note_det}) can always be written in
      the form of 
      multi-variable polynomial
      $P \in L\left[X_1, \dots, X_n\right]$ i.e. the polynomial
      coefficients are from $L$.  
    }
    Well, why? Because if
    it was identically zero, it would remain identically zero also
    after the base changed to $\bar{K}$. Since it has a $\bar{K}$
    point where it does not vanish.

    There are many $x \in L \otimes_K \bar{K}$ such that
    $g_i\left(x\right)$ form a basis. And over an infinite field, a
    polynomial which is not identically zero cannot vanish
    identically ($P \ne 0$). And over an infinite
    field, only a polynomial which 
    is identically 0 can vanish at every point.
    \footnote{
      I.e. polynomial of the following polynomial ring
      $P \in L\left[X_1, \dots, X_n\right]$ can vanish at every point
      only if it is identically zero $P = 0$.
    }

    Let me to clarify the point. $P \in K\left[X\right]$ has at most
    $\deg P$ roots.
    \footnote{
      Therefore it is impossible to have each $k \in K$ as a root of
      the polynomial because in the case the number of roots
      $\left|K\right| = \infty > \deg P$.
    }
    So if $K$ infinite and $P$ has every element of
    $K$ as a root then $P = 0$ ($P$ is zero as an element of
    $K\left[X\right]$).

    By induction we can get the same statement for a polynomial in
    several variables. So, our polynomial which is the determinant of
    the matrix,  
    is non zero, as a polynomial of several variable because it 
    does not have roots over algebraic closure $\bar{K}$. 
    And so, it also does not have roots over $K$. So, there
    exists a point $x \in L$ (not anymore in $L \otimes_K \bar{K}$)
    such that $\det(\dots) \ne 0$ at $x$ so $g_i\left(x\right)$ form a
    basis.

    If $K$ is finite then the argument with roots of a polynomial does
    not apply any more. But in the case the \nameref{def:galoisgroup} is
    cyclic i.e. $G = \left<\sigma\right>$ (see corollary
    \ref{cor:lec3_2}). 
    We have $id, \sigma, \dots,
    \sigma^{n-1}$ are linearly independent because they are linearly
    independent  over
    $\bar{K}$.
    \footnote{
      ???
      }
    Then the minimal polynomial of $\sigma$ as an
    \nameref{def:endomorphism} of $L$ over $K$ is $X^n - 1$.
    This is because a lower degree polynomial cannot vanish at
    $\sigma$
    since, the lower power of $\sigma$ are linearly independent.  
    Thus
    \[
    L \cong K\left[X\right]/\left(X^n - 1\right)
    \]
    as a $K\left[X\right]$-module with $X$ acting by $\sigma$. This is
    a \nameref{def:cyclicmodule}
    \footnote{
      ??? $ = \left<x\right>$
    }
    and any generator $x$ shall do i.e. $x, \sigma x, \dots,
    \sigma^{n-1} x$ form a basis.
  \end{proof}
  \label{thm:normalbase}
\end{theorem}

\section{Relation with coverings}

\begin{remark}
  If $L$ is a finite \nameref{def:galoisextension} of $K$ then
  $ L \otimes_K L$ is a direct sum of fields
  \footnote{
    see also definition \ref{def:directsummodules}
  }
  which are
  isomorphic to $L$. Sums are permuted by $G = Gal\left(L/K\right)$.
  \begin{proof}
  So if $L = K\left(\alpha\right)$ is a splitting field of the
  polynomial $P = \left(X - \alpha_1\right) \cdot \dots \cdot \left(X
  - \alpha_n\right)$ (where $\alpha \in \{\alpha_1, \dots,
  \alpha_n\}$) that is isomorphic to $K\left[X\right]/(P)$. If 
  we tensor it to $L$ we will get (see examples \ref{ex:lec4_1} and
  \ref{ex:lec5_1}) 
  \begin{eqnarray}
    L \otimes_K L \cong
    L \otimes_K K\left[X\right]/\left(X - \alpha_1\right) \cdot \dots \cdot \left(X
    - \alpha_n\right) \cong
    \nonumber \\
    \cong
  L\left[X\right]/\left(X - \alpha_1\right) \cdot \dots \cdot \left(X
  - \alpha_n\right) \cong
  \nonumber \\
  \cong
  L\left[X\right]/\left(X - \alpha_1\right) \times
  \dots \times
  L\left[X\right]/\left(X - \alpha_n\right)
  \nonumber
  \end{eqnarray}
  that is a product of copies of $L$ permuted by Galois action
  \end{proof}
  \label{rem:lec8_1}
\end{remark}

In topology one has Galois covering $Y \to X$. $G$ acts on $Y$, $X$
quotient. The covering is characterized by the property that $Y
\times_X Y = \sqcup_{g \in G} Y_g$ (is a disjoint union), $Y_g =
\{\left(y, gy\right)\}$. 
\footnote{
  ??? add an explanation
}
